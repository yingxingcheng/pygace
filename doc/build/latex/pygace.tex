%% Generated by Sphinx.
\def\sphinxdocclass{report}
\documentclass[letterpaper,10pt,english]{sphinxmanual}
\ifdefined\pdfpxdimen
   \let\sphinxpxdimen\pdfpxdimen\else\newdimen\sphinxpxdimen
\fi \sphinxpxdimen=.75bp\relax

\usepackage[utf8]{inputenc}
\ifdefined\DeclareUnicodeCharacter
 \ifdefined\DeclareUnicodeCharacterAsOptional
  \DeclareUnicodeCharacter{"00A0}{\nobreakspace}
  \DeclareUnicodeCharacter{"2500}{\sphinxunichar{2500}}
  \DeclareUnicodeCharacter{"2502}{\sphinxunichar{2502}}
  \DeclareUnicodeCharacter{"2514}{\sphinxunichar{2514}}
  \DeclareUnicodeCharacter{"251C}{\sphinxunichar{251C}}
  \DeclareUnicodeCharacter{"2572}{\textbackslash}
 \else
  \DeclareUnicodeCharacter{00A0}{\nobreakspace}
  \DeclareUnicodeCharacter{2500}{\sphinxunichar{2500}}
  \DeclareUnicodeCharacter{2502}{\sphinxunichar{2502}}
  \DeclareUnicodeCharacter{2514}{\sphinxunichar{2514}}
  \DeclareUnicodeCharacter{251C}{\sphinxunichar{251C}}
  \DeclareUnicodeCharacter{2572}{\textbackslash}
 \fi
\fi
\usepackage{cmap}
\usepackage[T1]{fontenc}
\usepackage{amsmath,amssymb,amstext}
\usepackage{babel}
\usepackage{times}
\usepackage[Bjarne]{fncychap}
\usepackage[dontkeepoldnames]{sphinx}

\usepackage{geometry}

% Include hyperref last.
\usepackage{hyperref}
% Fix anchor placement for figures with captions.
\usepackage{hypcap}% it must be loaded after hyperref.
% Set up styles of URL: it should be placed after hyperref.
\urlstyle{same}
\addto\captionsenglish{\renewcommand{\contentsname}{Contents:}}

\addto\captionsenglish{\renewcommand{\figurename}{Fig.}}
\addto\captionsenglish{\renewcommand{\tablename}{Table}}
\addto\captionsenglish{\renewcommand{\literalblockname}{Listing}}

\addto\captionsenglish{\renewcommand{\literalblockcontinuedname}{continued from previous page}}
\addto\captionsenglish{\renewcommand{\literalblockcontinuesname}{continues on next page}}

\addto\extrasenglish{\def\pageautorefname{page}}

\setcounter{tocdepth}{1}



\title{pygace Documentation}
\date{Dec 26, 2018}
\release{2018.12.13}
\author{Yingxing Cheng}
\newcommand{\sphinxlogo}{\vbox{}}
\renewcommand{\releasename}{Release}
\makeindex

\begin{document}

\maketitle
\sphinxtableofcontents
\phantomsection\label{\detokenize{index::doc}}



\chapter{pygace}
\label{\detokenize{modules::doc}}\label{\detokenize{modules:welcome-to-pygace-s-documentation}}\label{\detokenize{modules:pygace}}

\section{pygace package}
\label{\detokenize{pygace:pygace-package}}\label{\detokenize{pygace::doc}}

\subsection{Subpackages}
\label{\detokenize{pygace:subpackages}}

\subsubsection{pygace.examples package}
\label{\detokenize{pygace.examples::doc}}\label{\detokenize{pygace.examples:pygace-examples-package}}

\paragraph{Subpackages}
\label{\detokenize{pygace.examples:subpackages}}

\subparagraph{pygace.examples.general package}
\label{\detokenize{pygace.examples.general:pygace-examples-general-package}}\label{\detokenize{pygace.examples.general::doc}}

\subparagraph{Submodules}
\label{\detokenize{pygace.examples.general:submodules}}

\subparagraph{pygace.examples.general.test\_general module}
\label{\detokenize{pygace.examples.general:pygace-examples-general-test-general-module}}\label{\detokenize{pygace.examples.general:module-pygace.examples.general.test_general}}\index{pygace.examples.general.test\_general (module)}
Searching the most stable atomic-structure of a solid with point defects
(including the extrinsic alloying/doping elements), is one of the central issues in
materials science. Both adequate sampling of the configuration space and the
accurate energy evaluation at relatively low cost are demanding for the structure
prediction. In this work, we have developed a framework combining genetic
algorithm, cluster expansion (CE) method and first-principles calculations, which
can effectively locate the ground-state or meta-stable states of the relatively
large/complex systems. We employ this framework to search the stable structures
of two distinct systems, i.e., oxygen-vacancy-containing HfO(2-x) and the
Nb-doped SrTi(1-x)NbxO3 , and more stable structures are found compared with
the structures available in the literature. The present framework can be applied
to the ground-state search of extensive alloyed/doped materials, which is
particularly significant for the design of advanced engineering alloys and
semiconductors.


\subparagraph{Module contents}
\label{\detokenize{pygace.examples.general:module-contents}}\label{\detokenize{pygace.examples.general:module-pygace.examples.general}}\index{pygace.examples.general (module)}

\subparagraph{pygace.examples.hfo2 package}
\label{\detokenize{pygace.examples.hfo2::doc}}\label{\detokenize{pygace.examples.hfo2:pygace-examples-hfo2-package}}

\subparagraph{Submodules}
\label{\detokenize{pygace.examples.hfo2:submodules}}

\subparagraph{pygace.examples.hfo2.hfo2\_gace module}
\label{\detokenize{pygace.examples.hfo2:pygace-examples-hfo2-hfo2-gace-module}}\label{\detokenize{pygace.examples.hfo2:module-pygace.examples.hfo2.hfo2_gace}}\index{pygace.examples.hfo2.hfo2\_gace (module)}
A GA-to-CE example of oxygen-vacancy-containing HfO2 system given in
this module.
\index{HFO2App (class in pygace.examples.hfo2.hfo2\_gace)}

\begin{fulllineitems}
\phantomsection\label{\detokenize{pygace.examples.hfo2:pygace.examples.hfo2.hfo2_gace.HFO2App}}\pysiglinewithargsret{\sphinxbfcode{class }\sphinxcode{pygace.examples.hfo2.hfo2\_gace.}\sphinxbfcode{HFO2App}}{\emph{ce\_site=8}, \emph{ce\_dirname='./data/iter1'}, \emph{ele\_1st='O'}, \emph{ele\_2nd='Vac'}, \emph{params\_config\_dict=None}}{}
Bases: {\hyperref[\detokenize{pygace:pygace.gace.AbstractApp}]{\sphinxcrossref{\sphinxcode{pygace.gace.AbstractApp}}}}

An app of HfO(2-x) system which is implemented from AbstractApp object

This object is used to execute a GACE simulation, user only need to
implement several interfaces to custom their application.
\begin{quote}\begin{description}
\item[{Parameters}] \leavevmode\begin{description}
\item[{\sphinxstylestrong{ce\_site: int}}] \leavevmode
the concept of site used in ATAT program.

\item[{\sphinxstylestrong{ce\_dirname: str}}] \leavevmode
The name of a directory which contain information of MMAPS or MAPS
running

\item[{\sphinxstylestrong{ele\_1st: str}}] \leavevmode
The first type of element in the \sphinxcode{site} in \sphinxcode{ATAT}.

\item[{\sphinxstylestrong{ele\_2nd: str}}] \leavevmode
The second type of element in the \sphinxcode{site} in \sphinxcode{ATAT}.

\item[{\sphinxstylestrong{params\_config\_dict: dirt}}] \leavevmode
Parameter dict used to custom GACE AbstractApp.

\end{description}

\item[{Attributes}] \leavevmode\begin{description}
\item[{\sphinxstylestrong{app}}] \leavevmode{[}AbstractApp{]}
A subclass object of AbstractApp.

\item[{\sphinxstylestrong{iter\_idx}}] \leavevmode{[}int{]}
Index of GA-to-CE iteration.

\end{description}

\end{description}\end{quote}
\index{DEFAULT\_SETUP (pygace.examples.hfo2.hfo2\_gace.HFO2App attribute)}

\begin{fulllineitems}
\phantomsection\label{\detokenize{pygace.examples.hfo2:pygace.examples.hfo2.hfo2_gace.HFO2App.DEFAULT_SETUP}}\pysigline{\sphinxbfcode{DEFAULT\_SETUP}\sphinxbfcode{ = \{'MU\_OXYGEN': -4.91223, 'STEP': 1, 'TMP\_DIR': '/home/yxcheng/PycharmProjects/pygace/doc/tmp\_dir', 'NB\_DEFECT': 4, 'TEMPLATE\_FILE': './data/lat\_in.template', 'PERFECT\_HFO2': -976.3650933333331, 'DFT\_CAL\_DIR': './dft\_dirs', 'TEST\_RES\_DIR': '/home/yxcheng/PycharmProjects/pygace/doc/res\_dir', 'PICKLE\_DIR': '/home/yxcheng/PycharmProjects/pygace/doc/pickle\_bakup', 'NB\_SITES': 64\}}}
\end{fulllineitems}

\index{evalEnergy() (pygace.examples.hfo2.hfo2\_gace.HFO2App method)}

\begin{fulllineitems}
\phantomsection\label{\detokenize{pygace.examples.hfo2:pygace.examples.hfo2.hfo2_gace.HFO2App.evalEnergy}}\pysiglinewithargsret{\sphinxbfcode{evalEnergy}}{\emph{individual}}{}
Evaluation function for the ground-state searching problem.

The problem is to determine a configuration of n vacancies
on a crystalline structures such that the energy of crystalline
structures can obtain minimum value.
\begin{quote}\begin{description}
\item[{Parameters}] \leavevmode\begin{description}
\item[{\sphinxstylestrong{individual}}] \leavevmode
\end{description}

\item[{Returns}] \leavevmode\begin{description}
\item[{\sphinxstylestrong{float}}] \leavevmode
Fittness value

\end{description}

\end{description}\end{quote}

\end{fulllineitems}

\index{get\_epoch() (pygace.examples.hfo2.hfo2\_gace.HFO2App method)}

\begin{fulllineitems}
\phantomsection\label{\detokenize{pygace.examples.hfo2:pygace.examples.hfo2.hfo2_gace.HFO2App.get_epoch}}\pysiglinewithargsret{\sphinxbfcode{get\_epoch}}{\emph{nb\_vac}}{}
Obtain the epoch state of the previous running
\begin{quote}\begin{description}
\item[{Parameters}] \leavevmode\begin{description}
\item[{\sphinxstylestrong{nb\_vac}}] \leavevmode{[}int{]}
The number of vacancy

\end{description}

\item[{Returns}] \leavevmode\begin{description}
\item[{\sphinxstylestrong{int}}] \leavevmode
The epoch state of previous running.

\end{description}

\end{description}\end{quote}

\end{fulllineitems}

\index{ind\_to\_elis() (pygace.examples.hfo2.hfo2\_gace.HFO2App method)}

\begin{fulllineitems}
\phantomsection\label{\detokenize{pygace.examples.hfo2:pygace.examples.hfo2.hfo2_gace.HFO2App.ind_to_elis}}\pysiglinewithargsret{\sphinxbfcode{ind\_to\_elis}}{\emph{individual}}{}
Convert individual (number list) to element list
\begin{quote}\begin{description}
\item[{Parameters}] \leavevmode\begin{description}
\item[{\sphinxstylestrong{individual}}] \leavevmode
\end{description}

\item[{Returns}] \leavevmode\begin{description}
\item[{\sphinxstylestrong{list}}] \leavevmode
A list of element symbol string.

\end{description}

\end{description}\end{quote}

\end{fulllineitems}

\index{multiple\_run() (pygace.examples.hfo2.hfo2\_gace.HFO2App method)}

\begin{fulllineitems}
\phantomsection\label{\detokenize{pygace.examples.hfo2:pygace.examples.hfo2.hfo2_gace.HFO2App.multiple_run}}\pysiglinewithargsret{\sphinxbfcode{multiple\_run}}{\emph{mission\_name}, \emph{repeat\_iters}}{}
For multiple tasks
\begin{quote}\begin{description}
\item[{Parameters}] \leavevmode\begin{description}
\item[{\sphinxstylestrong{mission\_name}}] \leavevmode{[}str{]}
A string used to represent the name of current running.

\item[{\sphinxstylestrong{repeat\_iters}}] \leavevmode{[}int{]}
How many times should a simulation repeat for statistic error(
different grouond-state in different running iterations with
identical parameter setting).

\end{description}

\item[{Returns}] \leavevmode\begin{description}
\item[{\sphinxstylestrong{list}}] \leavevmode
A list contain the results of running

\end{description}

\end{description}\end{quote}

\end{fulllineitems}

\index{run() (pygace.examples.hfo2.hfo2\_gace.HFO2App method)}

\begin{fulllineitems}
\phantomsection\label{\detokenize{pygace.examples.hfo2:pygace.examples.hfo2.hfo2_gace.HFO2App.run}}\pysiglinewithargsret{\sphinxbfcode{run}}{\emph{iter\_idx=1}, \emph{target\_epoch=50}}{}
Main function to run a GACE simulation which will be called by
\sphinxtitleref{AbstractRunner}.
\begin{quote}\begin{description}
\item[{Parameters}] \leavevmode\begin{description}
\item[{\sphinxstylestrong{iter\_idx}}] \leavevmode{[}int{]}
Determine which iteration the ECI is used in.

\item[{\sphinxstylestrong{target\_epoch}}] \leavevmode{[}int{]}
Iteration in GA simulation.

\end{description}

\item[{Returns}] \leavevmode\begin{description}
\item[{\sphinxstylestrong{None}}] \leavevmode
\end{description}

\end{description}\end{quote}

\end{fulllineitems}

\index{single\_run() (pygace.examples.hfo2.hfo2\_gace.HFO2App method)}

\begin{fulllineitems}
\phantomsection\label{\detokenize{pygace.examples.hfo2:pygace.examples.hfo2.hfo2_gace.HFO2App.single_run}}\pysiglinewithargsret{\sphinxbfcode{single\_run}}{\emph{mission\_name}, \emph{repeat\_iter}}{}
A single running task.
\begin{quote}\begin{description}
\item[{Parameters}] \leavevmode\begin{description}
\item[{\sphinxstylestrong{mission\_name}}] \leavevmode{[}str{]}
A string used to represent the name of current running.

\item[{\sphinxstylestrong{repeat\_iters}}] \leavevmode{[}int{]}
How many times should a simulation repeat for statistic error(
different grouond-state in different running iterations with
identical parameter setting).

\end{description}

\item[{Returns}] \leavevmode\begin{description}
\item[{\sphinxstylestrong{tuple}}] \leavevmode
A tuple of pupulation, random states and hall of fame.

\end{description}

\end{description}\end{quote}

\end{fulllineitems}

\index{update\_ce() (pygace.examples.hfo2.hfo2\_gace.HFO2App method)}

\begin{fulllineitems}
\phantomsection\label{\detokenize{pygace.examples.hfo2:pygace.examples.hfo2.hfo2_gace.HFO2App.update_ce}}\pysiglinewithargsret{\sphinxbfcode{update\_ce}}{\emph{site=8}, \emph{dirname='./data/iter1'}}{}~\begin{quote}\begin{description}
\item[{Parameters}] \leavevmode\begin{description}
\item[{\sphinxstylestrong{site}}] \leavevmode{[}int, optional{]}
The site used in cluster expansion.

\item[{\sphinxstylestrong{dirname}}] \leavevmode{[}str, optional{]}
The name of directory which contains file required in cluster
expansion.

\end{description}

\item[{Returns}] \leavevmode\begin{description}
\item[{\sphinxstylestrong{None}}] \leavevmode
\end{description}

\end{description}\end{quote}

\end{fulllineitems}


\end{fulllineitems}

\index{HfO2EleIndv (class in pygace.examples.hfo2.hfo2\_gace)}

\begin{fulllineitems}
\phantomsection\label{\detokenize{pygace.examples.hfo2:pygace.examples.hfo2.hfo2_gace.HfO2EleIndv}}\pysiglinewithargsret{\sphinxbfcode{class }\sphinxcode{pygace.examples.hfo2.hfo2\_gace.}\sphinxbfcode{HfO2EleIndv}}{\emph{ele\_lis}, \emph{app=None}}{}
Bases: {\hyperref[\detokenize{pygace:pygace.utility.EleIndv}]{\sphinxcrossref{\sphinxcode{pygace.utility.EleIndv}}}}

A class that use list chemistry element to represent individual.
\begin{quote}\begin{description}
\item[{Parameters}] \leavevmode\begin{description}
\item[{\sphinxstylestrong{ele\_lis}}] \leavevmode{[}list{]}
A list of chemistry element.

\item[{\sphinxstylestrong{app}}] \leavevmode{[}AbstractApp{]}
An application of GACE which is used to obtain ground-state
structures based generic algorithm and cluster expansion method.

\end{description}

\item[{Attributes}] \leavevmode\begin{description}
\item[{\sphinxstylestrong{app: AbstractApp}}] \leavevmode
An application handling GACE running process.

\item[{\sphinxstylestrong{ele\_lis: list}}] \leavevmode
A list of chemistry element string.

\end{description}

\end{description}\end{quote}
\index{ce\_energy (pygace.examples.hfo2.hfo2\_gace.HfO2EleIndv attribute)}

\begin{fulllineitems}
\phantomsection\label{\detokenize{pygace.examples.hfo2:pygace.examples.hfo2.hfo2_gace.HfO2EleIndv.ce_energy}}\pysigline{\sphinxbfcode{ce\_energy}}
Return CE energy
\begin{quote}\begin{description}
\item[{Returns}] \leavevmode\begin{description}
\item[{\sphinxstylestrong{float}}] \leavevmode
Energy predicted by CE method.

\end{description}

\end{description}\end{quote}

\end{fulllineitems}

\index{ce\_energy\_corrdump (pygace.examples.hfo2.hfo2\_gace.HfO2EleIndv attribute)}

\begin{fulllineitems}
\phantomsection\label{\detokenize{pygace.examples.hfo2:pygace.examples.hfo2.hfo2_gace.HfO2EleIndv.ce_energy_corrdump}}\pysigline{\sphinxbfcode{ce\_energy\_corrdump}}
Return relative energy defined in \sphinxcode{ATAT} and computed by \sphinxcode{corrdump}
program.
\begin{quote}\begin{description}
\item[{Returns}] \leavevmode\begin{description}
\item[{\sphinxstylestrong{float}}] \leavevmode
Relative energy generated by \sphinxcode{corrdump} program.

\end{description}

\end{description}\end{quote}

\end{fulllineitems}

\index{dft\_energy() (pygace.examples.hfo2.hfo2\_gace.HfO2EleIndv method)}

\begin{fulllineitems}
\phantomsection\label{\detokenize{pygace.examples.hfo2:pygace.examples.hfo2.hfo2_gace.HfO2EleIndv.dft_energy}}\pysiglinewithargsret{\sphinxbfcode{dft\_energy}}{\emph{iters=None}, \emph{vasp\_cmd=None}, \emph{update\_eci=True}}{}
Return DFT energy
\begin{quote}\begin{description}
\item[{Parameters}] \leavevmode\begin{description}
\item[{\sphinxstylestrong{iters}}] \leavevmode{[}int{]}
index of iteration of GA-to-CE

\end{description}

\item[{Returns}] \leavevmode\begin{description}
\item[{\sphinxstylestrong{None or float}}] \leavevmode
\end{description}

\end{description}\end{quote}

\end{fulllineitems}

\index{dft\_energy\_deprecated() (pygace.examples.hfo2.hfo2\_gace.HfO2EleIndv method)}

\begin{fulllineitems}
\phantomsection\label{\detokenize{pygace.examples.hfo2:pygace.examples.hfo2.hfo2_gace.HfO2EleIndv.dft_energy_deprecated}}\pysiglinewithargsret{\sphinxbfcode{dft\_energy\_deprecated}}{\emph{iters=None}}{}
Return DFT energy
\begin{quote}\begin{description}
\item[{Parameters}] \leavevmode\begin{description}
\item[{\sphinxstylestrong{iters}}] \leavevmode{[}int{]}
index of iteration of GA-to-CE

\end{description}

\item[{Returns}] \leavevmode\begin{description}
\item[{\sphinxstylestrong{None or float}}] \leavevmode
\end{description}

\end{description}\end{quote}

\end{fulllineitems}


\end{fulllineitems}

\index{Runner (class in pygace.examples.hfo2.hfo2\_gace)}

\begin{fulllineitems}
\phantomsection\label{\detokenize{pygace.examples.hfo2:pygace.examples.hfo2.hfo2_gace.Runner}}\pysiglinewithargsret{\sphinxbfcode{class }\sphinxcode{pygace.examples.hfo2.hfo2\_gace.}\sphinxbfcode{Runner}}{\emph{app=None}, \emph{iter\_idx=None}}{}
Bases: {\hyperref[\detokenize{pygace:pygace.gace.AbstractRunner}]{\sphinxcrossref{\sphinxcode{pygace.gace.AbstractRunner}}}}

A runner for running a GACE simulation.

This object is used to execute a GACE simulation in HfO2 system.
\begin{quote}\begin{description}
\item[{Parameters}] \leavevmode\begin{description}
\item[{\sphinxstylestrong{app}}] \leavevmode{[}subclass of HFO2App{]}
A subclass object of HFO2App, default is \sphinxtitleref{None}.

\item[{\sphinxstylestrong{iter\_idx}}] \leavevmode{[}int{]}
Index of GA-to-CE iteration, default is \sphinxtitleref{None}.

\end{description}

\item[{Attributes}] \leavevmode\begin{description}
\item[{\sphinxstylestrong{app}}] \leavevmode{[}HFO2App{]}
A subclass object of HFO2App.

\item[{\sphinxstylestrong{iter\_idx}}] \leavevmode{[}int{]}
Index of GA-to-CE iteration.

\end{description}

\end{description}\end{quote}
\index{compare\_gs() (pygace.examples.hfo2.hfo2\_gace.Runner method)}

\begin{fulllineitems}
\phantomsection\label{\detokenize{pygace.examples.hfo2:pygace.examples.hfo2.hfo2_gace.Runner.compare_gs}}\pysiglinewithargsret{\sphinxbfcode{compare\_gs}}{\emph{new\_gs}, \emph{old\_gs}}{}
Determine whether current and previous ground-state are identical.
\begin{quote}\begin{description}
\item[{Parameters}] \leavevmode\begin{description}
\item[{\sphinxstylestrong{new\_gs}}] \leavevmode{[}str{]}
Ground-state configuration predicted by current iteration.

\item[{\sphinxstylestrong{old\_gs :}}] \leavevmode
Ground-state configuration predicted by previous iteration.

\end{description}

\item[{Returns}] \leavevmode\begin{description}
\item[{\sphinxstylestrong{bool}}] \leavevmode
\item[{\sphinxstylestrong{Raises:}}] \leavevmode
\item[{\sphinxstylestrong{RuntimeError}}] \leavevmode
when the number of point defect (oxygen vacancy here) is not equal
in two iteration.

\end{description}

\end{description}\end{quote}

\end{fulllineitems}

\index{print\_gs() (pygace.examples.hfo2.hfo2\_gace.Runner method)}

\begin{fulllineitems}
\phantomsection\label{\detokenize{pygace.examples.hfo2:pygace.examples.hfo2.hfo2_gace.Runner.print_gs}}\pysiglinewithargsret{\sphinxbfcode{print\_gs}}{}{}
Function used to extract ground-state information from pickle file
saved during GACE running.
\begin{quote}\begin{description}
\item[{Returns}] \leavevmode\begin{description}
\item[{\sphinxstylestrong{None}}] \leavevmode
\end{description}

\end{description}\end{quote}

\end{fulllineitems}

\index{run() (pygace.examples.hfo2.hfo2\_gace.Runner method)}

\begin{fulllineitems}
\phantomsection\label{\detokenize{pygace.examples.hfo2:pygace.examples.hfo2.hfo2_gace.Runner.run}}\pysiglinewithargsret{\sphinxbfcode{run}}{}{}
Main function to run.
\begin{quote}\begin{description}
\item[{Returns}] \leavevmode\begin{description}
\item[{\sphinxstylestrong{None}}] \leavevmode
\end{description}

\end{description}\end{quote}

\end{fulllineitems}

\index{str2energy() (pygace.examples.hfo2.hfo2\_gace.Runner method)}

\begin{fulllineitems}
\phantomsection\label{\detokenize{pygace.examples.hfo2:pygace.examples.hfo2.hfo2_gace.Runner.str2energy}}\pysiglinewithargsret{\sphinxbfcode{str2energy}}{\emph{string}}{}
Obtain energy from string consists of numbers joined by ‘\_’, e.g.,
\sphinxcode{'1\_2\_3\_19\_'}, in which the number is the position index in lattice
structure template file.
\begin{quote}\begin{description}
\item[{Parameters}] \leavevmode\begin{description}
\item[{\sphinxstylestrong{string}}] \leavevmode{[}str{]}
The string consists by index of point defect.

\item[{\sphinxstylestrong{Returns}}] \leavevmode
\item[{\sphinxstylestrong{——-}}] \leavevmode
\item[{\sphinxstylestrong{float}}] \leavevmode
CE energy.

\end{description}

\end{description}\end{quote}

\end{fulllineitems}


\end{fulllineitems}



\subparagraph{Module contents}
\label{\detokenize{pygace.examples.hfo2:module-contents}}\label{\detokenize{pygace.examples.hfo2:module-pygace.examples.hfo2}}\index{pygace.examples.hfo2 (module)}
A GA-to-CE example of oxygen-vacancy containing HfO2 given in this module.


\subparagraph{pygace.examples.sto package}
\label{\detokenize{pygace.examples.sto:pygace-examples-sto-package}}\label{\detokenize{pygace.examples.sto::doc}}

\subparagraph{Submodules}
\label{\detokenize{pygace.examples.sto:submodules}}

\subparagraph{pygace.examples.sto.sto\_gace module}
\label{\detokenize{pygace.examples.sto:pygace-examples-sto-sto-gace-module}}\label{\detokenize{pygace.examples.sto:module-pygace.examples.sto.sto_gace}}\index{pygace.examples.sto.sto\_gace (module)}
A GA-to-CE example given in this module.
\index{Runner (class in pygace.examples.sto.sto\_gace)}

\begin{fulllineitems}
\phantomsection\label{\detokenize{pygace.examples.sto:pygace.examples.sto.sto_gace.Runner}}\pysiglinewithargsret{\sphinxbfcode{class }\sphinxcode{pygace.examples.sto.sto\_gace.}\sphinxbfcode{Runner}}{\emph{app=None}, \emph{iter\_idx=None}}{}
Bases: {\hyperref[\detokenize{pygace:pygace.gace.AbstractRunner}]{\sphinxcrossref{\sphinxcode{pygace.gace.AbstractRunner}}}}

A runner for running a GACE simulation.

This object is used to execute a GACE simulation in STO system.
\begin{quote}\begin{description}
\item[{Parameters}] \leavevmode\begin{description}
\item[{\sphinxstylestrong{app}}] \leavevmode{[}subclass of STOApp{]}
A subclass object of STOApp, default is \sphinxtitleref{None}.

\item[{\sphinxstylestrong{iter\_idx}}] \leavevmode{[}int{]}
Index of GA-to-CE iteration, default is \sphinxtitleref{None}.

\end{description}

\item[{Attributes}] \leavevmode\begin{description}
\item[{\sphinxstylestrong{app}}] \leavevmode{[}STOApp{]}
A subclass object of HFO2App.

\item[{\sphinxstylestrong{iter\_idx}}] \leavevmode{[}int{]}
Index of GA-to-CE iteration.

\end{description}

\end{description}\end{quote}
\index{create\_dir\_for\_DFT() (pygace.examples.sto.sto\_gace.Runner method)}

\begin{fulllineitems}
\phantomsection\label{\detokenize{pygace.examples.sto:pygace.examples.sto.sto_gace.Runner.create_dir_for_DFT}}\pysiglinewithargsret{\sphinxbfcode{create\_dir\_for\_DFT}}{\emph{task\_fname='./DFT\_task.dat'}}{}
Function to create directories for DFT calculation for ground-state
configurations candidates in each GA iteration. This function is used
to God\_view function. The identical functional method in a standard
GACE iteration is included \sphinxtitleref{print\_gs} member function in STOApp.
\begin{quote}\begin{description}
\item[{Parameters}] \leavevmode\begin{description}
\item[{\sphinxstylestrong{task\_fname}}] \leavevmode{[}str{]}
The name of file contain the information of directory in which
DFT task should be performed.

\end{description}

\item[{Returns}] \leavevmode\begin{description}
\item[{\sphinxstylestrong{None}}] \leavevmode
\end{description}

\end{description}\end{quote}

\end{fulllineitems}

\index{god\_view() (pygace.examples.sto.sto\_gace.Runner method)}

\begin{fulllineitems}
\phantomsection\label{\detokenize{pygace.examples.sto:pygace.examples.sto.sto_gace.Runner.god_view}}\pysiglinewithargsret{\sphinxbfcode{god\_view}}{}{}
In some cases, the number of all candidate configurations in sample
space is limited, and we can enumerate these configurations one by one
to calculate CE energis, which is a fast and efficient way to obtain
potential ground-state structures than standard genetic algorithms selection.
\begin{quote}\begin{description}
\item[{Returns}] \leavevmode\begin{description}
\item[{\sphinxstylestrong{None}}] \leavevmode
\end{description}

\end{description}\end{quote}

\end{fulllineitems}

\index{print\_gs() (pygace.examples.sto.sto\_gace.Runner method)}

\begin{fulllineitems}
\phantomsection\label{\detokenize{pygace.examples.sto:pygace.examples.sto.sto_gace.Runner.print_gs}}\pysiglinewithargsret{\sphinxbfcode{print\_gs}}{}{}
Function used to extract ground-state information from pickle file
saved during GACE running.
\begin{quote}\begin{description}
\item[{Returns}] \leavevmode\begin{description}
\item[{\sphinxstylestrong{None}}] \leavevmode
\end{description}

\end{description}\end{quote}

\end{fulllineitems}

\index{run() (pygace.examples.sto.sto\_gace.Runner method)}

\begin{fulllineitems}
\phantomsection\label{\detokenize{pygace.examples.sto:pygace.examples.sto.sto_gace.Runner.run}}\pysiglinewithargsret{\sphinxbfcode{run}}{}{}
Main function to run GA-to-CE iterations.
\begin{quote}\begin{description}
\item[{Returns}] \leavevmode\begin{description}
\item[{\sphinxstylestrong{None}}] \leavevmode
\end{description}

\end{description}\end{quote}

\end{fulllineitems}


\end{fulllineitems}

\index{STOApp (class in pygace.examples.sto.sto\_gace)}

\begin{fulllineitems}
\phantomsection\label{\detokenize{pygace.examples.sto:pygace.examples.sto.sto_gace.STOApp}}\pysiglinewithargsret{\sphinxbfcode{class }\sphinxcode{pygace.examples.sto.sto\_gace.}\sphinxbfcode{STOApp}}{\emph{ce\_site=1}, \emph{ce\_dirname='./data/iter0'}, \emph{ele\_1st='Ti\_sv'}, \emph{ele\_2nd='Nb\_sv'}, \emph{params\_config\_dict=None}}{}
Bases: {\hyperref[\detokenize{pygace:pygace.gace.AbstractApp}]{\sphinxcrossref{\sphinxcode{pygace.gace.AbstractApp}}}}

An app of SrTi(1-x)Nb(x)O3 system which is implemented from AbstractApp
object

This object is used to execute a GACE simulation, user only need to
implement several interfaces to custom their application.
\begin{quote}\begin{description}
\item[{Parameters}] \leavevmode\begin{description}
\item[{\sphinxstylestrong{ce\_site: int}}] \leavevmode
the concept of site used in ATAT program.

\item[{\sphinxstylestrong{ce\_dirname: str}}] \leavevmode
The name of a directory which contain information of MMAPS or MAPS
running

\item[{\sphinxstylestrong{ele\_1st: str}}] \leavevmode
The first type of element in the \sphinxcode{site} in \sphinxcode{ATAT}.

\item[{\sphinxstylestrong{ele\_2nd: str}}] \leavevmode
The second type of element in the \sphinxcode{site} in \sphinxcode{ATAT}.

\item[{\sphinxstylestrong{params\_config\_dict: dirt}}] \leavevmode
Parameter dict used to custom GACE AbstractApp.

\end{description}

\item[{Attributes}] \leavevmode\begin{description}
\item[{\sphinxstylestrong{app}}] \leavevmode{[}AbstractApp{]}
A subclass object of AbstractApp.

\item[{\sphinxstylestrong{iter\_idx}}] \leavevmode{[}int{]}
Index of GA-to-CE iteration.

\end{description}

\end{description}\end{quote}
\index{DEFAULT\_SETUP (pygace.examples.sto.sto\_gace.STOApp attribute)}

\begin{fulllineitems}
\phantomsection\label{\detokenize{pygace.examples.sto:pygace.examples.sto.sto_gace.STOApp.DEFAULT_SETUP}}\pysigline{\sphinxbfcode{DEFAULT\_SETUP}\sphinxbfcode{ = \{'MU\_OXYGEN': -4.91223, 'TMP\_DIR': '/home/yxcheng/PycharmProjects/pygace/doc/tmp\_dir', 'NB\_DEFECT': 12, 'TEMPLATE\_FILE': './data/lat\_in.template', 'DFT\_CAL\_DIR': './dft\_dirs', 'TEST\_RES\_DIR': '/home/yxcheng/PycharmProjects/pygace/doc/res\_dir', 'PICKLE\_DIR': '/home/yxcheng/PycharmProjects/pygace/doc/pickle\_bakup', 'PERFECT\_STO': 0.0, 'NB\_SITES': 15\}}}
\end{fulllineitems}

\index{create\_dir\_for\_DFT() (pygace.examples.sto.sto\_gace.STOApp method)}

\begin{fulllineitems}
\phantomsection\label{\detokenize{pygace.examples.sto:pygace.examples.sto.sto_gace.STOApp.create_dir_for_DFT}}\pysiglinewithargsret{\sphinxbfcode{create\_dir\_for\_DFT}}{\emph{task\_fname='./DFT\_task.dat'}}{}
Function to create directories for DFT calculation for ground-state
configurations candidates in each GA iteration. This function is used
to God\_view function. The identical functional method in a standard
GACE iteration is included \sphinxtitleref{print\_gs} member function in STOApp.
\begin{quote}\begin{description}
\item[{Parameters}] \leavevmode\begin{description}
\item[{\sphinxstylestrong{task\_fname}}] \leavevmode{[}str{]}
Name of file restoring the directory in which DFT task should be
performed.

\end{description}

\item[{Returns}] \leavevmode\begin{description}
\item[{\sphinxstylestrong{None}}] \leavevmode
\end{description}

\end{description}\end{quote}

\end{fulllineitems}

\index{evalEnergy() (pygace.examples.sto.sto\_gace.STOApp method)}

\begin{fulllineitems}
\phantomsection\label{\detokenize{pygace.examples.sto:pygace.examples.sto.sto_gace.STOApp.evalEnergy}}\pysiglinewithargsret{\sphinxbfcode{evalEnergy}}{\emph{individual}}{}
Evaluation function for the ground-state searching problem.

The problem is to determine a configuration of n vacancies
on a crystalline structures such that the energy of crystalline
structures can obtain minimum value.
\begin{quote}\begin{description}
\item[{Parameters}] \leavevmode\begin{description}
\item[{\sphinxstylestrong{individual}}] \leavevmode
\end{description}

\item[{Returns}] \leavevmode\begin{description}
\item[{\sphinxstylestrong{tuple}}] \leavevmode
A tuple contains energy in the first position, which is compatible
with \sphinxcode{DEAP}.

\end{description}

\end{description}\end{quote}

\end{fulllineitems}

\index{ind\_to\_elis() (pygace.examples.sto.sto\_gace.STOApp method)}

\begin{fulllineitems}
\phantomsection\label{\detokenize{pygace.examples.sto:pygace.examples.sto.sto_gace.STOApp.ind_to_elis}}\pysiglinewithargsret{\sphinxbfcode{ind\_to\_elis}}{\emph{individual}}{}
Convert individual (number list) to element list.
\begin{quote}\begin{description}
\item[{Parameters}] \leavevmode\begin{description}
\item[{\sphinxstylestrong{individual}}] \leavevmode
\end{description}

\item[{Returns}] \leavevmode\begin{description}
\item[{\sphinxstylestrong{list}}] \leavevmode
A list of element symbol string.

\end{description}

\end{description}\end{quote}

\end{fulllineitems}

\index{multiple\_run() (pygace.examples.sto.sto\_gace.STOApp method)}

\begin{fulllineitems}
\phantomsection\label{\detokenize{pygace.examples.sto:pygace.examples.sto.sto_gace.STOApp.multiple_run}}\pysiglinewithargsret{\sphinxbfcode{multiple\_run}}{\emph{ce\_iter}, \emph{ga\_iters}}{}
For multiple tasks
\begin{quote}\begin{description}
\item[{Parameters}] \leavevmode\begin{description}
\item[{\sphinxstylestrong{ce\_iter}}] \leavevmode{[}int{]}
How many times should a simulation repeat for statistic error(
different grouond-state in different running iterations with
identical parameter setting).

\item[{\sphinxstylestrong{ga\_iters}}] \leavevmode{[}int{]}
How many times should a GA simulation repeat in each GA-to-CE
iteration.

\end{description}

\item[{Returns}] \leavevmode\begin{description}
\item[{\sphinxstylestrong{list}}] \leavevmode
A list contain the results of running

\end{description}

\end{description}\end{quote}

\end{fulllineitems}

\index{run() (pygace.examples.sto.sto\_gace.STOApp method)}

\begin{fulllineitems}
\phantomsection\label{\detokenize{pygace.examples.sto:pygace.examples.sto.sto_gace.STOApp.run}}\pysiglinewithargsret{\sphinxbfcode{run}}{\emph{iter\_idx=0}, \emph{target\_epoch=50}}{}
Main function to run a GACE simulation which will be called by
\sphinxtitleref{AbstractRunner}.
\begin{quote}\begin{description}
\item[{Parameters}] \leavevmode\begin{description}
\item[{\sphinxstylestrong{iter\_idx}}] \leavevmode{[}int{]}
Determine which iteration the ECI is used in.

\item[{\sphinxstylestrong{target\_epoch}}] \leavevmode{[}int{]}
Iteration in GA simulation.

\end{description}

\item[{Returns}] \leavevmode\begin{description}
\item[{\sphinxstylestrong{None}}] \leavevmode
\end{description}

\end{description}\end{quote}

\end{fulllineitems}

\index{single\_run() (pygace.examples.sto.sto\_gace.STOApp method)}

\begin{fulllineitems}
\phantomsection\label{\detokenize{pygace.examples.sto:pygace.examples.sto.sto_gace.STOApp.single_run}}\pysiglinewithargsret{\sphinxbfcode{single\_run}}{\emph{ce\_iter}, \emph{ga\_iter}}{}
A single running task.
\begin{quote}\begin{description}
\item[{Parameters}] \leavevmode\begin{description}
\item[{\sphinxstylestrong{ce\_iter}}] \leavevmode{[}str{]}
How many times should a simulation repeat for statistic error(
different grouond-state in different running iterations with
identical parameter setting).

\item[{\sphinxstylestrong{ga\_iter}}] \leavevmode{[}int{]}
How many times should a GA simulation repeat in each GA-to-CE
iteration.

\end{description}

\item[{Returns}] \leavevmode\begin{description}
\item[{\sphinxstylestrong{tuple}}] \leavevmode
A tuple of pupulation, random states and hall of fame.

\end{description}

\end{description}\end{quote}

\end{fulllineitems}

\index{update\_ce() (pygace.examples.sto.sto\_gace.STOApp method)}

\begin{fulllineitems}
\phantomsection\label{\detokenize{pygace.examples.sto:pygace.examples.sto.sto_gace.STOApp.update_ce}}\pysiglinewithargsret{\sphinxbfcode{update\_ce}}{\emph{site=1}, \emph{dirname='./data/iter0'}}{}
Function to update inner CE object
\begin{quote}\begin{description}
\item[{Parameters}] \leavevmode\begin{description}
\item[{\sphinxstylestrong{site}}] \leavevmode{[}int, optional{]}
The site defined in \sphinxcode{lat.in} which is one of input files in ATAT.

\item[{\sphinxstylestrong{dirname}}] \leavevmode{[}str{]}
The name of directory contain running results of \sphinxcode{MMAPS}.

\end{description}

\item[{Returns}] \leavevmode\begin{description}
\item[{\sphinxstylestrong{None}}] \leavevmode
\end{description}

\end{description}\end{quote}

\end{fulllineitems}


\end{fulllineitems}



\subparagraph{Module contents}
\label{\detokenize{pygace.examples.sto:module-contents}}\label{\detokenize{pygace.examples.sto:module-pygace.examples.sto}}\index{pygace.examples.sto (module)}
A GA-to-CE example of Nb-doped SrTiO3 given in this module.


\paragraph{Module contents}
\label{\detokenize{pygace.examples:module-contents}}\label{\detokenize{pygace.examples:module-pygace.examples}}\index{pygace.examples (module)}
The module contains an example of Nb-doped SrTiO3 with 75 atoms supercell.

In this module, we will introduce two methods for obtaining ground-state
structures. One is used in general mode to obtain ground-state with GA-to-CE
iteration, the other one can be used in the case with samples of small
order of magnitude.


\subsubsection{pygace.scripts package}
\label{\detokenize{pygace.scripts:pygace-scripts-package}}\label{\detokenize{pygace.scripts::doc}}

\paragraph{Submodules}
\label{\detokenize{pygace.scripts:submodules}}

\paragraph{pygace.scripts.rungace module}
\label{\detokenize{pygace.scripts:pygace-scripts-rungace-module}}\label{\detokenize{pygace.scripts:module-pygace.scripts.rungace}}\index{pygace.scripts.rungace (module)}
Searching the most stable atomic-structure of a solid with point defects
(including the extrinsic alloying/doping elements), is one of the central issues in
materials science. Both adequate sampling of the configuration space and the
accurate energy evaluation at relatively low cost are demanding for the structure
prediction. In this work, we have developed a framework combining genetic
algorithm, cluster expansion (CE) method and first-principles calculations, which
can effectively locate the ground-state or meta-stable states of the relatively
large/complex systems. We employ this framework to search the stable structures
of two distinct systems, i.e., oxygen-vacancy-containing HfO(2-x) and the
Nb-doped SrTi(1-x)NbxO3 , and more stable structures are found compared with
the structures available in the literature. The present framework can be applied
to the ground-state search of extensive alloyed/doped materials, which is
particularly significant for the design of advanced engineering alloys and
semiconductors.
\index{build\_supercell\_template() (in module pygace.scripts.rungace)}

\begin{fulllineitems}
\phantomsection\label{\detokenize{pygace.scripts:pygace.scripts.rungace.build_supercell_template}}\pysiglinewithargsret{\sphinxcode{pygace.scripts.rungace.}\sphinxbfcode{build\_supercell\_template}}{\emph{scale}}{}
Create supercell for GA-to-CE simulation.
\begin{quote}\begin{description}
\item[{Parameters}] \leavevmode\begin{description}
\item[{\sphinxstylestrong{scale}}] \leavevmode{[}list or arrary like{]}
A list used to determine the size of supercell.

\end{description}

\item[{Returns}] \leavevmode\begin{description}
\item[{\sphinxstylestrong{None}}] \leavevmode
\end{description}

\end{description}\end{quote}

\end{fulllineitems}

\index{rungace() (in module pygace.scripts.rungace)}

\begin{fulllineitems}
\phantomsection\label{\detokenize{pygace.scripts:pygace.scripts.rungace.rungace}}\pysiglinewithargsret{\sphinxcode{pygace.scripts.rungace.}\sphinxbfcode{rungace}}{\emph{cell\_scale}, \emph{ele\_list}, \emph{ele\_nb}, \emph{*args}, \emph{**kwargs}}{}
Command for running GA-to-CE simulation.
\begin{quote}\begin{description}
\item[{Parameters}] \leavevmode\begin{description}
\item[{\sphinxstylestrong{cell\_scale}}] \leavevmode{[}list or arrary like{]}
A list used to specify the size of supercell.

\item[{\sphinxstylestrong{ele\_list}}] \leavevmode{[}list{]}
A list of elements contained in structure.

\item[{\sphinxstylestrong{ele\_nb}}] \leavevmode{[}list{]}
A list of maximum of the number of point defect in supercell structures.

\end{description}

\item[{Returns}] \leavevmode\begin{description}
\item[{\sphinxstylestrong{None}}] \leavevmode
\end{description}

\end{description}\end{quote}

\end{fulllineitems}

\index{show\_results() (in module pygace.scripts.rungace)}

\begin{fulllineitems}
\phantomsection\label{\detokenize{pygace.scripts:pygace.scripts.rungace.show_results}}\pysiglinewithargsret{\sphinxcode{pygace.scripts.rungace.}\sphinxbfcode{show\_results}}{\emph{ele\_type\_list}, \emph{defect\_con\_list}, \emph{use\_nb\_iter=False}, \emph{nb\_iter\_gace=None}, \emph{vasp\_cmd=None}, \emph{*args}, \emph{**kwargs}}{}
Show results of GA-to-CE simulation.
\begin{quote}\begin{description}
\item[{Parameters}] \leavevmode\begin{description}
\item[{\sphinxstylestrong{cell\_scale}}] \leavevmode{[}list or arrary like{]}
A list used to specify the size of supercell.

\item[{\sphinxstylestrong{ele\_list}}] \leavevmode{[}list{]}
A list of elements contained in structure.

\item[{\sphinxstylestrong{ele\_nb}}] \leavevmode{[}list{]}
A list of maximum of the number of point defect in supercell structures.

\item[{\sphinxstylestrong{nb\_iter\_gace}}] \leavevmode{[}bool{]}
Whether or not to determine stop criteria based on the number of iteration.

\item[{\sphinxstylestrong{vasp\_cmd}}] \leavevmode{[}str{]}
The command of VASP.

\end{description}

\item[{Returns}] \leavevmode\begin{description}
\item[{\sphinxstylestrong{None}}] \leavevmode
\end{description}

\end{description}\end{quote}

\end{fulllineitems}



\paragraph{Module contents}
\label{\detokenize{pygace.scripts:module-pygace.scripts}}\label{\detokenize{pygace.scripts:module-contents}}\index{pygace.scripts (module)}

\subsection{Submodules}
\label{\detokenize{pygace:submodules}}

\subsection{pygace.ce module}
\label{\detokenize{pygace:module-pygace.ce}}\label{\detokenize{pygace:pygace-ce-module}}\index{pygace.ce (module)}
The module wrapper cluster expansion method (CE) implemented in \sphinxcode{MMAPS} in
\sphinxcode{ATAT}.
\index{CE (class in pygace.ce)}

\begin{fulllineitems}
\phantomsection\label{\detokenize{pygace:pygace.ce.CE}}\pysiglinewithargsret{\sphinxbfcode{class }\sphinxcode{pygace.ce.}\sphinxbfcode{CE}}{\emph{lat\_in=None}, \emph{site=16}, \emph{corrdump\_cmd=None}, \emph{compare\_crystal\_cmd=None}}{}
Bases: \sphinxcode{object}

An wrapper for commends in \sphinxcode{ATAT}.

This class provides several commands that are commonly used in \sphinxcode{ATAT}.
\begin{quote}\begin{description}
\item[{Parameters}] \leavevmode\begin{description}
\item[{\sphinxstylestrong{lat\_in}}] \leavevmode{[}str{]}
File name of \sphinxcode{lat.in} in \sphinxcode{ATAT}.

\item[{\sphinxstylestrong{site}}] \leavevmode{[}int{]}
The site in which different can occupy.

\item[{\sphinxstylestrong{corrdump\_cmd}}] \leavevmode{[}str{]}
Command of \sphinxcode{corrdump} in \sphinxcode{ATAT}, default is ‘corrdump’

\item[{\sphinxstylestrong{compare\_crystal\_cmd}}] \leavevmode{[}str{]}
Command of program which is used to determine whether two crystal
structures are identical using symmetry analysis.

\end{description}

\item[{Attributes}] \leavevmode\begin{description}
\item[{\sphinxstylestrong{COMPARE\_CRYSTAL}}] \leavevmode{[}str{]}
This string restore a command used to determine whether two
configurations are identical in symmetry.

\item[{\sphinxstylestrong{CORRDUMP}}] \leavevmode{[}str{]}
This string restore the command of \sphinxtitleref{corrdump} in \sphinxcode{ATAT}.

\item[{\sphinxstylestrong{clster\_info}}] \leavevmode{[}str{]}
Filename of cluster information, default is \sphinxcode{clusters.out}
in \sphinxcode{ATAT}.

\item[{\sphinxstylestrong{eci\_out}}] \leavevmode{[}str{]}
File name of \sphinxcode{eci.out} in \sphinxcode{ATAT}.

\item[{\sphinxstylestrong{lat\_in}}] \leavevmode{[}str{]}
File name of \sphinxtitleref{lat.in} in \sphinxcode{ATAT}.

\item[{\sphinxstylestrong{per\_atom\_energy}}] \leavevmode{[}dict{]}
A dict restore element type and their responding energy defined in
\sphinxcode{atoms.out} file in \sphinxcode{ATAT}

\item[{\sphinxstylestrong{site}}] \leavevmode{[}int{]}
The site in which different can occupy.

\item[{\sphinxstylestrong{work\_path}}] \leavevmode{[}str{]}
The directory of \sphinxcode{MMAPS} or \sphinxcode{MAPS} running.

\end{description}

\end{description}\end{quote}
\index{COMPARE\_CRYSTAL (pygace.ce.CE attribute)}

\begin{fulllineitems}
\phantomsection\label{\detokenize{pygace:pygace.ce.CE.COMPARE_CRYSTAL}}\pysigline{\sphinxbfcode{COMPARE\_CRYSTAL}\sphinxbfcode{ = None}}
\end{fulllineitems}

\index{CORRDUMP (pygace.ce.CE attribute)}

\begin{fulllineitems}
\phantomsection\label{\detokenize{pygace:pygace.ce.CE.CORRDUMP}}\pysigline{\sphinxbfcode{CORRDUMP}\sphinxbfcode{ = '/home/yxcheng/usr/local/atat/bin/corrdump'}}
\end{fulllineitems}

\index{compare\_crystal() (pygace.ce.CE static method)}

\begin{fulllineitems}
\phantomsection\label{\detokenize{pygace:pygace.ce.CE.compare_crystal}}\pysiglinewithargsret{\sphinxbfcode{static }\sphinxbfcode{compare\_crystal}}{\emph{str1}, \emph{str2}, \emph{compare\_crystal\_cmd=None}, \emph{**kwargs}}{}
To determine whether structures are identical based crystal symmetry
analysis. The program used in this package is based on \sphinxcode{XXX} library
which developed by XXX.
\begin{quote}\begin{description}
\item[{Parameters}] \leavevmode\begin{description}
\item[{\sphinxstylestrong{str1}}] \leavevmode{[}str{]}
The first string used to represent elements .

\item[{\sphinxstylestrong{str2}}] \leavevmode{[}str{]}
The second string used to represent elements.

\item[{\sphinxstylestrong{compare\_crystal\_cmd}}] \leavevmode{[}str{]}
The program developed to determine whether two
crystal structures are identical, default \sphinxtitleref{None}.

\item[{\sphinxstylestrong{kwargs}}] \leavevmode{[}dict arguments{]}
Other arguments used in \sphinxtitleref{compare\_crystal\_cmd}.

\end{description}

\item[{Returns}] \leavevmode\begin{description}
\item[{\sphinxstylestrong{bool}}] \leavevmode
True for yes and False for no.

\end{description}

\end{description}\end{quote}

\end{fulllineitems}

\index{corrdump() (pygace.ce.CE method)}

\begin{fulllineitems}
\phantomsection\label{\detokenize{pygace:pygace.ce.CE.corrdump}}\pysiglinewithargsret{\sphinxbfcode{corrdump}}{\emph{cmd}}{}
Obtain energy predicted by \sphinxcode{corrdump} command in \sphinxcode{ATAT}.
\begin{quote}\begin{description}
\item[{Parameters}] \leavevmode\begin{description}
\item[{\sphinxstylestrong{cmd}}] \leavevmode{[}str{]}
Shell command which call system \sphinxcode{corrdump} command of \sphinxcode{ATAT}.

\end{description}

\item[{Returns}] \leavevmode\begin{description}
\item[{\sphinxstylestrong{float}}] \leavevmode
Energy predicted by \sphinxcode{corrdump} command.

\end{description}

\end{description}\end{quote}

\end{fulllineitems}

\index{fit() (pygace.ce.CE method)}

\begin{fulllineitems}
\phantomsection\label{\detokenize{pygace:pygace.ce.CE.fit}}\pysiglinewithargsret{\sphinxbfcode{fit}}{\emph{dirname='./.tmp\_atat\_ce\_dir'}}{}
Obtain all information of a directory in which a correct calculation of
CE fitting has been performed.
\begin{quote}\begin{description}
\item[{Parameters}] \leavevmode\begin{description}
\item[{\sphinxstylestrong{dirname}}] \leavevmode{[}str{]}
Which directory of the CE running.

\end{description}

\item[{Returns}] \leavevmode\begin{description}
\item[{\sphinxstylestrong{None}}] \leavevmode
\end{description}

\end{description}\end{quote}

\end{fulllineitems}

\index{get\_total\_energy() (pygace.ce.CE method)}

\begin{fulllineitems}
\phantomsection\label{\detokenize{pygace:pygace.ce.CE.get_total_energy}}\pysiglinewithargsret{\sphinxbfcode{get\_total\_energy}}{\emph{x}, \emph{is\_corrdump=False}, \emph{is\_ref=False}, \emph{site\_repeat=-1}, \emph{sum\_corr=0.0}, \emph{delete\_file=True}}{}
Calculate absolute energy of a crystal structure like first-principles
calculation software package computed.
\begin{quote}\begin{description}
\item[{Parameters}] \leavevmode\begin{description}
\item[{\sphinxstylestrong{x}}] \leavevmode{[}str{]}
String for filename of lattice crystal, default \sphinxcode{str.out}.

\item[{\sphinxstylestrong{is\_corrdump}}] \leavevmode{[}bool{]}
Determine whether function use energy computed by \sphinxcode{corrdump}
command to replace absolute energy, default \sphinxtitleref{False}.

\item[{\sphinxstylestrong{is\_ref}}] \leavevmode{[}bool{]}
Determine whether function use relative energy provided by users.

\item[{\sphinxstylestrong{site\_repeat}}] \leavevmode{[}int{]}
This variable should be used seriously when a lattice structure
cannot map parent lattice.

\item[{\sphinxstylestrong{sum\_corr}}] \leavevmode{[}float{]}
If \sphinxtitleref{is\_ref} is \sphinxtitleref{True} this value will be input as energy predicted
by \sphinxcode{corrdump} command.

\item[{\sphinxstylestrong{delete\_file :}}] \leavevmode
Whether to delete tmp file generated by program.

\end{description}

\item[{Returns}] \leavevmode\begin{description}
\item[{\sphinxstylestrong{float :}}] \leavevmode
Total energy or corrdump energy.

\end{description}

\end{description}\end{quote}

\end{fulllineitems}

\index{make\_template() (pygace.ce.CE method)}

\begin{fulllineitems}
\phantomsection\label{\detokenize{pygace:pygace.ce.CE.make_template}}\pysiglinewithargsret{\sphinxbfcode{make\_template}}{\emph{scale}}{}~
\end{fulllineitems}

\index{mmaps() (pygace.ce.CE static method)}

\begin{fulllineitems}
\phantomsection\label{\detokenize{pygace:pygace.ce.CE.mmaps}}\pysiglinewithargsret{\sphinxbfcode{static }\sphinxbfcode{mmaps}}{\emph{dirname}, \emph{maps\_args='-d'}}{}
Call \sphinxcode{MMAPS} command in system.
\begin{quote}\begin{description}
\item[{Parameters}] \leavevmode\begin{description}
\item[{\sphinxstylestrong{dirname}}] \leavevmode{[}str{]}
Directory name of \sphinxcode{MMAPS} command running. Usually, it contains a
\sphinxcode{lat.in} file, \sphinxcode{vasp.wrap} or other wrap file for different
first-principles calculation.

\item[{\sphinxstylestrong{cal}}] \leavevmode{[}bool{]}
Determine whether to run a CE fitting. If \sphinxtitleref{False}, the function
will return when clusters information is obtained, and vice versa
CE fitting is running until users stop it.

\item[{\sphinxstylestrong{args}}] \leavevmode{[}position arg{]}
Position arguments for \sphinxcode{MMAPS} command.

\item[{\sphinxstylestrong{kwargs}}] \leavevmode{[}dict arg{]}
Dict arguments for \sphinxcode{MMAPS} command.

\end{description}

\item[{Returns}] \leavevmode\begin{description}
\item[{\sphinxstylestrong{None}}] \leavevmode
\end{description}

\end{description}\end{quote}

\end{fulllineitems}

\index{predict() (pygace.ce.CE method)}

\begin{fulllineitems}
\phantomsection\label{\detokenize{pygace:pygace.ce.CE.predict}}\pysiglinewithargsret{\sphinxbfcode{predict}}{\emph{x}}{}
Predict energy by given file name of structure.
\begin{quote}\begin{description}
\item[{Parameters}] \leavevmode\begin{description}
\item[{\sphinxstylestrong{x}}] \leavevmode{[}str{]}
‘x’ is a name of lattice structure, such as \sphinxcode{str.out} in \sphinxcode{ATAT}.

\end{description}

\item[{Returns}] \leavevmode\begin{description}
\item[{\sphinxstylestrong{str}}] \leavevmode
Energy predicted by corrdump command in \sphinxcode{ATAT}

\end{description}

\end{description}\end{quote}

\end{fulllineitems}


\end{fulllineitems}



\subsection{pygace.config module}
\label{\detokenize{pygace:pygace-config-module}}\label{\detokenize{pygace:module-pygace.config}}\index{pygace.config (module)}
The module contains config file for pygace running.


\subsection{pygace.ga module}
\label{\detokenize{pygace:module-pygace.ga}}\label{\detokenize{pygace:pygace-ga-module}}\index{pygace.ga (module)}
The module contains genetic algorithms and relevant operator used in GA, e.g.,
crossover operator, mutation operator.
\index{Cromosome (class in pygace.ga)}

\begin{fulllineitems}
\phantomsection\label{\detokenize{pygace:pygace.ga.Cromosome}}\pysiglinewithargsret{\sphinxbfcode{class }\sphinxcode{pygace.ga.}\sphinxbfcode{Cromosome}}{\emph{gene\_length=64}, \emph{fitness=0.0}}{}
Bases: \sphinxcode{object}

A wrapper class for individual in GA.
\begin{quote}\begin{description}
\item[{Parameters}] \leavevmode\begin{description}
\item[{\sphinxstylestrong{gene\_length}}] \leavevmode{[}int{]}
The length of gene.

\item[{\sphinxstylestrong{fitness}}] \leavevmode{[}float{]}
The fitness value of gene

\end{description}

\end{description}\end{quote}
\index{generate\_cromosome() (pygace.ga.Cromosome method)}

\begin{fulllineitems}
\phantomsection\label{\detokenize{pygace:pygace.ga.Cromosome.generate_cromosome}}\pysiglinewithargsret{\sphinxbfcode{generate\_cromosome}}{}{}
Greate a random cromosome
\begin{quote}\begin{description}
\item[{Returns}] \leavevmode\begin{description}
\item[{\sphinxstylestrong{None}}] \leavevmode
\end{description}

\end{description}\end{quote}

\end{fulllineitems}

\index{get\_gene() (pygace.ga.Cromosome method)}

\begin{fulllineitems}
\phantomsection\label{\detokenize{pygace:pygace.ga.Cromosome.get_gene}}\pysiglinewithargsret{\sphinxbfcode{get\_gene}}{\emph{index}}{}
Obtain gene in the position of \sphinxtitleref{index}.
\begin{quote}\begin{description}
\item[{Parameters}] \leavevmode\begin{description}
\item[{\sphinxstylestrong{index}}] \leavevmode{[}int{]}
The index of position.

\end{description}

\item[{Returns}] \leavevmode\begin{description}
\item[{\sphinxstylestrong{int}}] \leavevmode
The gene

\end{description}

\end{description}\end{quote}

\end{fulllineitems}

\index{set\_gene() (pygace.ga.Cromosome method)}

\begin{fulllineitems}
\phantomsection\label{\detokenize{pygace:pygace.ga.Cromosome.set_gene}}\pysiglinewithargsret{\sphinxbfcode{set\_gene}}{\emph{index}, \emph{value}}{}
Set gene in position of \sphinxtitleref{index}.
\begin{quote}\begin{description}
\item[{Parameters}] \leavevmode\begin{description}
\item[{\sphinxstylestrong{index}}] \leavevmode{[}int{]}
The index of position.

\item[{\sphinxstylestrong{value}}] \leavevmode{[}float{]}
The value of gene.

\end{description}

\item[{Returns}] \leavevmode\begin{description}
\item[{\sphinxstylestrong{None}}] \leavevmode
\end{description}

\end{description}\end{quote}

\end{fulllineitems}

\index{size() (pygace.ga.Cromosome method)}

\begin{fulllineitems}
\phantomsection\label{\detokenize{pygace:pygace.ga.Cromosome.size}}\pysiglinewithargsret{\sphinxbfcode{size}}{}{}
Return the length of cromosome.
\begin{quote}\begin{description}
\item[{Returns}] \leavevmode\begin{description}
\item[{\sphinxstylestrong{int}}] \leavevmode
The length of cromosome

\end{description}

\end{description}\end{quote}

\end{fulllineitems}

\index{valid\_type (pygace.ga.Cromosome attribute)}

\begin{fulllineitems}
\phantomsection\label{\detokenize{pygace:pygace.ga.Cromosome.valid_type}}\pysigline{\sphinxbfcode{valid\_type}\sphinxbfcode{ = {[}'Vac', 'Replace'{]}}}
\end{fulllineitems}


\end{fulllineitems}

\index{Individual (class in pygace.ga)}

\begin{fulllineitems}
\phantomsection\label{\detokenize{pygace:pygace.ga.Individual}}\pysigline{\sphinxbfcode{class }\sphinxcode{pygace.ga.}\sphinxbfcode{Individual}}
Bases: \sphinxcode{object}

Individual object contain several Cromosome objects.

\end{fulllineitems}

\index{cycle\_crossover() (in module pygace.ga)}

\begin{fulllineitems}
\phantomsection\label{\detokenize{pygace:pygace.ga.cycle_crossover}}\pysiglinewithargsret{\sphinxcode{pygace.ga.}\sphinxbfcode{cycle\_crossover}}{\emph{ind1}, \emph{ind2}, \emph{cross\_number}}{}
Cycle crossover (CX).

CX algorithm:
\begin{itemize}
\item {} 
parent1: \sphinxcode{{[}\textbar{}1 \textbar{}2 3 \textbar{}4 \textbar{}5 6 7 8 \textbar{}9{]}}

\item {} 
parent2: \sphinxcode{{[}\textbar{}5 \textbar{}4 6 \textbar{}9 \textbar{}2 3 7 8 \textbar{}1{]}}

\item {} 
child1: \sphinxcode{{[}\textbar{}1 \textbar{}2 6 \textbar{}4 \textbar{}5 3 7 8 \textbar{}9{]}}

\item {} 
child2: \sphinxcode{{[}\textbar{}5 \textbar{}4 3 \textbar{}9 \textbar{}2 6 7 8 \textbar{}1{]}}

\end{itemize}
\begin{quote}\begin{description}
\item[{Parameters}] \leavevmode\begin{description}
\item[{\sphinxstylestrong{ind1}}] \leavevmode{[}iteration object{]}
The first cromosome participating in the crossover.

\item[{\sphinxstylestrong{ind2}}] \leavevmode{[}iteration object{]}
The second cromosome participating in the crossover.

\item[{\sphinxstylestrong{cross\_number}}] \leavevmode{[}int{]}
The number of crossover is not used in this algorithm.

\end{description}

\item[{Returns}] \leavevmode\begin{description}
\item[{\sphinxstylestrong{tuple}}] \leavevmode
A tuple of two cromosomes

\end{description}

\end{description}\end{quote}
\paragraph{References}

More details can bee seen Ref. %
\begin{footnote}[1]\sphinxAtStartFootnote
Oliver, I.; Smith, D.; Holland, J. A study of permutation crossover
operators on the traveling salesman problem. Proceedings of the 2nd
International Conference on Genetic Algorithms, J.J. Grefenstette (ed.).
Hillsdale, New Jersey, 1987; pp 224-230.
%
\end{footnote}.

\end{fulllineitems}

\index{gaceCrossover() (in module pygace.ga)}

\begin{fulllineitems}
\phantomsection\label{\detokenize{pygace:pygace.ga.gaceCrossover}}\pysiglinewithargsret{\sphinxcode{pygace.ga.}\sphinxbfcode{gaceCrossover}}{\emph{indiv1}, \emph{indiv2}, \emph{crossover\_type=1}, \emph{cross\_num=8}}{}
Executes a crossover specified by crossover type \sphinxtitleref{crossover\_type{}`and the
number of crossover {}`cross\_num} on the input \sphinxtitleref{sequence} individuals.
The two individuals are modified in place and both keep their original
length.
\begin{quote}\begin{description}
\item[{Parameters}] \leavevmode\begin{description}
\item[{\sphinxstylestrong{indiv1}}] \leavevmode{[}Cromosome object{]}
The first individual participating in the crossover.

\item[{\sphinxstylestrong{indiv2}}] \leavevmode{[}Cromosome object{]}
The second individual participating in the crossover.

\item[{\sphinxstylestrong{crossover\_type}}] \leavevmode{[}int{]}
The type of crossover method:
\begin{itemize}
\item {} 
\sphinxcode{1}: Partially-mapped crossover (PMX)

\item {} 
\sphinxcode{2}: Order Crossover (OX1)

\item {} 
\sphinxcode{3}: Position based crossover (POS)

\item {} 
\sphinxcode{4}: Order based Crossover (OX2)

\item {} 
\sphinxcode{5}: Cycle crossover (CX)

\item {} 
\sphinxcode{6}: Subtour exchange crossover (SXX)

\end{itemize}

\item[{\sphinxstylestrong{cross\_num}}] \leavevmode{[}int{]}
The number of crossover which determine the number exchange in each
crossover operation.

\end{description}

\item[{Returns}] \leavevmode\begin{description}
\item[{\sphinxstylestrong{tuple}}] \leavevmode
A tuple of two individuals.

\end{description}

\end{description}\end{quote}

\end{fulllineitems}

\index{gaceGA() (in module pygace.ga)}

\begin{fulllineitems}
\phantomsection\label{\detokenize{pygace:pygace.ga.gaceGA}}\pysiglinewithargsret{\sphinxcode{pygace.ga.}\sphinxbfcode{gaceGA}}{\emph{population}, \emph{toolbox}, \emph{cxpb}, \emph{ngen}, \emph{stats=None}, \emph{halloffame=None}, \emph{verbose=True}, \emph{checkpoint=None}, \emph{freq=10}}{}
Genetic algorithm (GA) used in \sphinxcode{pygace}. Users can define their algorithms
based on \sphinxcode{DEAP} package or other GA framework.
\begin{quote}\begin{description}
\item[{Parameters}] \leavevmode\begin{description}
\item[{\sphinxstylestrong{population}}] \leavevmode{[}list{]}
A list represent population consists of all individual.

\item[{\sphinxstylestrong{toolbox}}] \leavevmode{[}toolbox object{]}
\sphinxcode{DEAP} toolbox object

\item[{\sphinxstylestrong{cxpb}}] \leavevmode{[}float{]}
The probability of crossover happens.

\item[{\sphinxstylestrong{ngen}}] \leavevmode{[}int{]}
The number of generations.

\item[{\sphinxstylestrong{stats :}}] \leavevmode
The random state of simulation.

\item[{\sphinxstylestrong{halloffame}}] \leavevmode{[}list{]}
Restored individual.

\item[{\sphinxstylestrong{verbose}}] \leavevmode{[}bool{]}
Whether to show more message of running.

\item[{\sphinxstylestrong{checkpoint}}] \leavevmode{[}str{]}
The filename of checkpoint file.

\item[{\sphinxstylestrong{freq}}] \leavevmode{[}int{]}
The number that determine how many step to write a checkpoint.

\end{description}

\item[{Returns}] \leavevmode\begin{description}
\item[{\sphinxstylestrong{tuple}}] \leavevmode
A tuple of population and log file.

\end{description}

\end{description}\end{quote}

\end{fulllineitems}

\index{gaceMutShuffleIndexes() (in module pygace.ga)}

\begin{fulllineitems}
\phantomsection\label{\detokenize{pygace:pygace.ga.gaceMutShuffleIndexes}}\pysiglinewithargsret{\sphinxcode{pygace.ga.}\sphinxbfcode{gaceMutShuffleIndexes}}{\emph{individual}, \emph{indpb}}{}
Shuffle the attributes of the input individual and return the mutant.
The \sphinxtitleref{individual} is expected to be a \sphinxtitleref{sequence}. The \sphinxtitleref{indpb} argument
is the probability of each attribute to be moved. Usually this mutation is
applied on vector of indices.
\begin{quote}\begin{description}
\item[{Parameters}] \leavevmode\begin{description}
\item[{\sphinxstylestrong{individual}}] \leavevmode{[}Individual object{]}
Individual to be mutated.

\item[{\sphinxstylestrong{indpb}}] \leavevmode{[}float{]}
Independent probability for each attribute to be exchanged to another
position.

\end{description}

\item[{Returns}] \leavevmode\begin{description}
\item[{\sphinxstylestrong{tuple}}] \leavevmode
A tuple of one individual.

\end{description}

\end{description}\end{quote}

\end{fulllineitems}

\index{gaceVarAnd() (in module pygace.ga)}

\begin{fulllineitems}
\phantomsection\label{\detokenize{pygace:pygace.ga.gaceVarAnd}}\pysiglinewithargsret{\sphinxcode{pygace.ga.}\sphinxbfcode{gaceVarAnd}}{\emph{population}, \emph{toolbox}, \emph{cxpb}}{}
Execute crossover and mutation operation in genetic algorithm running
process.
\begin{quote}\begin{description}
\item[{Parameters}] \leavevmode\begin{description}
\item[{\sphinxstylestrong{population}}] \leavevmode{[}list{]}
The population of all individual.

\item[{\sphinxstylestrong{toolbox}}] \leavevmode{[}Toolbox object{]}
The \sphinxtitleref{Toolbox} object defined in \sphinxcode{DEAP}.

\item[{\sphinxstylestrong{cxpb}}] \leavevmode{[}float{]}
The probability or crossover.

\end{description}

\item[{Returns}] \leavevmode\begin{description}
\item[{\sphinxstylestrong{list}}] \leavevmode
The new generation.

\end{description}

\end{description}\end{quote}

\end{fulllineitems}

\index{order\_based\_crossover() (in module pygace.ga)}

\begin{fulllineitems}
\phantomsection\label{\detokenize{pygace:pygace.ga.order_based_crossover}}\pysiglinewithargsret{\sphinxcode{pygace.ga.}\sphinxbfcode{order\_based\_crossover}}{\emph{ind1}, \emph{ind2}, \emph{cross\_number}}{}
Order based Crossover (OX2) operator selects at random several positions
in a parent tour, and the order of the cities in the selected positions
of this parent is imposed on the other parent.

OX2 algorithm example:
\begin{itemize}
\item {} 
parent1 \sphinxcode{{[}1 \textbar{}2 3 4 \textbar{}5 \textbar{}6 7 8 \textbar{}9{]}}

\item {} 
parent2 \sphinxcode{{[}5 \textbar{}4 6 3 \textbar{}1 \textbar{}9 2 7 \textbar{}8{]}}

\item {} 
child1 \sphinxcode{{[}2 \textbar{}4 5 \textbar{}3 \textbar{}1 6 9 \textbar{}7 \textbar{}8{]}}

\item {} 
child2 \sphinxcode{{[}4 \textbar{}2 \textbar{}3 1 \textbar{}5 \textbar{}6 \textbar{}7 9 8{]}}

\end{itemize}
\begin{quote}\begin{description}
\item[{Parameters}] \leavevmode\begin{description}
\item[{\sphinxstylestrong{ind1}}] \leavevmode{[}iteration object{]}
The first cromosome participating in the crossover.

\item[{\sphinxstylestrong{ind2}}] \leavevmode{[}iteration object{]}
The second cromosome participating in the crossover.

\item[{\sphinxstylestrong{cross\_number}}] \leavevmode{[}int{]}
The number of crossover which determine the number exchange in each
crossover operation.

\end{description}

\item[{Returns}] \leavevmode\begin{description}
\item[{\sphinxstylestrong{tuple}}] \leavevmode
A tuple of two cromosomes

\end{description}

\end{description}\end{quote}
\paragraph{References}

More details about OX2 can be seen Ref. %
\begin{footnote}[2]\sphinxAtStartFootnote
Syswerda, G. Handbook of Genetic Algorithms 1991, 332-349.
%
\end{footnote}.

\end{fulllineitems}

\index{order\_crossover() (in module pygace.ga)}

\begin{fulllineitems}
\phantomsection\label{\detokenize{pygace:pygace.ga.order_crossover}}\pysiglinewithargsret{\sphinxcode{pygace.ga.}\sphinxbfcode{order\_crossover}}{\emph{ind1}, \emph{ind2}, \emph{cross\_number}}{}
Order crossover (OX1) operator was proposed by Davis (1985). The OX1 exploits
a property of the path representation, that the order of cities (not their
positions) are important. It constructs an offspring by choosing a subtour
of one parent and preserving the relative order of cities of the other
parent.

order crossover algorithm example:
\begin{itemize}
\item {} 
parent1: \sphinxcode{{[}1 2 \textbar{}3 4 5 6\textbar{} 7 8 9{]}}

\item {} 
parent2: \sphinxcode{{[}5 7 \textbar{}4 9 1 3\textbar{} 6 2 8{]}}

\item {} 
child1: \sphinxcode{{[}7 9 \textbar{}3 4 5 6\textbar{} 1 2 8{]}}

\item {} 
child2: \sphinxcode{{[}2 5 \textbar{}4 9 1 3\textbar{} 6 7 8{]}}

\end{itemize}
\begin{quote}\begin{description}
\item[{Parameters}] \leavevmode\begin{description}
\item[{\sphinxstylestrong{ind1}}] \leavevmode{[}iteration object{]}
The first cromosome participating in the crossover.

\item[{\sphinxstylestrong{ind2}}] \leavevmode{[}iteration object{]}
The second cromosome participating in the crossover.

\item[{\sphinxstylestrong{cross\_number}}] \leavevmode{[}int{]}
The number of crossover which determine the number exchange in each
crossover operation.

\end{description}

\item[{Returns}] \leavevmode\begin{description}
\item[{\sphinxstylestrong{tuple}}] \leavevmode
A tuple of two cromosomes

\end{description}

\end{description}\end{quote}
\paragraph{References}

More details about OX1 can be seen Ref. %
\begin{footnote}[3]\sphinxAtStartFootnote
Davis, L. Applying Adaptive Algorithms to Epistatic Domains. Proceedings
of the 9th International Joint Conference on Artificial Intelligence -
Volume 1. San Francisco, CA, USA, 1985; pp 162-164.
%
\end{footnote}.

\end{fulllineitems}

\index{partial\_mapped\_crossover() (in module pygace.ga)}

\begin{fulllineitems}
\phantomsection\label{\detokenize{pygace:pygace.ga.partial_mapped_crossover}}\pysiglinewithargsret{\sphinxcode{pygace.ga.}\sphinxbfcode{partial\_mapped\_crossover}}{\emph{ind1}, \emph{ind2}, \emph{cross\_number}}{}
Partially-mapped crossover (PMX) operator was suggested by Goldberg and
Lingle (1985). It passes on ordering and value information from the
parent tours to the offspring tours. A portion of one parents’s string
is mapped onto a portion of the other parent’s string and the remaining
informatin is exchanged..

The algorithm example:
\begin{itemize}
\item {} 
parent1: \sphinxcode{{[}1,2,\textbar{}3,4,5,6\textbar{},7,8,9{]}}

\item {} 
parent2: \sphinxcode{{[}5,4,\textbar{}6,9,2,1\textbar{},7,8,3{]}}

\item {} 
child1: \sphinxcode{{[}3,5,\textbar{}6,9,2,1\textbar{},7,8,4{]}}

\item {} 
child2: \sphinxcode{{[}2,9,\textbar{}3,4,5,6\textbar{},7,8,1{]}}

\end{itemize}
\begin{quote}\begin{description}
\item[{Parameters}] \leavevmode\begin{description}
\item[{\sphinxstylestrong{ind1}}] \leavevmode{[}iteration object{]}
The first individual participating in the crossover.

\item[{\sphinxstylestrong{ind2}}] \leavevmode{[}iteration object{]}
The second individual participating in the crossover.

\item[{\sphinxstylestrong{cross\_number}}] \leavevmode{[}int{]}
The number of crossover which determine the number exchange in each
crossover operation.

\end{description}

\item[{Returns}] \leavevmode\begin{description}
\item[{\sphinxstylestrong{tuple}}] \leavevmode
A tuple of two individuals

\end{description}

\end{description}\end{quote}
\paragraph{References}

More details about PMX can be seen in Ref. %
\begin{footnote}[4]\sphinxAtStartFootnote
Goldberg, D.; Lingle, R.; Alleles, L. the Travelling Salesman Problem.
Proceedings of the 1st International Conference on Genetic Algorithms and
their Applications, J.J. Grefenstette (ed.). Carneige-Mellon University,
Pittsburgh, 1985.
%
\end{footnote}.

\end{fulllineitems}

\index{position\_based\_crossover() (in module pygace.ga)}

\begin{fulllineitems}
\phantomsection\label{\detokenize{pygace:pygace.ga.position_based_crossover}}\pysiglinewithargsret{\sphinxcode{pygace.ga.}\sphinxbfcode{position\_based\_crossover}}{\emph{ind1}, \emph{ind2}, \emph{cross\_number}}{}
Position-based crossover (PBC)

PBC algorithm:
\begin{itemize}
\item {} 
parent1: \sphinxcode{{[}1 \textbar{}2 3 4 \textbar{}5 \textbar{}6 7 8 \textbar{}9{]}}

\item {} 
parent2: \sphinxcode{{[}5 \textbar{}4 6 4 \textbar{}1 \textbar{}9 2 7 \textbar{}8{]}}

\item {} 
child1: \sphinxcode{{[}4 \textbar{}2 3 1 \textbar{}5 \textbar{}6 7 8 \textbar{}9{]}}

\item {} 
child2: \sphinxcode{{[}2 \textbar{}4 3 5 \textbar{}1 \textbar{}9 6 7 \textbar{}8{]}}

\end{itemize}
\begin{quote}\begin{description}
\item[{Parameters}] \leavevmode\begin{description}
\item[{\sphinxstylestrong{ind1}}] \leavevmode{[}iteration object{]}
The first cromosome participating in the crossover.

\item[{\sphinxstylestrong{ind2}}] \leavevmode{[}iteration object{]}
The second cromosome participating in the crossover.

\item[{\sphinxstylestrong{cross\_number}}] \leavevmode{[}int{]}
The number of crossover which determine the number exchange in each
crossover operation.

\end{description}

\item[{Returns}] \leavevmode\begin{description}
\item[{\sphinxstylestrong{tuple}}] \leavevmode
A tuple of two cromosomes

\end{description}

\end{description}\end{quote}
\paragraph{References}

More details can be seen Ref. %
\begin{footnote}[5]\sphinxAtStartFootnote
Syswerda, G. Handbook of Genetic Algorithms 1991, 332-349.
%
\end{footnote}.

\end{fulllineitems}

\index{subtour\_exchange\_crossover() (in module pygace.ga)}

\begin{fulllineitems}
\phantomsection\label{\detokenize{pygace:pygace.ga.subtour_exchange_crossover}}\pysiglinewithargsret{\sphinxcode{pygace.ga.}\sphinxbfcode{subtour\_exchange\_crossover}}{\emph{ind1}, \emph{ind2}, \emph{cross\_number}}{}
Subtour exchange crossover (SXX).

SXX algorithm:
\begin{itemize}
\item {} 
parent1: \sphinxcode{{[}1 2 3 \textbar{}4 5 6 7\textbar{} 8 9{]}}

\item {} 
parent2: \sphinxcode{{[}3 \textbar{}4 9 \textbar{}7 8 \textbar{}5 2 1 \textbar{}6{]}}

\item {} 
child1: \sphinxcode{{[}1 2 3 \textbar{}4 7 5 6\textbar{} 8 9{]}}

\item {} 
child2: \sphinxcode{{[}3 \textbar{}4 9 \textbar{}5 8 \textbar{}6 2 1 \textbar{}7{]}}

\end{itemize}
\begin{quote}\begin{description}
\item[{Parameters}] \leavevmode\begin{description}
\item[{\sphinxstylestrong{ind1}}] \leavevmode{[}iteration object{]}
The first cromosome participating in the crossover.

\item[{\sphinxstylestrong{ind2}}] \leavevmode{[}iteration object{]}
The second cromosome participating in the crossover.

\item[{\sphinxstylestrong{cross\_number}}] \leavevmode{[}int{]}
The number of crossover is not used in this algorithm.

\end{description}

\item[{Returns}] \leavevmode\begin{description}
\item[{\sphinxstylestrong{tuple}}] \leavevmode
A tuple of two cromosomes

\end{description}

\end{description}\end{quote}
\paragraph{References}

More details about SXX can be seen Ref. %
\begin{footnote}[6]\sphinxAtStartFootnote
Yamamura, M.; Ono, T.; Kobayashi, S. Japanese Society for Artificial
Intelligence.
%
\end{footnote}.

\end{fulllineitems}

\index{transfer\_from() (in module pygace.ga)}

\begin{fulllineitems}
\phantomsection\label{\detokenize{pygace:pygace.ga.transfer_from}}\pysiglinewithargsret{\sphinxcode{pygace.ga.}\sphinxbfcode{transfer\_from}}{\emph{ind}}{}~
\end{fulllineitems}



\subsection{pygace.gace module}
\label{\detokenize{pygace:module-pygace.gace}}\label{\detokenize{pygace:pygace-gace-module}}\index{pygace.gace (module)}
GACE framework module

This module provide abstract GACE object used to be implemented by users in
their application, and it defines several interface which are called in
concreate application.
\index{AbstractApp (class in pygace.gace)}

\begin{fulllineitems}
\phantomsection\label{\detokenize{pygace:pygace.gace.AbstractApp}}\pysiglinewithargsret{\sphinxbfcode{class }\sphinxcode{pygace.gace.}\sphinxbfcode{AbstractApp}}{\emph{ce\_site=None}, \emph{ce\_dirname='./data/iter1'}, \emph{params\_config\_dict=None}}{}
Bases: \sphinxcode{object}

Abstract application object for \sphinxcode{GACE} framework.

AbstractApp initial process needs input parameters of CE simulation
and informatin of output directory. Also, the parameters for DFT
calculation should also be included in \sphinxcode{params\_config\_dict} for
user custom.
\begin{quote}\begin{description}
\item[{Parameters}] \leavevmode\begin{description}
\item[{\sphinxstylestrong{ce\_site}}] \leavevmode{[}int{]}
The concept of site used in \sphinxcode{MAPS} or \sphinxcode{MMAPS} in \sphinxcode{ATAT}
program.

\item[{\sphinxstylestrong{ce\_dirname}}] \leavevmode{[}:obj: str, optional{]}\begin{description}
\item[{A path of directory which contains information after running}] \leavevmode
\sphinxcode{MMAPS} or \sphinxcode{MAPS}.

\end{description}

\item[{\sphinxstylestrong{params\_config\_dict}}] \leavevmode{[}dict, optional{]}
A dict used to update DEFAULT\_DICT of AbstractApp object.

\end{description}

\item[{Attributes}] \leavevmode\begin{description}
\item[{\sphinxstylestrong{ce}}] \leavevmode{[}CE{]}
CE object defined in \sphinxtitleref{ce.CE}.

\item[{\sphinxstylestrong{params\_config\_dict}}] \leavevmode{[}dict{]}
Parameters used in to construct CE object and other parameters used
in GACE simulation. User can custom this dict for their own needs.

\item[{\sphinxstylestrong{energy\_database\_fname}}] \leavevmode{[}str{]}
Filename of file that restore energies for different configurations
to accelerate energy-calculation of a energy-unknown configuration.

\item[{\sphinxstylestrong{toolbox}}] \leavevmode{[}ToolBox{]}
The ToolBox object defined in \sphinxtitleref{deap.tools}.

\item[{\sphinxstylestrong{DEFAULT\_SETUP}}] \leavevmode{[}dict{]}
Class attribute which restores revelant parameters used in GA and
CE simulation process. See also \sphinxtitleref{params\_config\_dict} for custom.

\item[{\sphinxstylestrong{ENERGY\_DICAT}}] \leavevmode{[}dict{]}
A dict in which key is list of num representing a configuration and
value is the fitness value of the configuration, e.g., total energy or
formation energy of point defects.

\item[{\sphinxstylestrong{PREVIOUS\_COUNT}}] \leavevmode{[}int{]}
A parameter used to restore the execution step of previous simulation
in order to run from previous stop step.

\item[{\sphinxstylestrong{TYPES\_ENERGY\_DICT}}] \leavevmode{[}dict{]}
A dict restores different elements and their responding number index
in order to convert a element to a number in GA simulation, e.g.,
\{‘Hf’:1, ‘O’:2, ‘Vac’:3\}.

\item[{\sphinxstylestrong{TEMPLATE\_FILE\_STR}}] \leavevmode{[}str{]}
A string to restore the template of \sphinxtitleref{lat.in} which is a main
input file in \sphinxcode{ATAT}.

\end{description}

\end{description}\end{quote}
\index{DEFAULT\_SETUP (pygace.gace.AbstractApp attribute)}

\begin{fulllineitems}
\phantomsection\label{\detokenize{pygace:pygace.gace.AbstractApp.DEFAULT_SETUP}}\pysigline{\sphinxbfcode{DEFAULT\_SETUP}\sphinxbfcode{ = \{'TMP\_DIR': '/home/yxcheng/PycharmProjects/pygace/doc/tmp\_dir', 'TEMPLATE\_FILE': './data/lat\_in.template', 'TEST\_RES\_DIR': '/home/yxcheng/PycharmProjects/pygace/doc/res\_dir', 'NB\_DEFECT': None, 'DFT\_CAL\_DIR': './dft\_dirs', 'PICKLE\_DIR': '/home/yxcheng/PycharmProjects/pygace/doc/pickle\_bakup'\}}}
\end{fulllineitems}

\index{evalEnergy() (pygace.gace.AbstractApp method)}

\begin{fulllineitems}
\phantomsection\label{\detokenize{pygace:pygace.gace.AbstractApp.evalEnergy}}\pysiglinewithargsret{\sphinxbfcode{evalEnergy}}{\emph{individual}}{}~
\end{fulllineitems}

\index{get\_ce() (pygace.gace.AbstractApp method)}

\begin{fulllineitems}
\phantomsection\label{\detokenize{pygace:pygace.gace.AbstractApp.get_ce}}\pysiglinewithargsret{\sphinxbfcode{get\_ce}}{}{}
obtain inner ce object
\begin{quote}\begin{description}
\item[{Returns}] \leavevmode\begin{description}
\item[{\sphinxstylestrong{CE object}}] \leavevmode
\end{description}

\end{description}\end{quote}

\end{fulllineitems}

\index{get\_energy\_info\_from\_database() (pygace.gace.AbstractApp method)}

\begin{fulllineitems}
\phantomsection\label{\detokenize{pygace:pygace.gace.AbstractApp.get_energy_info_from_database}}\pysiglinewithargsret{\sphinxbfcode{get\_energy\_info\_from\_database}}{}{}
Initial energy database
\begin{quote}\begin{description}
\item[{Returns}] \leavevmode\begin{description}
\item[{\sphinxstylestrong{None}}] \leavevmode
\end{description}

\end{description}\end{quote}

\end{fulllineitems}

\index{ind\_to\_elis() (pygace.gace.AbstractApp method)}

\begin{fulllineitems}
\phantomsection\label{\detokenize{pygace:pygace.gace.AbstractApp.ind_to_elis}}\pysiglinewithargsret{\sphinxbfcode{ind\_to\_elis}}{\emph{individual}}{}
Convert a object used in GA to a object used in \sphinxcode{ATAT}.

This method is used to convert a list which contains number
to a list containing chemistry element, e.g., {[}2,2,1,3{]} to
{[}‘Hf’, ‘Hf’, ‘O’, ‘Vac’{]}
\begin{quote}\begin{description}
\item[{Parameters}] \leavevmode\begin{description}
\item[{\sphinxstylestrong{individual: list}}] \leavevmode
Convert a list of \sphinxcode{int} to a list of chemistry element.

\end{description}

\item[{Returns}] \leavevmode\begin{description}
\item[{\sphinxstylestrong{None}}] \leavevmode
\end{description}

\item[{Raises}] \leavevmode\begin{description}
\item[{\sphinxstylestrong{NotImplementedError}}] \leavevmode
This method must be implemented in subclass.

\end{description}

\end{description}\end{quote}

\end{fulllineitems}

\index{initial() (pygace.gace.AbstractApp method)}

\begin{fulllineitems}
\phantomsection\label{\detokenize{pygace:pygace.gace.AbstractApp.initial}}\pysiglinewithargsret{\sphinxbfcode{initial}}{}{}
Initialization for GA simulation.
\begin{quote}\begin{description}
\item[{Returns}] \leavevmode\begin{description}
\item[{\sphinxstylestrong{toolbox}}] \leavevmode{[}Toolbox{]}
A Toolbox object contains responding parameters used in GA.

\end{description}

\end{description}\end{quote}

\end{fulllineitems}

\index{run() (pygace.gace.AbstractApp method)}

\begin{fulllineitems}
\phantomsection\label{\detokenize{pygace:pygace.gace.AbstractApp.run}}\pysiglinewithargsret{\sphinxbfcode{run}}{\emph{iter\_idx=1}, \emph{target\_epoch=0}}{}~\begin{quote}\begin{description}
\item[{Parameters}] \leavevmode\begin{description}
\item[{\sphinxstylestrong{iter\_idx}}] \leavevmode{[}int{]}
The index of GA-to-CE iteration, in which a DFT calculation is
usually executed for update \sphinxcode{eci.out} file in \sphinxcode{ATAT}.

\item[{\sphinxstylestrong{target\_epoch}}] \leavevmode{[}int{]}
The repeat times of identical simulation of GA, for which the
results of GA simulation is relevant with random number, thus
a different GA simulation maybe select a different ground-state
configuration. This is useful especially in complex system with
substantial \sphinxcode{sites} to substitute for different configurations.

\end{description}

\item[{Returns}] \leavevmode\begin{description}
\item[{\sphinxstylestrong{None}}] \leavevmode
\end{description}

\item[{Raises}] \leavevmode\begin{description}
\item[{\sphinxstylestrong{NotImplementedError}}] \leavevmode
If this method is not implemented, this type error would be raised.

\end{description}

\end{description}\end{quote}

\end{fulllineitems}

\index{set\_dir() (pygace.gace.AbstractApp method)}

\begin{fulllineitems}
\phantomsection\label{\detokenize{pygace:pygace.gace.AbstractApp.set_dir}}\pysiglinewithargsret{\sphinxbfcode{set\_dir}}{}{}
Initial directory.
\begin{quote}\begin{description}
\item[{Returns}] \leavevmode\begin{description}
\item[{\sphinxstylestrong{None}}] \leavevmode
\end{description}

\end{description}\end{quote}

\end{fulllineitems}

\index{transver\_to\_struct() (pygace.gace.AbstractApp method)}

\begin{fulllineitems}
\phantomsection\label{\detokenize{pygace:pygace.gace.AbstractApp.transver_to_struct}}\pysiglinewithargsret{\sphinxbfcode{transver\_to\_struct}}{\emph{element\_lis}}{}
Convert element list to \sphinxtitleref{ATAT} \sphinxtitleref{str.out} file

The chemistry symbol in \sphinxcode{element\_lis} would be substituted in
\sphinxcode{str.out} file in \sphinxcode{ATAT}.
\begin{quote}\begin{description}
\item[{Parameters}] \leavevmode\begin{description}
\item[{\sphinxstylestrong{element\_lis}}] \leavevmode{[}list{]}
a list of chemistry symbol, e.g. {[}‘Hf’, ‘Hf’, ‘O’{]}

\item[{\sphinxstylestrong{test\_param1}}] \leavevmode{[}int{]}
the first test parameter

\end{description}

\item[{Returns}] \leavevmode\begin{description}
\item[{\sphinxstylestrong{str}}] \leavevmode
filename of \sphinxcode{ATAT} structure file, default \sphinxtitleref{str.out}

\end{description}

\end{description}\end{quote}

\end{fulllineitems}

\index{update\_ce() (pygace.gace.AbstractApp method)}

\begin{fulllineitems}
\phantomsection\label{\detokenize{pygace:pygace.gace.AbstractApp.update_ce}}\pysiglinewithargsret{\sphinxbfcode{update\_ce}}{\emph{site=1}, \emph{dirname=None}}{}
Update inner CE object.

The parameters should contained the \sphinxcode{site} information in \sphinxcode{MMAPS}
and a path of directory containing output file after a CE fitting.
\begin{quote}\begin{description}
\item[{Parameters}] \leavevmode\begin{description}
\item[{\sphinxstylestrong{site: :obj: {}`int{}`, optional}}] \leavevmode
The number of \sphinxcode{site} in a crystal structure, which does not
contain a specific element instead of a site used to restore
different type of atoms to simulate alloy configurations in
\sphinxcode{ATAT}, more detail see \sphinxcode{lat.in} file in \sphinxcode{ATAT}.

\item[{\sphinxstylestrong{dirname: :obj: {}`str{}`, optional}}] \leavevmode
\end{description}

\item[{Returns}] \leavevmode\begin{description}
\item[{\sphinxstylestrong{None}}] \leavevmode
\end{description}

\end{description}\end{quote}

\end{fulllineitems}


\end{fulllineitems}

\index{AbstractRunner (class in pygace.gace)}

\begin{fulllineitems}
\phantomsection\label{\detokenize{pygace:pygace.gace.AbstractRunner}}\pysiglinewithargsret{\sphinxbfcode{class }\sphinxcode{pygace.gace.}\sphinxbfcode{AbstractRunner}}{\emph{app=None}, \emph{iter\_idx=None}}{}
Bases: \sphinxcode{object}

Abstract Runner for running a GACE simulation.

This object is used to execute a GACE simulation, user only need to
implement several interfaces to custom their application.
\begin{quote}\begin{description}
\item[{Parameters}] \leavevmode\begin{description}
\item[{\sphinxstylestrong{app}}] \leavevmode{[}subclass of AbstractApp{]}
A subclass object of AbstractApp, default is \sphinxtitleref{None}.

\item[{\sphinxstylestrong{iter\_idx}}] \leavevmode{[}int{]}
Index of GA-to-CE iteration, default is \sphinxtitleref{None}.

\end{description}

\item[{Raises}] \leavevmode\begin{description}
\item[{\sphinxstylestrong{NotImplementedError}}] \leavevmode
If \sphinxtitleref{run()} or \sphinxtitleref{print\_gs()} method is not implemented by subclass of
\sphinxtitleref{AbstractRunner}, this type of error would be raised.

\end{description}

\item[{Attributes}] \leavevmode\begin{description}
\item[{\sphinxstylestrong{app}}] \leavevmode{[}AbstractApp{]}
A subclass object of AbstractApp.

\item[{\sphinxstylestrong{iter\_idx}}] \leavevmode{[}int{]}
Index of GA-to-CE iteration.

\end{description}

\end{description}\end{quote}
\index{app (pygace.gace.AbstractRunner attribute)}

\begin{fulllineitems}
\phantomsection\label{\detokenize{pygace:pygace.gace.AbstractRunner.app}}\pysigline{\sphinxbfcode{app}}~
\end{fulllineitems}

\index{iter\_idx (pygace.gace.AbstractRunner attribute)}

\begin{fulllineitems}
\phantomsection\label{\detokenize{pygace:pygace.gace.AbstractRunner.iter_idx}}\pysigline{\sphinxbfcode{iter\_idx}}~
\end{fulllineitems}

\index{print\_gs() (pygace.gace.AbstractRunner method)}

\begin{fulllineitems}
\phantomsection\label{\detokenize{pygace:pygace.gace.AbstractRunner.print_gs}}\pysiglinewithargsret{\sphinxbfcode{print\_gs}}{}{}
Function used to check ground-state configurations, to obtain their
formation energy predicted by CE, and to determine whether a DFT
calculation is needed to executed for next GA-to-CE iteration.
\begin{quote}\begin{description}
\item[{Returns}] \leavevmode\begin{description}
\item[{\sphinxstylestrong{None}}] \leavevmode
\end{description}

\item[{Raises}] \leavevmode\begin{description}
\item[{\sphinxstylestrong{NotImplementedError}}] \leavevmode
if this function is not implemented in their subclass, this type
error would be raised.

\end{description}

\end{description}\end{quote}

\end{fulllineitems}

\index{run() (pygace.gace.AbstractRunner method)}

\begin{fulllineitems}
\phantomsection\label{\detokenize{pygace:pygace.gace.AbstractRunner.run}}\pysiglinewithargsret{\sphinxbfcode{run}}{}{}
Main runction for running GACE simulation.
\begin{quote}\begin{description}
\item[{Returns}] \leavevmode\begin{description}
\item[{\sphinxstylestrong{None}}] \leavevmode
\end{description}

\item[{Raises}] \leavevmode\begin{description}
\item[{\sphinxstylestrong{NotImplementedError}}] \leavevmode
if this function is not implemented in their subclass, this type
error would be raised.

\end{description}

\end{description}\end{quote}

\end{fulllineitems}


\end{fulllineitems}



\subsection{pygace.general\_gace module}
\label{\detokenize{pygace:pygace-general-gace-module}}\label{\detokenize{pygace:module-pygace.general_gace}}\index{pygace.general\_gace (module)}
A GA-to-CE example of oxygen-vacancy-containing HfO2 system given in
this module.
\index{GeneralApp (class in pygace.general\_gace)}

\begin{fulllineitems}
\phantomsection\label{\detokenize{pygace:pygace.general_gace.GeneralApp}}\pysiglinewithargsret{\sphinxbfcode{class }\sphinxcode{pygace.general\_gace.}\sphinxbfcode{GeneralApp}}{\emph{ele\_type\_list}, \emph{defect\_concentrations}, \emph{ce\_dirname='./data/iter1'}, \emph{params\_config\_dict=None}}{}
Bases: {\hyperref[\detokenize{pygace:pygace.gace.AbstractApp}]{\sphinxcrossref{\sphinxcode{pygace.gace.AbstractApp}}}}

An app of general system which is implemented from AbstractApp object

This object is used to execute a GACE simulation, user only need to
implement several interfaces to custom their application.
\begin{quote}\begin{description}
\item[{Parameters}] \leavevmode\begin{description}
\item[{\sphinxstylestrong{ce\_site: int}}] \leavevmode
the concept of site used in ATAT program.

\item[{\sphinxstylestrong{ce\_dirname: str}}] \leavevmode
The name of a directory which contain information of MMAPS or MAPS
running

\item[{\sphinxstylestrong{ele\_1st: str}}] \leavevmode
The first type of element in the \sphinxcode{site} in \sphinxcode{ATAT}.

\item[{\sphinxstylestrong{ele\_2nd: str}}] \leavevmode
The second type of element in the \sphinxcode{site} in \sphinxcode{ATAT}.

\item[{\sphinxstylestrong{params\_config\_dict: dirt}}] \leavevmode
Parameter dict used to custom GACE AbstractApp.

\end{description}

\item[{Attributes}] \leavevmode\begin{description}
\item[{\sphinxstylestrong{app}}] \leavevmode{[}AbstractApp{]}
A subclass object of AbstractApp.

\item[{\sphinxstylestrong{iter\_idx}}] \leavevmode{[}int{]}
Index of GA-to-CE iteration.

\end{description}

\end{description}\end{quote}
\index{evalEnergy() (pygace.general\_gace.GeneralApp method)}

\begin{fulllineitems}
\phantomsection\label{\detokenize{pygace:pygace.general_gace.GeneralApp.evalEnergy}}\pysiglinewithargsret{\sphinxbfcode{evalEnergy}}{\emph{individual}}{}
Evaluation function for the ground-state searching problem.

The problem is to determine a configuration of n vacancies
on a crystalline structures such that the energy of crystalline
structures can obtain minimum value.
\begin{quote}\begin{description}
\item[{Parameters}] \leavevmode\begin{description}
\item[{\sphinxstylestrong{individual}}] \leavevmode
\end{description}

\item[{Returns}] \leavevmode\begin{description}
\item[{\sphinxstylestrong{float}}] \leavevmode
Fittness value

\end{description}

\end{description}\end{quote}

\end{fulllineitems}

\index{ind\_to\_elis() (pygace.general\_gace.GeneralApp method)}

\begin{fulllineitems}
\phantomsection\label{\detokenize{pygace:pygace.general_gace.GeneralApp.ind_to_elis}}\pysiglinewithargsret{\sphinxbfcode{ind\_to\_elis}}{\emph{individual}}{}
Convert individual (number list) to element list
\begin{quote}\begin{description}
\item[{Parameters}] \leavevmode\begin{description}
\item[{\sphinxstylestrong{individual}}] \leavevmode
\end{description}

\item[{Returns}] \leavevmode\begin{description}
\item[{\sphinxstylestrong{list}}] \leavevmode
A list of element symbol string.

\end{description}

\end{description}\end{quote}

\end{fulllineitems}

\index{run() (pygace.general\_gace.GeneralApp method)}

\begin{fulllineitems}
\phantomsection\label{\detokenize{pygace:pygace.general_gace.GeneralApp.run}}\pysiglinewithargsret{\sphinxbfcode{run}}{\emph{iter\_idx=1}, \emph{default\_epoch=4}, \emph{target\_epoch=4}, \emph{cross\_method=1}, \emph{cross\_num=8}, \emph{cp\_fname\_prefix='ground\_states\_iter'}, \emph{task\_prefix='general-app'}, \emph{gs\_selection=1}}{}
Main function to run a GACE simulation which will be called by
\sphinxtitleref{AbstractRunner}.
\begin{quote}\begin{description}
\item[{Parameters}] \leavevmode\begin{description}
\item[{\sphinxstylestrong{iter\_idx}}] \leavevmode{[}int{]}
Determine which iteration the ECI is used in.

\item[{\sphinxstylestrong{target\_epoch}}] \leavevmode{[}int{]}
Iteration in GA simulation.

\item[{\sphinxstylestrong{default\_epoch}}] \leavevmode{[}int{]}
Default epoch setting for GA.

\item[{\sphinxstylestrong{target\_epoch}}] \leavevmode{[}int{]}
Target epoch for GA.

\item[{\sphinxstylestrong{cross\_method}}] \leavevmode{[}int{]}
Crossover operator type.

\item[{\sphinxstylestrong{cross\_num :}}] \leavevmode
The exchange number used in crossover operator.

\item[{\sphinxstylestrong{cp\_fname\_prefix}}] \leavevmode{[}str{]}
The prefix of checkpoint file name.

\item[{\sphinxstylestrong{task\_prefix}}] \leavevmode{[}str{]}
The prefix of task filename of a single simulation.

\item[{\sphinxstylestrong{gs\_selection}}] \leavevmode{[}int{]}
Ground-state structures selected from \sphinxtitleref{target\_epoch} GA simulation.

\end{description}

\item[{Returns}] \leavevmode\begin{description}
\item[{\sphinxstylestrong{None}}] \leavevmode
\end{description}

\end{description}\end{quote}

\end{fulllineitems}

\index{update\_defect\_concentration() (pygace.general\_gace.GeneralApp method)}

\begin{fulllineitems}
\phantomsection\label{\detokenize{pygace:pygace.general_gace.GeneralApp.update_defect_concentration}}\pysiglinewithargsret{\sphinxbfcode{update\_defect\_concentration}}{\emph{c=None}}{}~
\end{fulllineitems}


\end{fulllineitems}

\index{GeneralEleIndv (class in pygace.general\_gace)}

\begin{fulllineitems}
\phantomsection\label{\detokenize{pygace:pygace.general_gace.GeneralEleIndv}}\pysiglinewithargsret{\sphinxbfcode{class }\sphinxcode{pygace.general\_gace.}\sphinxbfcode{GeneralEleIndv}}{\emph{ele\_lis}, \emph{app=None}}{}
Bases: {\hyperref[\detokenize{pygace:pygace.utility.EleIndv}]{\sphinxcrossref{\sphinxcode{pygace.utility.EleIndv}}}}

A class that use list chemistry element to represent individual.
\begin{quote}\begin{description}
\item[{Parameters}] \leavevmode\begin{description}
\item[{\sphinxstylestrong{ele\_lis}}] \leavevmode{[}list{]}
A list of chemistry element.

\item[{\sphinxstylestrong{app}}] \leavevmode{[}AbstractApp{]}
An application of GACE which is used to obtain ground-state
structures based generic algorithm and cluster expansion method.

\end{description}

\item[{Attributes}] \leavevmode\begin{description}
\item[{\sphinxstylestrong{app: AbstractApp}}] \leavevmode
An application handling GACE running process.

\item[{\sphinxstylestrong{ele\_lis: list}}] \leavevmode
A list of chemistry element string.

\end{description}

\end{description}\end{quote}
\index{ce\_energy (pygace.general\_gace.GeneralEleIndv attribute)}

\begin{fulllineitems}
\phantomsection\label{\detokenize{pygace:pygace.general_gace.GeneralEleIndv.ce_energy}}\pysigline{\sphinxbfcode{ce\_energy}}
Return CE energy
\begin{quote}\begin{description}
\item[{Returns}] \leavevmode\begin{description}
\item[{\sphinxstylestrong{float}}] \leavevmode
Energy predicted by CE method.

\end{description}

\end{description}\end{quote}

\end{fulllineitems}

\index{ce\_energy\_corrdump (pygace.general\_gace.GeneralEleIndv attribute)}

\begin{fulllineitems}
\phantomsection\label{\detokenize{pygace:pygace.general_gace.GeneralEleIndv.ce_energy_corrdump}}\pysigline{\sphinxbfcode{ce\_energy\_corrdump}}
Return relative energy defined in \sphinxcode{ATAT} and computed by \sphinxcode{corrdump}
program.
\begin{quote}\begin{description}
\item[{Returns}] \leavevmode\begin{description}
\item[{\sphinxstylestrong{float}}] \leavevmode
Relative energy generated by \sphinxcode{corrdump} program.

\end{description}

\end{description}\end{quote}

\end{fulllineitems}

\index{dft\_energy() (pygace.general\_gace.GeneralEleIndv method)}

\begin{fulllineitems}
\phantomsection\label{\detokenize{pygace:pygace.general_gace.GeneralEleIndv.dft_energy}}\pysiglinewithargsret{\sphinxbfcode{dft\_energy}}{\emph{iters=None}, \emph{vasp\_cmd=None}, \emph{update\_eci=True}}{}
Return DFT energy
\begin{quote}\begin{description}
\item[{Parameters}] \leavevmode\begin{description}
\item[{\sphinxstylestrong{iters}}] \leavevmode{[}int{]}
index of iteration of GA-to-CE

\end{description}

\item[{Returns}] \leavevmode\begin{description}
\item[{\sphinxstylestrong{None or float}}] \leavevmode
\end{description}

\end{description}\end{quote}

\end{fulllineitems}

\index{run\_fake\_vasp() (pygace.general\_gace.GeneralEleIndv method)}

\begin{fulllineitems}
\phantomsection\label{\detokenize{pygace:pygace.general_gace.GeneralEleIndv.run_fake_vasp}}\pysiglinewithargsret{\sphinxbfcode{run\_fake\_vasp}}{}{}~
\end{fulllineitems}

\index{run\_vasp() (pygace.general\_gace.GeneralEleIndv method)}

\begin{fulllineitems}
\phantomsection\label{\detokenize{pygace:pygace.general_gace.GeneralEleIndv.run_vasp}}\pysiglinewithargsret{\sphinxbfcode{run\_vasp}}{\emph{vasp\_cmd}}{}~
\end{fulllineitems}


\end{fulllineitems}

\index{Runner (class in pygace.general\_gace)}

\begin{fulllineitems}
\phantomsection\label{\detokenize{pygace:pygace.general_gace.Runner}}\pysiglinewithargsret{\sphinxbfcode{class }\sphinxcode{pygace.general\_gace.}\sphinxbfcode{Runner}}{\emph{app=None}, \emph{iter\_idx=None}}{}
Bases: {\hyperref[\detokenize{pygace:pygace.gace.AbstractRunner}]{\sphinxcrossref{\sphinxcode{pygace.gace.AbstractRunner}}}}

A runner for running a GACE simulation.

This object is used to execute a GACE simulation in HfO2 system.
\begin{quote}\begin{description}
\item[{Parameters}] \leavevmode\begin{description}
\item[{\sphinxstylestrong{app}}] \leavevmode{[}subclass of GeneralApp{]}
A subclass object of GeneralApp, default is \sphinxtitleref{None}.

\item[{\sphinxstylestrong{iter\_idx}}] \leavevmode{[}int{]}
Index of GA-to-CE iteration, default is \sphinxtitleref{None}.

\end{description}

\item[{Attributes}] \leavevmode\begin{description}
\item[{\sphinxstylestrong{app}}] \leavevmode{[}GeneralApp{]}
A subclass object of GeneralApp.

\item[{\sphinxstylestrong{iter\_idx}}] \leavevmode{[}int{]}
Index of GA-to-CE iteration.

\end{description}

\end{description}\end{quote}
\index{compare\_gs() (pygace.general\_gace.Runner method)}

\begin{fulllineitems}
\phantomsection\label{\detokenize{pygace:pygace.general_gace.Runner.compare_gs}}\pysiglinewithargsret{\sphinxbfcode{compare\_gs}}{\emph{new\_gs}, \emph{old\_gs}}{}
Determine whether current and previous ground-state are identical.
\begin{quote}\begin{description}
\item[{Parameters}] \leavevmode\begin{description}
\item[{\sphinxstylestrong{new\_gs}}] \leavevmode{[}str{]}
Ground-state configuration predicted by current iteration.

\item[{\sphinxstylestrong{old\_gs :}}] \leavevmode
Ground-state configuration predicted by previous iteration.

\end{description}

\item[{Returns}] \leavevmode\begin{description}
\item[{\sphinxstylestrong{bool}}] \leavevmode
\item[{\sphinxstylestrong{Raises:}}] \leavevmode
\item[{\sphinxstylestrong{RuntimeError}}] \leavevmode
when the number of point defect (oxygen vacancy here) is not equal
in two iteration.

\end{description}

\end{description}\end{quote}

\end{fulllineitems}

\index{print\_gs() (pygace.general\_gace.Runner method)}

\begin{fulllineitems}
\phantomsection\label{\detokenize{pygace:pygace.general_gace.Runner.print_gs}}\pysiglinewithargsret{\sphinxbfcode{print\_gs}}{\emph{vasp\_cmd=None}}{}
Function used to extract ground-state information from pickle file
saved during GACE running.
\begin{quote}\begin{description}
\item[{Returns}] \leavevmode\begin{description}
\item[{\sphinxstylestrong{None}}] \leavevmode
\end{description}

\end{description}\end{quote}

\end{fulllineitems}

\index{run() (pygace.general\_gace.Runner method)}

\begin{fulllineitems}
\phantomsection\label{\detokenize{pygace:pygace.general_gace.Runner.run}}\pysiglinewithargsret{\sphinxbfcode{run}}{}{}
Main function to run.
\begin{quote}\begin{description}
\item[{Returns}] \leavevmode\begin{description}
\item[{\sphinxstylestrong{None}}] \leavevmode
\end{description}

\end{description}\end{quote}

\end{fulllineitems}

\index{str2energy() (pygace.general\_gace.Runner method)}

\begin{fulllineitems}
\phantomsection\label{\detokenize{pygace:pygace.general_gace.Runner.str2energy}}\pysiglinewithargsret{\sphinxbfcode{str2energy}}{\emph{string}}{}
Obtain energy from string consists of numbers joined by ‘\_’, e.g.,
\sphinxcode{'1\_2\_3\_19\_'}, in which the number is the position index in lattice
structure template file.
\begin{quote}\begin{description}
\item[{Parameters}] \leavevmode\begin{description}
\item[{\sphinxstylestrong{string}}] \leavevmode{[}str{]}
The string consists by index of point defect.

\item[{\sphinxstylestrong{Returns}}] \leavevmode
\item[{\sphinxstylestrong{——-}}] \leavevmode
\item[{\sphinxstylestrong{float}}] \leavevmode
CE energy.

\end{description}

\end{description}\end{quote}

\end{fulllineitems}


\end{fulllineitems}



\subsection{pygace.parse module}
\label{\detokenize{pygace:module-pygace.parse}}\label{\detokenize{pygace:pygace-parse-module}}\index{pygace.parse (module)}\index{GaceMcsqs (class in pygace.parse)}

\begin{fulllineitems}
\phantomsection\label{\detokenize{pygace:pygace.parse.GaceMcsqs}}\pysiglinewithargsret{\sphinxbfcode{class }\sphinxcode{pygace.parse.}\sphinxbfcode{GaceMcsqs}}{\emph{structure}}{}~\index{nb\_occu\_sites (pygace.parse.GaceMcsqs attribute)}

\begin{fulllineitems}
\phantomsection\label{\detokenize{pygace:pygace.parse.GaceMcsqs.nb_occu_sites}}\pysigline{\sphinxbfcode{nb\_occu\_sites}}~
\end{fulllineitems}

\index{structure\_from\_string() (pygace.parse.GaceMcsqs static method)}

\begin{fulllineitems}
\phantomsection\label{\detokenize{pygace:pygace.parse.GaceMcsqs.structure_from_string}}\pysiglinewithargsret{\sphinxbfcode{static }\sphinxbfcode{structure\_from\_string}}{\emph{data}}{}
Parses a rndstr.in or lat.in file into pymatgen’s
Structure format.
\begin{quote}\begin{description}
\item[{Parameters}] \leavevmode
\sphinxstyleliteralstrong{data} \textendash{} contents of a rndstr.in or lat.in file

\item[{Returns}] \leavevmode
Structure object

\end{description}\end{quote}

\end{fulllineitems}

\index{to\_string() (pygace.parse.GaceMcsqs method)}

\begin{fulllineitems}
\phantomsection\label{\detokenize{pygace:pygace.parse.GaceMcsqs.to_string}}\pysiglinewithargsret{\sphinxbfcode{to\_string}}{}{}
Returns a structure in mcsqs rndstr.in format.
:return (str):

\end{fulllineitems}

\index{to\_template() (pygace.parse.GaceMcsqs method)}

\begin{fulllineitems}
\phantomsection\label{\detokenize{pygace:pygace.parse.GaceMcsqs.to_template}}\pysiglinewithargsret{\sphinxbfcode{to\_template}}{\emph{ele\_dict=None}}{}
Returns a structure in lat.in format template.
\begin{quote}\begin{description}
\item[{Parameters}] \leavevmode\begin{description}
\item[{\sphinxstylestrong{ele\_dict}}] \leavevmode{[}dict{]}
A dict used to convert element type in ATAT to
element type in pymatgen.

\end{description}

\item[{Returns}] \leavevmode\begin{description}
\item[{\sphinxstylestrong{str}}] \leavevmode
\end{description}

\end{description}\end{quote}

\end{fulllineitems}


\end{fulllineitems}



\subsection{pygace.utility module}
\label{\detokenize{pygace:module-pygace.utility}}\label{\detokenize{pygace:pygace-utility-module}}\index{pygace.utility (module)}
There are some general helper function defined in this module.
\index{EleIndv (class in pygace.utility)}

\begin{fulllineitems}
\phantomsection\label{\detokenize{pygace:pygace.utility.EleIndv}}\pysiglinewithargsret{\sphinxbfcode{class }\sphinxcode{pygace.utility.}\sphinxbfcode{EleIndv}}{\emph{ele\_lis}, \emph{app=None}}{}
Bases: \sphinxcode{object}

A class that use list chemistry element to represent individual.
\begin{quote}\begin{description}
\item[{Parameters}] \leavevmode\begin{description}
\item[{\sphinxstylestrong{ele\_lis}}] \leavevmode{[}list{]}
A list of chemistry element.

\item[{\sphinxstylestrong{app}}] \leavevmode{[}AbstractApp{]}
An application of GACE which is used to obtain ground-state
structures based generic algorithm and cluster expansion method.

\end{description}

\item[{Attributes}] \leavevmode\begin{description}
\item[{\sphinxstylestrong{app: AbstractApp}}] \leavevmode
An application handling GACE running process.

\item[{\sphinxstylestrong{ele\_lis: list}}] \leavevmode
A list of chemistry element string.

\end{description}

\end{description}\end{quote}
\index{ce\_energy (pygace.utility.EleIndv attribute)}

\begin{fulllineitems}
\phantomsection\label{\detokenize{pygace:pygace.utility.EleIndv.ce_energy}}\pysigline{\sphinxbfcode{ce\_energy}}
The absolute energy predicted by CE.
\begin{quote}\begin{description}
\item[{Returns}] \leavevmode\begin{description}
\item[{\sphinxstylestrong{float}}] \leavevmode
CE absolute energy.

\end{description}

\end{description}\end{quote}

\end{fulllineitems}

\index{ce\_energy\_ref (pygace.utility.EleIndv attribute)}

\begin{fulllineitems}
\phantomsection\label{\detokenize{pygace:pygace.utility.EleIndv.ce_energy_ref}}\pysigline{\sphinxbfcode{ce\_energy\_ref}}
The relative energy predicted by CE.
\begin{quote}\begin{description}
\item[{Returns}] \leavevmode\begin{description}
\item[{\sphinxstylestrong{float}}] \leavevmode
CE relative energy

\end{description}

\end{description}\end{quote}

\end{fulllineitems}

\index{ce\_object (pygace.utility.EleIndv attribute)}

\begin{fulllineitems}
\phantomsection\label{\detokenize{pygace:pygace.utility.EleIndv.ce_object}}\pysigline{\sphinxbfcode{ce\_object}}~
\end{fulllineitems}

\index{dft\_energy() (pygace.utility.EleIndv method)}

\begin{fulllineitems}
\phantomsection\label{\detokenize{pygace:pygace.utility.EleIndv.dft_energy}}\pysiglinewithargsret{\sphinxbfcode{dft\_energy}}{\emph{iters=None}}{}
The DFT energy of individual represented by element list.
\begin{quote}\begin{description}
\item[{Parameters}] \leavevmode\begin{description}
\item[{\sphinxstylestrong{iters}}] \leavevmode{[}int{]}
Specific which iteration DFT energy are computed.

\end{description}

\item[{Returns}] \leavevmode\begin{description}
\item[{\sphinxstylestrong{float or None}}] \leavevmode
If the directory of DFT calculated exists and the calculation has
been finished the DFT energy will be return, or a new DFT
calculation directory will be created and first-principles
calculation should be performed in this directory.

\end{description}

\end{description}\end{quote}

\end{fulllineitems}

\index{is\_correct() (pygace.utility.EleIndv method)}

\begin{fulllineitems}
\phantomsection\label{\detokenize{pygace:pygace.utility.EleIndv.is_correct}}\pysiglinewithargsret{\sphinxbfcode{is\_correct}}{}{}
Determine whether the dft energy and the ce energy of indv equivalent
are identical within error.
\begin{quote}\begin{description}
\item[{Returns}] \leavevmode\begin{description}
\item[{\sphinxstylestrong{bool}}] \leavevmode
\end{description}

\end{description}\end{quote}

\end{fulllineitems}

\index{set\_app() (pygace.utility.EleIndv method)}

\begin{fulllineitems}
\phantomsection\label{\detokenize{pygace:pygace.utility.EleIndv.set_app}}\pysiglinewithargsret{\sphinxbfcode{set\_app}}{\emph{app}}{}~
\end{fulllineitems}


\end{fulllineitems}

\index{compare\_crystal() (in module pygace.utility)}

\begin{fulllineitems}
\phantomsection\label{\detokenize{pygace:pygace.utility.compare_crystal}}\pysiglinewithargsret{\sphinxcode{pygace.utility.}\sphinxbfcode{compare\_crystal}}{\emph{str1}, \emph{str2}, \emph{compare\_crystal\_cmd='CompareCrystal '}, \emph{str\_template=None}, \emph{**kwargs}}{}
To determine whether structures are identical based crystal symmetry
analysis. The program used in this package is based on \sphinxcode{XtalComp} library
which developed by David C. Lonie.
\begin{quote}\begin{description}
\item[{Parameters}] \leavevmode\begin{description}
\item[{\sphinxstylestrong{str1}}] \leavevmode{[}str{]}
The first string used to represent elements .

\item[{\sphinxstylestrong{str2}}] \leavevmode{[}str{]}
The second string used to represent elements.

\item[{\sphinxstylestrong{compare\_crystal\_cmd}}] \leavevmode{[}str{]}
The program developed to determine whether two
crystal structures are identical, default \sphinxtitleref{CompareCrystal}.

\item[{\sphinxstylestrong{str\_template}}] \leavevmode{[}str{]}
String template for the definition of lattice site.

\item[{\sphinxstylestrong{kwargs}}] \leavevmode{[}dict arguments{]}
Other arguments used in \sphinxtitleref{compare\_crystal\_cmd}.

\end{description}

\item[{Returns}] \leavevmode\begin{description}
\item[{\sphinxstylestrong{bool}}] \leavevmode
\end{description}

\end{description}\end{quote}
\paragraph{References}

\sphinxurl{https://github.com/allisonvacanti/XtalComp}

\end{fulllineitems}

\index{copytree() (in module pygace.utility)}

\begin{fulllineitems}
\phantomsection\label{\detokenize{pygace:pygace.utility.copytree}}\pysiglinewithargsret{\sphinxcode{pygace.utility.}\sphinxbfcode{copytree}}{\emph{src}, \emph{dst}, \emph{symlinks=False}, \emph{ignore=None}}{}
Recursively copy a directory tree using copy2().

The destination directory must not already exist.
If exception(s) occur, an Error is raised with a list of reasons.

If the optional symlinks flag is true, symbolic links in the
source tree result in symbolic links in the destination tree; if
it is false, the contents of the files pointed to by symbolic
links are copied.

The optional ignore argument is a callable. If given, it
is called with the \sphinxtitleref{src} parameter, which is the directory
being visited by copytree(), and \sphinxtitleref{names} which is the list of
\sphinxtitleref{src} contents, as returned by os.listdir():
\begin{quote}

callable(src, names) -\textgreater{} ignored\_names
\end{quote}

Since copytree() is called recursively, the callable will be
called once for each directory that is copied. It returns a
list of names relative to the \sphinxtitleref{src} directory that should
not be copied.

XXX Consider this example code rather than the ultimate tool.

\end{fulllineitems}

\index{get\_num\_lis() (in module pygace.utility)}

\begin{fulllineitems}
\phantomsection\label{\detokenize{pygace:pygace.utility.get_num_lis}}\pysiglinewithargsret{\sphinxcode{pygace.utility.}\sphinxbfcode{get\_num\_lis}}{\emph{nb\_Nb}, \emph{nb\_site}}{}
Get number list by given the number point defect and site defined in
lattice file
\begin{quote}\begin{description}
\item[{Parameters}] \leavevmode\begin{description}
\item[{\sphinxstylestrong{nb\_Nb}}] \leavevmode{[}the number of point defect{]}
\item[{\sphinxstylestrong{nb\_site}}] \leavevmode{[}int{]}
The number of site defined in lattice file

\end{description}

\item[{Yields}] \leavevmode\begin{description}
\item[{\sphinxstylestrong{All combinations.}}] \leavevmode
\end{description}

\end{description}\end{quote}

\end{fulllineitems}

\index{reverse\_dict() (in module pygace.utility)}

\begin{fulllineitems}
\phantomsection\label{\detokenize{pygace:pygace.utility.reverse_dict}}\pysiglinewithargsret{\sphinxcode{pygace.utility.}\sphinxbfcode{reverse\_dict}}{\emph{d}}{}
Exchange \sphinxtitleref{key} and \sphinxtitleref{value} of given dict
\begin{quote}\begin{description}
\item[{Parameters}] \leavevmode\begin{description}
\item[{\sphinxstylestrong{d}}] \leavevmode{[}dict{]}
A dict needed to be converted.

\end{description}

\item[{Returns}] \leavevmode\begin{description}
\item[{\sphinxstylestrong{Dict}}] \leavevmode
The new dict in which \sphinxtitleref{key} and \sphinxtitleref{value} are exchanged with respect to
original dict.

\end{description}

\end{description}\end{quote}

\end{fulllineitems}

\index{save\_to\_pickle() (in module pygace.utility)}

\begin{fulllineitems}
\phantomsection\label{\detokenize{pygace:pygace.utility.save_to_pickle}}\pysiglinewithargsret{\sphinxcode{pygace.utility.}\sphinxbfcode{save\_to\_pickle}}{\emph{f}, \emph{python\_obj}}{}
Save python object in pickle file.
\begin{quote}\begin{description}
\item[{Parameters}] \leavevmode\begin{description}
\item[{\sphinxstylestrong{f}}] \leavevmode{[}fileobj{]}
File object to restore python object

\item[{\sphinxstylestrong{python\_obj}}] \leavevmode{[}obj{]}
Object need to be saved.

\end{description}

\item[{Returns}] \leavevmode\begin{description}
\item[{\sphinxstylestrong{None}}] \leavevmode
\end{description}

\end{description}\end{quote}

\end{fulllineitems}



\subsection{Module contents}
\label{\detokenize{pygace:module-pygace}}\label{\detokenize{pygace:module-contents}}\index{pygace (module)}
Searching the most stable atomic-structure of a solid with point defects
(including the extrinsic alloying/doping elements), is one of the central issues in
materials science. Both adequate sampling of the configuration space and the
accurate energy evaluation at relatively low cost are demanding for the structure
prediction. In this work, we have developed a framework combining genetic
algorithm, cluster expansion (CE) method and first-principles calculations, which
can effectively locate the ground-state or meta-stable states of the relatively
large/complex systems. We employ this framework to search the stable structures
of two distinct systems, i.e., oxygen-vacancy-containing HfO(2-x) and the
Nb-doped SrTi(1-x)NbxO3 , and more stable structures are found compared with
the structures available in the literature. The present framework can be applied
to the ground-state search of extensive alloyed/doped materials, which is
particularly significant for the design of advanced engineering alloys and
semiconductors.


\chapter{Indices and tables}
\label{\detokenize{index:indices-and-tables}}\begin{itemize}
\item {} 
\DUrole{xref,std,std-ref}{genindex}

\item {} 
\DUrole{xref,std,std-ref}{modindex}

\item {} 
\DUrole{xref,std,std-ref}{search}

\end{itemize}


\renewcommand{\indexname}{Python Module Index}
\begin{sphinxtheindex}
\def\bigletter#1{{\Large\sffamily#1}\nopagebreak\vspace{1mm}}
\bigletter{p}
\item {\sphinxstyleindexentry{pygace}}\sphinxstyleindexpageref{pygace:\detokenize{module-pygace}}
\item {\sphinxstyleindexentry{pygace.ce}}\sphinxstyleindexpageref{pygace:\detokenize{module-pygace.ce}}
\item {\sphinxstyleindexentry{pygace.config}}\sphinxstyleindexpageref{pygace:\detokenize{module-pygace.config}}
\item {\sphinxstyleindexentry{pygace.examples}}\sphinxstyleindexpageref{pygace.examples:\detokenize{module-pygace.examples}}
\item {\sphinxstyleindexentry{pygace.examples.general}}\sphinxstyleindexpageref{pygace.examples.general:\detokenize{module-pygace.examples.general}}
\item {\sphinxstyleindexentry{pygace.examples.general.test\_general}}\sphinxstyleindexpageref{pygace.examples.general:\detokenize{module-pygace.examples.general.test_general}}
\item {\sphinxstyleindexentry{pygace.examples.hfo2}}\sphinxstyleindexpageref{pygace.examples.hfo2:\detokenize{module-pygace.examples.hfo2}}
\item {\sphinxstyleindexentry{pygace.examples.hfo2.hfo2\_gace}}\sphinxstyleindexpageref{pygace.examples.hfo2:\detokenize{module-pygace.examples.hfo2.hfo2_gace}}
\item {\sphinxstyleindexentry{pygace.examples.sto}}\sphinxstyleindexpageref{pygace.examples.sto:\detokenize{module-pygace.examples.sto}}
\item {\sphinxstyleindexentry{pygace.examples.sto.sto\_gace}}\sphinxstyleindexpageref{pygace.examples.sto:\detokenize{module-pygace.examples.sto.sto_gace}}
\item {\sphinxstyleindexentry{pygace.ga}}\sphinxstyleindexpageref{pygace:\detokenize{module-pygace.ga}}
\item {\sphinxstyleindexentry{pygace.gace}}\sphinxstyleindexpageref{pygace:\detokenize{module-pygace.gace}}
\item {\sphinxstyleindexentry{pygace.general\_gace}}\sphinxstyleindexpageref{pygace:\detokenize{module-pygace.general_gace}}
\item {\sphinxstyleindexentry{pygace.parse}}\sphinxstyleindexpageref{pygace:\detokenize{module-pygace.parse}}
\item {\sphinxstyleindexentry{pygace.scripts}}\sphinxstyleindexpageref{pygace.scripts:\detokenize{module-pygace.scripts}}
\item {\sphinxstyleindexentry{pygace.scripts.rungace}}\sphinxstyleindexpageref{pygace.scripts:\detokenize{module-pygace.scripts.rungace}}
\item {\sphinxstyleindexentry{pygace.utility}}\sphinxstyleindexpageref{pygace:\detokenize{module-pygace.utility}}
\end{sphinxtheindex}

\renewcommand{\indexname}{Index}
\printindex
\end{document}