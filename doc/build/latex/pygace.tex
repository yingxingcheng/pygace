%% Generated by Sphinx.
\def\sphinxdocclass{report}
\documentclass[letterpaper,10pt,english]{sphinxmanual}
\ifdefined\pdfpxdimen
   \let\sphinxpxdimen\pdfpxdimen\else\newdimen\sphinxpxdimen
\fi \sphinxpxdimen=.75bp\relax
\ifdefined\pdfimageresolution
    \pdfimageresolution= \numexpr \dimexpr1in\relax/\sphinxpxdimen\relax
\fi
%% let collapsable pdf bookmarks panel have high depth per default
\PassOptionsToPackage{bookmarksdepth=5}{hyperref}

\PassOptionsToPackage{warn}{textcomp}
\usepackage[utf8]{inputenc}
\ifdefined\DeclareUnicodeCharacter
% support both utf8 and utf8x syntaxes
  \ifdefined\DeclareUnicodeCharacterAsOptional
    \def\sphinxDUC#1{\DeclareUnicodeCharacter{"#1}}
  \else
    \let\sphinxDUC\DeclareUnicodeCharacter
  \fi
  \sphinxDUC{00A0}{\nobreakspace}
  \sphinxDUC{2500}{\sphinxunichar{2500}}
  \sphinxDUC{2502}{\sphinxunichar{2502}}
  \sphinxDUC{2514}{\sphinxunichar{2514}}
  \sphinxDUC{251C}{\sphinxunichar{251C}}
  \sphinxDUC{2572}{\textbackslash}
\fi
\usepackage{cmap}
\usepackage[T1]{fontenc}
\usepackage{amsmath,amssymb,amstext}
\usepackage{babel}



\usepackage{tgtermes}
\usepackage{tgheros}
\renewcommand{\ttdefault}{txtt}



\usepackage[Bjarne]{fncychap}
\usepackage{sphinx}

\fvset{fontsize=auto}
\usepackage{geometry}


% Include hyperref last.
\usepackage{hyperref}
% Fix anchor placement for figures with captions.
\usepackage{hypcap}% it must be loaded after hyperref.
% Set up styles of URL: it should be placed after hyperref.
\urlstyle{same}

\addto\captionsenglish{\renewcommand{\contentsname}{Contents:}}

\usepackage{sphinxmessages}
\setcounter{tocdepth}{1}



\title{pygace Documentation}
\date{Aug 04, 2021}
\release{2018.12.13}
\author{Yingxing Cheng}
\newcommand{\sphinxlogo}{\vbox{}}
\renewcommand{\releasename}{Release}
\makeindex
\begin{document}

\pagestyle{empty}
\sphinxmaketitle
\pagestyle{plain}
\sphinxtableofcontents
\pagestyle{normal}
\phantomsection\label{\detokenize{index::doc}}



\chapter{pygace}
\label{\detokenize{modules:pygace}}\label{\detokenize{modules::doc}}

\section{pygace package}
\label{\detokenize{pygace:pygace-package}}\label{\detokenize{pygace::doc}}

\subsection{Subpackages}
\label{\detokenize{pygace:subpackages}}

\subsubsection{pygace.scripts package}
\label{\detokenize{pygace.scripts:pygace-scripts-package}}\label{\detokenize{pygace.scripts::doc}}

\paragraph{Submodules}
\label{\detokenize{pygace.scripts:submodules}}

\paragraph{pygace.scripts.rungace module}
\label{\detokenize{pygace.scripts:module-pygace.scripts.rungace}}\label{\detokenize{pygace.scripts:pygace-scripts-rungace-module}}\index{module@\spxentry{module}!pygace.scripts.rungace@\spxentry{pygace.scripts.rungace}}\index{pygace.scripts.rungace@\spxentry{pygace.scripts.rungace}!module@\spxentry{module}}
\sphinxAtStartPar
Searching the most stable atomic\sphinxhyphen{}structure of a solid with point defects
(including the extrinsic alloying/doping elements), is one of the central issues in
materials science. Both adequate sampling of the configuration space and the
accurate energy evaluation at relatively low cost are demanding for the structure
prediction. In this work, we have developed a framework combining genetic
algorithm, cluster expansion (CE) method and first\sphinxhyphen{}principles calculations, which
can effectively locate the ground\sphinxhyphen{}state or meta\sphinxhyphen{}stable states of the relatively
large/complex systems. We employ this framework to search the stable structures
of two distinct systems, i.e., oxygen\sphinxhyphen{}vacancy\sphinxhyphen{}containing HfO(2\sphinxhyphen{}x) and the
Nb\sphinxhyphen{}doped SrTi(1\sphinxhyphen{}x)NbxO3 , and more stable structures are found compared with
the structures available in the literature. The present framework can be applied
to the ground\sphinxhyphen{}state search of extensive alloyed/doped materials, which is
particularly significant for the design of advanced engineering alloys and
semiconductors.
\index{build\_supercell\_template() (in module pygace.scripts.rungace)@\spxentry{build\_supercell\_template()}\spxextra{in module pygace.scripts.rungace}}

\begin{fulllineitems}
\phantomsection\label{\detokenize{pygace.scripts:pygace.scripts.rungace.build_supercell_template}}\pysiglinewithargsret{\sphinxcode{\sphinxupquote{pygace.scripts.rungace.}}\sphinxbfcode{\sphinxupquote{build\_supercell\_template}}}{\emph{\DUrole{n}{scale}}}{}
\sphinxAtStartPar
Create supercell for GA\sphinxhyphen{}to\sphinxhyphen{}CE simulation.
\begin{quote}\begin{description}
\item[{Parameters}] \leavevmode\begin{description}
\item[{\sphinxstylestrong{scale}}] \leavevmode{[}list or arrary like{]}
\sphinxAtStartPar
A list used to determine the size of supercell.

\end{description}

\item[{Returns}] \leavevmode\begin{description}
\item[{None}] \leavevmode
\end{description}

\end{description}\end{quote}

\end{fulllineitems}

\index{rungace() (in module pygace.scripts.rungace)@\spxentry{rungace()}\spxextra{in module pygace.scripts.rungace}}

\begin{fulllineitems}
\phantomsection\label{\detokenize{pygace.scripts:pygace.scripts.rungace.rungace}}\pysiglinewithargsret{\sphinxcode{\sphinxupquote{pygace.scripts.rungace.}}\sphinxbfcode{\sphinxupquote{rungace}}}{\emph{\DUrole{n}{cell\_scale}}, \emph{\DUrole{n}{ele\_list}}, \emph{\DUrole{n}{ele\_nb}}, \emph{\DUrole{n}{max\_lis}}, \emph{\DUrole{o}{*}\DUrole{n}{args}}, \emph{\DUrole{o}{**}\DUrole{n}{kwargs}}}{}
\sphinxAtStartPar
Command for running GA\sphinxhyphen{}to\sphinxhyphen{}CE simulation.
\begin{quote}\begin{description}
\item[{Parameters}] \leavevmode\begin{description}
\item[{\sphinxstylestrong{cell\_scale}}] \leavevmode{[}list or array like{]}
\sphinxAtStartPar
A list used to specify the size of supercell.

\item[{\sphinxstylestrong{ele\_list}}] \leavevmode{[}list{]}
\sphinxAtStartPar
A list of elements contained in structure.

\item[{\sphinxstylestrong{ele\_nb}}] \leavevmode{[}list{]}
\sphinxAtStartPar
A list of maximum of the number of point defect in supercell structures.

\item[{\sphinxstylestrong{max\_lis}}] \leavevmode{[}int{]}
\sphinxAtStartPar
The maximum iterations

\end{description}

\item[{Returns}] \leavevmode\begin{description}
\item[{None}] \leavevmode
\end{description}

\end{description}\end{quote}

\end{fulllineitems}

\index{show\_results() (in module pygace.scripts.rungace)@\spxentry{show\_results()}\spxextra{in module pygace.scripts.rungace}}

\begin{fulllineitems}
\phantomsection\label{\detokenize{pygace.scripts:pygace.scripts.rungace.show_results}}\pysiglinewithargsret{\sphinxcode{\sphinxupquote{pygace.scripts.rungace.}}\sphinxbfcode{\sphinxupquote{show\_results}}}{\emph{\DUrole{n}{ele\_type\_list}}, \emph{\DUrole{n}{defect\_con\_list}}, \emph{\DUrole{n}{max\_ele\_concentration}}, \emph{\DUrole{n}{use\_nb\_iter}\DUrole{o}{=}\DUrole{default_value}{False}}, \emph{\DUrole{n}{nb\_iter\_gace}\DUrole{o}{=}\DUrole{default_value}{None}}, \emph{\DUrole{n}{vasp\_cmd}\DUrole{o}{=}\DUrole{default_value}{None}}, \emph{\DUrole{o}{*}\DUrole{n}{args}}, \emph{\DUrole{o}{**}\DUrole{n}{kwargs}}}{}
\sphinxAtStartPar
Show results of GA\sphinxhyphen{}to\sphinxhyphen{}CE simulation.
\begin{quote}\begin{description}
\item[{Parameters}] \leavevmode\begin{description}
\item[{\sphinxstylestrong{ele\_type\_list}}] \leavevmode{[}list{]}
\sphinxAtStartPar
A list of element symbols

\item[{\sphinxstylestrong{defect\_con\_list}}] \leavevmode{[}list{]}
\sphinxAtStartPar
A list of defect concentration in supercell structures.

\item[{\sphinxstylestrong{max\_ele\_concentration :}}] \leavevmode
\sphinxAtStartPar
The maximum element concentration.

\item[{\sphinxstylestrong{use\_nb\_iter}}] \leavevmode{[}bool, default=False{]}
\sphinxAtStartPar
Whether stop after {\color{red}\bfseries{}\textasciigrave{}}n\textasciigrave{}th iteration if it is not converged.

\item[{\sphinxstylestrong{nb\_iter\_gace}}] \leavevmode{[}bool{]}
\sphinxAtStartPar
Whether or not to determine stop criteria based on the number of iteration.

\item[{\sphinxstylestrong{vasp\_cmd}}] \leavevmode{[}str, default=None{]}
\sphinxAtStartPar
The command of \sphinxcode{\sphinxupquote{VASP}}.

\end{description}

\item[{Returns}] \leavevmode\begin{description}
\item[{None}] \leavevmode
\end{description}

\end{description}\end{quote}

\end{fulllineitems}



\paragraph{Module contents}
\label{\detokenize{pygace.scripts:module-pygace.scripts}}\label{\detokenize{pygace.scripts:module-contents}}\index{module@\spxentry{module}!pygace.scripts@\spxentry{pygace.scripts}}\index{pygace.scripts@\spxentry{pygace.scripts}!module@\spxentry{module}}

\subsection{Submodules}
\label{\detokenize{pygace:submodules}}

\subsection{pygace.ce module}
\label{\detokenize{pygace:module-pygace.ce}}\label{\detokenize{pygace:pygace-ce-module}}\index{module@\spxentry{module}!pygace.ce@\spxentry{pygace.ce}}\index{pygace.ce@\spxentry{pygace.ce}!module@\spxentry{module}}
\sphinxAtStartPar
The module wrapper cluster expansion method (CE) implemented in \sphinxcode{\sphinxupquote{MMAPS}} in
\sphinxcode{\sphinxupquote{ATAT}}.
\index{CE (class in pygace.ce)@\spxentry{CE}\spxextra{class in pygace.ce}}

\begin{fulllineitems}
\phantomsection\label{\detokenize{pygace:pygace.ce.CE}}\pysiglinewithargsret{\sphinxbfcode{\sphinxupquote{class }}\sphinxcode{\sphinxupquote{pygace.ce.}}\sphinxbfcode{\sphinxupquote{CE}}}{\emph{\DUrole{n}{lat\_in}\DUrole{o}{=}\DUrole{default_value}{None}}, \emph{\DUrole{n}{site}\DUrole{o}{=}\DUrole{default_value}{16}}, \emph{\DUrole{n}{corrdump\_cmd}\DUrole{o}{=}\DUrole{default_value}{None}}, \emph{\DUrole{n}{compare\_crystal\_cmd}\DUrole{o}{=}\DUrole{default_value}{None}}}{}
\sphinxAtStartPar
Bases: \sphinxcode{\sphinxupquote{object}}

\sphinxAtStartPar
An wrapper for commends in \sphinxcode{\sphinxupquote{ATAT}}.

\sphinxAtStartPar
This class provides several commands that are commonly used in \sphinxcode{\sphinxupquote{ATAT}}.
\begin{quote}\begin{description}
\item[{Parameters}] \leavevmode\begin{description}
\item[{\sphinxstylestrong{lat\_in}}] \leavevmode{[}str{]}
\sphinxAtStartPar
File name of \sphinxcode{\sphinxupquote{lat.in}} in \sphinxcode{\sphinxupquote{ATAT}}.

\item[{\sphinxstylestrong{site}}] \leavevmode{[}int{]}
\sphinxAtStartPar
The site in which different can occupy.

\item[{\sphinxstylestrong{corrdump\_cmd}}] \leavevmode{[}str{]}
\sphinxAtStartPar
Command of \sphinxcode{\sphinxupquote{corrdump}} in \sphinxcode{\sphinxupquote{ATAT}}, default is ‘corrdump’

\item[{\sphinxstylestrong{compare\_crystal\_cmd}}] \leavevmode{[}str{]}
\sphinxAtStartPar
Command of program which is used to determine whether two crystal
structures are identical using symmetry analysis.

\end{description}

\item[{Attributes}] \leavevmode\begin{description}
\item[{\sphinxstylestrong{COMPARE\_CRYSTAL}}] \leavevmode{[}str{]}
\sphinxAtStartPar
This string restore a command used to determine whether two
configurations are identical in symmetry.

\item[{\sphinxstylestrong{CORRDUMP}}] \leavevmode{[}str{]}
\sphinxAtStartPar
This string restore the command of \sphinxtitleref{corrdump} in \sphinxcode{\sphinxupquote{ATAT}}.

\item[{\sphinxstylestrong{clster\_info}}] \leavevmode{[}str{]}
\sphinxAtStartPar
Filename of cluster information, default is \sphinxcode{\sphinxupquote{clusters.out}}
in \sphinxcode{\sphinxupquote{ATAT}}.

\item[{\sphinxstylestrong{eci\_out}}] \leavevmode{[}str{]}
\sphinxAtStartPar
File name of \sphinxcode{\sphinxupquote{eci.out}} in \sphinxcode{\sphinxupquote{ATAT}}.

\item[{\sphinxstylestrong{lat\_in}}] \leavevmode{[}str{]}
\sphinxAtStartPar
File name of \sphinxtitleref{lat.in} in \sphinxcode{\sphinxupquote{ATAT}}.

\item[{\sphinxstylestrong{per\_atom\_energy}}] \leavevmode{[}dict{]}
\sphinxAtStartPar
A dict restore element type and their responding energy defined in
\sphinxcode{\sphinxupquote{atoms.out}} file in \sphinxcode{\sphinxupquote{ATAT}}

\item[{\sphinxstylestrong{site}}] \leavevmode{[}int{]}
\sphinxAtStartPar
The site in which different can occupy.

\item[{\sphinxstylestrong{work\_path}}] \leavevmode{[}str{]}
\sphinxAtStartPar
The directory of \sphinxcode{\sphinxupquote{MMAPS}} or \sphinxcode{\sphinxupquote{MAPS}} running.

\end{description}

\end{description}\end{quote}
\index{COMPARE\_CRYSTAL (pygace.ce.CE attribute)@\spxentry{COMPARE\_CRYSTAL}\spxextra{pygace.ce.CE attribute}}

\begin{fulllineitems}
\phantomsection\label{\detokenize{pygace:pygace.ce.CE.COMPARE_CRYSTAL}}\pysigline{\sphinxbfcode{\sphinxupquote{COMPARE\_CRYSTAL}}\sphinxbfcode{\sphinxupquote{ = None}}}
\end{fulllineitems}

\index{CORRDUMP (pygace.ce.CE attribute)@\spxentry{CORRDUMP}\spxextra{pygace.ce.CE attribute}}

\begin{fulllineitems}
\phantomsection\label{\detokenize{pygace:pygace.ce.CE.CORRDUMP}}\pysigline{\sphinxbfcode{\sphinxupquote{CORRDUMP}}\sphinxbfcode{\sphinxupquote{ = \textquotesingle{}/Users/yxcheng/softwares/atat/corrdump\textquotesingle{}}}}
\end{fulllineitems}

\index{compare\_crystal() (pygace.ce.CE static method)@\spxentry{compare\_crystal()}\spxextra{pygace.ce.CE static method}}

\begin{fulllineitems}
\phantomsection\label{\detokenize{pygace:pygace.ce.CE.compare_crystal}}\pysiglinewithargsret{\sphinxbfcode{\sphinxupquote{static }}\sphinxbfcode{\sphinxupquote{compare\_crystal}}}{\emph{\DUrole{n}{str1}}, \emph{\DUrole{n}{str2}}, \emph{\DUrole{n}{compare\_crystal\_cmd}\DUrole{o}{=}\DUrole{default_value}{None}}, \emph{\DUrole{o}{**}\DUrole{n}{kwargs}}}{}
\sphinxAtStartPar
To determine whether structures are identical based crystal symmetry
analysis. The program used in this package is based on \sphinxcode{\sphinxupquote{XXX}} library
which developed by XXX.
\begin{quote}\begin{description}
\item[{Parameters}] \leavevmode\begin{description}
\item[{\sphinxstylestrong{str1}}] \leavevmode{[}str{]}
\sphinxAtStartPar
The first string used to represent elements .

\item[{\sphinxstylestrong{str2}}] \leavevmode{[}str{]}
\sphinxAtStartPar
The second string used to represent elements.

\item[{\sphinxstylestrong{compare\_crystal\_cmd}}] \leavevmode{[}str{]}
\sphinxAtStartPar
The program developed to determine whether two
crystal structures are identical, default \sphinxtitleref{None}.

\item[{\sphinxstylestrong{kwargs}}] \leavevmode{[}dict arguments{]}
\sphinxAtStartPar
Other arguments used in \sphinxtitleref{compare\_crystal\_cmd}.

\end{description}

\item[{Returns}] \leavevmode\begin{description}
\item[{bool}] \leavevmode
\sphinxAtStartPar
True for yes and False for no.

\end{description}

\end{description}\end{quote}

\end{fulllineitems}

\index{corrdump() (pygace.ce.CE method)@\spxentry{corrdump()}\spxextra{pygace.ce.CE method}}

\begin{fulllineitems}
\phantomsection\label{\detokenize{pygace:pygace.ce.CE.corrdump}}\pysiglinewithargsret{\sphinxbfcode{\sphinxupquote{corrdump}}}{\emph{\DUrole{n}{cmd}}}{}
\sphinxAtStartPar
Obtain energy predicted by \sphinxcode{\sphinxupquote{corrdump}} command in \sphinxcode{\sphinxupquote{ATAT}}.
\begin{quote}\begin{description}
\item[{Parameters}] \leavevmode\begin{description}
\item[{\sphinxstylestrong{cmd}}] \leavevmode{[}str{]}
\sphinxAtStartPar
Shell command which call system \sphinxcode{\sphinxupquote{corrdump}} command of \sphinxcode{\sphinxupquote{ATAT}}.

\end{description}

\item[{Returns}] \leavevmode\begin{description}
\item[{float}] \leavevmode
\sphinxAtStartPar
Energy predicted by \sphinxcode{\sphinxupquote{corrdump}} command.

\end{description}

\end{description}\end{quote}

\end{fulllineitems}

\index{fit() (pygace.ce.CE method)@\spxentry{fit()}\spxextra{pygace.ce.CE method}}

\begin{fulllineitems}
\phantomsection\label{\detokenize{pygace:pygace.ce.CE.fit}}\pysiglinewithargsret{\sphinxbfcode{\sphinxupquote{fit}}}{\emph{\DUrole{n}{dirname}\DUrole{o}{=}\DUrole{default_value}{\textquotesingle{}./.tmp\_atat\_ce\_dir\textquotesingle{}}}}{}
\sphinxAtStartPar
Obtain all information of a directory in which a correct calculation of
CE fitting has been performed.
\begin{quote}\begin{description}
\item[{Parameters}] \leavevmode\begin{description}
\item[{\sphinxstylestrong{dirname}}] \leavevmode{[}str{]}
\sphinxAtStartPar
Which directory of the CE running.

\end{description}

\item[{Returns}] \leavevmode\begin{description}
\item[{None}] \leavevmode
\end{description}

\end{description}\end{quote}

\end{fulllineitems}

\index{get\_total\_energy() (pygace.ce.CE method)@\spxentry{get\_total\_energy()}\spxextra{pygace.ce.CE method}}

\begin{fulllineitems}
\phantomsection\label{\detokenize{pygace:pygace.ce.CE.get_total_energy}}\pysiglinewithargsret{\sphinxbfcode{\sphinxupquote{get\_total\_energy}}}{\emph{\DUrole{n}{x}}, \emph{\DUrole{n}{is\_corrdump}\DUrole{o}{=}\DUrole{default_value}{False}}, \emph{\DUrole{n}{is\_ref}\DUrole{o}{=}\DUrole{default_value}{False}}, \emph{\DUrole{n}{site\_repeat}\DUrole{o}{=}\DUrole{default_value}{\sphinxhyphen{} 1}}, \emph{\DUrole{n}{sum\_corr}\DUrole{o}{=}\DUrole{default_value}{0.0}}, \emph{\DUrole{n}{delete\_file}\DUrole{o}{=}\DUrole{default_value}{True}}}{}
\sphinxAtStartPar
Calculate absolute energy of a crystal structure like first\sphinxhyphen{}principles
calculation software package computed.
\begin{quote}\begin{description}
\item[{Parameters}] \leavevmode\begin{description}
\item[{\sphinxstylestrong{x}}] \leavevmode{[}str{]}
\sphinxAtStartPar
String for filename of lattice crystal, default \sphinxcode{\sphinxupquote{str.out}}.

\item[{\sphinxstylestrong{is\_corrdump}}] \leavevmode{[}bool{]}
\sphinxAtStartPar
Determine whether function use energy computed by \sphinxcode{\sphinxupquote{corrdump}}
command to replace absolute energy, default \sphinxtitleref{False}.

\item[{\sphinxstylestrong{is\_ref}}] \leavevmode{[}bool{]}
\sphinxAtStartPar
Determine whether function use relative energy provided by users.

\item[{\sphinxstylestrong{site\_repeat}}] \leavevmode{[}int{]}
\sphinxAtStartPar
This variable should be used seriously when a lattice structure
cannot map parent lattice.

\item[{\sphinxstylestrong{sum\_corr}}] \leavevmode{[}float{]}
\sphinxAtStartPar
If \sphinxtitleref{is\_ref} is \sphinxtitleref{True} this value will be input as energy predicted
by \sphinxcode{\sphinxupquote{corrdump}} command.

\item[{\sphinxstylestrong{delete\_file :}}] \leavevmode
\sphinxAtStartPar
Whether to delete tmp file generated by program.

\end{description}

\item[{Returns}] \leavevmode\begin{description}
\item[{float :}] \leavevmode
\sphinxAtStartPar
Total energy or corrdump energy.

\end{description}

\end{description}\end{quote}

\end{fulllineitems}

\index{make\_template() (pygace.ce.CE method)@\spxentry{make\_template()}\spxextra{pygace.ce.CE method}}

\begin{fulllineitems}
\phantomsection\label{\detokenize{pygace:pygace.ce.CE.make_template}}\pysiglinewithargsret{\sphinxbfcode{\sphinxupquote{make\_template}}}{\emph{\DUrole{n}{scale}}}{}~
\end{fulllineitems}

\index{mmaps() (pygace.ce.CE static method)@\spxentry{mmaps()}\spxextra{pygace.ce.CE static method}}

\begin{fulllineitems}
\phantomsection\label{\detokenize{pygace:pygace.ce.CE.mmaps}}\pysiglinewithargsret{\sphinxbfcode{\sphinxupquote{static }}\sphinxbfcode{\sphinxupquote{mmaps}}}{\emph{\DUrole{n}{dirname}}, \emph{\DUrole{n}{maps\_args}\DUrole{o}{=}\DUrole{default_value}{\textquotesingle{}\sphinxhyphen{}d\textquotesingle{}}}}{}
\sphinxAtStartPar
Call \sphinxcode{\sphinxupquote{MMAPS}} command in system.
\begin{quote}\begin{description}
\item[{Parameters}] \leavevmode\begin{description}
\item[{\sphinxstylestrong{dirname}}] \leavevmode{[}str{]}
\sphinxAtStartPar
Directory name of \sphinxcode{\sphinxupquote{MMAPS}} command running. Usually, it contains a
\sphinxcode{\sphinxupquote{lat.in}} file, \sphinxcode{\sphinxupquote{vasp.wrap}} or other wrap file for different
first\sphinxhyphen{}principles calculation.

\item[{\sphinxstylestrong{cal}}] \leavevmode{[}bool{]}
\sphinxAtStartPar
Determine whether to run a CE fitting. If \sphinxtitleref{False}, the function
will return when clusters information is obtained, and vice versa
CE fitting is running until users stop it.

\item[{\sphinxstylestrong{args}}] \leavevmode{[}position arg{]}
\sphinxAtStartPar
Position arguments for \sphinxcode{\sphinxupquote{MMAPS}} command.

\item[{\sphinxstylestrong{kwargs}}] \leavevmode{[}dict arg{]}
\sphinxAtStartPar
Dict arguments for \sphinxcode{\sphinxupquote{MMAPS}} command.

\end{description}

\item[{Returns}] \leavevmode\begin{description}
\item[{None}] \leavevmode
\end{description}

\end{description}\end{quote}

\end{fulllineitems}

\index{predict() (pygace.ce.CE method)@\spxentry{predict()}\spxextra{pygace.ce.CE method}}

\begin{fulllineitems}
\phantomsection\label{\detokenize{pygace:pygace.ce.CE.predict}}\pysiglinewithargsret{\sphinxbfcode{\sphinxupquote{predict}}}{\emph{\DUrole{n}{x}}}{}
\sphinxAtStartPar
Predict energy by given file name of structure.
\begin{quote}\begin{description}
\item[{Parameters}] \leavevmode\begin{description}
\item[{\sphinxstylestrong{x}}] \leavevmode{[}str{]}
\sphinxAtStartPar
‘x’ is a name of lattice structure, such as \sphinxcode{\sphinxupquote{str.out}} in \sphinxcode{\sphinxupquote{ATAT}}.

\end{description}

\item[{Returns}] \leavevmode\begin{description}
\item[{str}] \leavevmode
\sphinxAtStartPar
Energy predicted by corrdump command in \sphinxcode{\sphinxupquote{ATAT}}

\end{description}

\end{description}\end{quote}

\end{fulllineitems}


\end{fulllineitems}



\subsection{pygace.config module}
\label{\detokenize{pygace:module-pygace.config}}\label{\detokenize{pygace:pygace-config-module}}\index{module@\spxentry{module}!pygace.config@\spxentry{pygace.config}}\index{pygace.config@\spxentry{pygace.config}!module@\spxentry{module}}
\sphinxAtStartPar
The module contains config file for pygace running.


\subsection{pygace.ga module}
\label{\detokenize{pygace:module-pygace.ga}}\label{\detokenize{pygace:pygace-ga-module}}\index{module@\spxentry{module}!pygace.ga@\spxentry{pygace.ga}}\index{pygace.ga@\spxentry{pygace.ga}!module@\spxentry{module}}
\sphinxAtStartPar
The module contains genetic algorithms and relevant operator used in GA, e.g.,
crossover operator, mutation operator.
\index{Cromosome (class in pygace.ga)@\spxentry{Cromosome}\spxextra{class in pygace.ga}}

\begin{fulllineitems}
\phantomsection\label{\detokenize{pygace:pygace.ga.Cromosome}}\pysiglinewithargsret{\sphinxbfcode{\sphinxupquote{class }}\sphinxcode{\sphinxupquote{pygace.ga.}}\sphinxbfcode{\sphinxupquote{Cromosome}}}{\emph{\DUrole{n}{gene\_length}\DUrole{o}{=}\DUrole{default_value}{64}}, \emph{\DUrole{n}{fitness}\DUrole{o}{=}\DUrole{default_value}{0.0}}}{}
\sphinxAtStartPar
Bases: \sphinxcode{\sphinxupquote{object}}

\sphinxAtStartPar
A wrapper class for individual in GA.
\begin{quote}\begin{description}
\item[{Parameters}] \leavevmode\begin{description}
\item[{\sphinxstylestrong{gene\_length}}] \leavevmode{[}int{]}
\sphinxAtStartPar
The length of gene.

\item[{\sphinxstylestrong{fitness}}] \leavevmode{[}float{]}
\sphinxAtStartPar
The fitness value of gene

\end{description}

\end{description}\end{quote}
\index{generate\_cromosome() (pygace.ga.Cromosome method)@\spxentry{generate\_cromosome()}\spxextra{pygace.ga.Cromosome method}}

\begin{fulllineitems}
\phantomsection\label{\detokenize{pygace:pygace.ga.Cromosome.generate_cromosome}}\pysiglinewithargsret{\sphinxbfcode{\sphinxupquote{generate\_cromosome}}}{}{}
\sphinxAtStartPar
Greate a random cromosome
\begin{quote}\begin{description}
\item[{Returns}] \leavevmode\begin{description}
\item[{None}] \leavevmode
\end{description}

\end{description}\end{quote}

\end{fulllineitems}

\index{get\_gene() (pygace.ga.Cromosome method)@\spxentry{get\_gene()}\spxextra{pygace.ga.Cromosome method}}

\begin{fulllineitems}
\phantomsection\label{\detokenize{pygace:pygace.ga.Cromosome.get_gene}}\pysiglinewithargsret{\sphinxbfcode{\sphinxupquote{get\_gene}}}{\emph{\DUrole{n}{index}}}{}
\sphinxAtStartPar
Obtain gene in the position of \sphinxtitleref{index}.
\begin{quote}\begin{description}
\item[{Parameters}] \leavevmode\begin{description}
\item[{\sphinxstylestrong{index}}] \leavevmode{[}int{]}
\sphinxAtStartPar
The index of position.

\end{description}

\item[{Returns}] \leavevmode\begin{description}
\item[{int}] \leavevmode
\sphinxAtStartPar
The gene

\end{description}

\end{description}\end{quote}

\end{fulllineitems}

\index{set\_gene() (pygace.ga.Cromosome method)@\spxentry{set\_gene()}\spxextra{pygace.ga.Cromosome method}}

\begin{fulllineitems}
\phantomsection\label{\detokenize{pygace:pygace.ga.Cromosome.set_gene}}\pysiglinewithargsret{\sphinxbfcode{\sphinxupquote{set\_gene}}}{\emph{\DUrole{n}{index}}, \emph{\DUrole{n}{value}}}{}
\sphinxAtStartPar
Set gene in position of \sphinxtitleref{index}.
\begin{quote}\begin{description}
\item[{Parameters}] \leavevmode\begin{description}
\item[{\sphinxstylestrong{index}}] \leavevmode{[}int{]}
\sphinxAtStartPar
The index of position.

\item[{\sphinxstylestrong{value}}] \leavevmode{[}float{]}
\sphinxAtStartPar
The value of gene.

\end{description}

\item[{Returns}] \leavevmode\begin{description}
\item[{None}] \leavevmode
\end{description}

\end{description}\end{quote}

\end{fulllineitems}

\index{size() (pygace.ga.Cromosome method)@\spxentry{size()}\spxextra{pygace.ga.Cromosome method}}

\begin{fulllineitems}
\phantomsection\label{\detokenize{pygace:pygace.ga.Cromosome.size}}\pysiglinewithargsret{\sphinxbfcode{\sphinxupquote{size}}}{}{}
\sphinxAtStartPar
Return the length of cromosome.
\begin{quote}\begin{description}
\item[{Returns}] \leavevmode\begin{description}
\item[{int}] \leavevmode
\sphinxAtStartPar
The length of cromosome

\end{description}

\end{description}\end{quote}

\end{fulllineitems}

\index{valid\_type (pygace.ga.Cromosome attribute)@\spxentry{valid\_type}\spxextra{pygace.ga.Cromosome attribute}}

\begin{fulllineitems}
\phantomsection\label{\detokenize{pygace:pygace.ga.Cromosome.valid_type}}\pysigline{\sphinxbfcode{\sphinxupquote{valid\_type}}\sphinxbfcode{\sphinxupquote{ = {[}\textquotesingle{}Vac\textquotesingle{}, \textquotesingle{}Replace\textquotesingle{}{]}}}}
\end{fulllineitems}


\end{fulllineitems}

\index{Individual (class in pygace.ga)@\spxentry{Individual}\spxextra{class in pygace.ga}}

\begin{fulllineitems}
\phantomsection\label{\detokenize{pygace:pygace.ga.Individual}}\pysigline{\sphinxbfcode{\sphinxupquote{class }}\sphinxcode{\sphinxupquote{pygace.ga.}}\sphinxbfcode{\sphinxupquote{Individual}}}
\sphinxAtStartPar
Bases: \sphinxcode{\sphinxupquote{object}}

\sphinxAtStartPar
Individual object contain several Cromosome objects.

\end{fulllineitems}

\index{cycle\_crossover() (in module pygace.ga)@\spxentry{cycle\_crossover()}\spxextra{in module pygace.ga}}

\begin{fulllineitems}
\phantomsection\label{\detokenize{pygace:pygace.ga.cycle_crossover}}\pysiglinewithargsret{\sphinxcode{\sphinxupquote{pygace.ga.}}\sphinxbfcode{\sphinxupquote{cycle\_crossover}}}{\emph{\DUrole{n}{ind1}}, \emph{\DUrole{n}{ind2}}, \emph{\DUrole{n}{cross\_number}}}{}
\sphinxAtStartPar
Cycle crossover (CX).

\sphinxAtStartPar
CX algorithm:
\begin{itemize}
\item {} 
\sphinxAtStartPar
parent1: \sphinxcode{\sphinxupquote{{[}|1 |2 3 |4 |5 6 7 8 |9{]}}}

\item {} 
\sphinxAtStartPar
parent2: \sphinxcode{\sphinxupquote{{[}|5 |4 6 |9 |2 3 7 8 |1{]}}}

\item {} 
\sphinxAtStartPar
child1: \sphinxcode{\sphinxupquote{{[}|1 |2 6 |4 |5 3 7 8 |9{]}}}

\item {} 
\sphinxAtStartPar
child2: \sphinxcode{\sphinxupquote{{[}|5 |4 3 |9 |2 6 7 8 |1{]}}}

\end{itemize}
\begin{quote}\begin{description}
\item[{Parameters}] \leavevmode\begin{description}
\item[{\sphinxstylestrong{ind1}}] \leavevmode{[}iteration object{]}
\sphinxAtStartPar
The first cromosome participating in the crossover.

\item[{\sphinxstylestrong{ind2}}] \leavevmode{[}iteration object{]}
\sphinxAtStartPar
The second cromosome participating in the crossover.

\item[{\sphinxstylestrong{cross\_number}}] \leavevmode{[}int{]}
\sphinxAtStartPar
The number of crossover is not used in this algorithm.

\end{description}

\item[{Returns}] \leavevmode\begin{description}
\item[{tuple}] \leavevmode
\sphinxAtStartPar
A tuple of two cromosomes

\end{description}

\end{description}\end{quote}
\subsubsection*{References}

\sphinxAtStartPar
More details can bee seen Ref. %
\begin{footnote}[1]\sphinxAtStartFootnote
Oliver, I.; Smith, D.; Holland, J. A study of permutation crossover
operators on the traveling salesman problem. Proceedings of the 2nd
International Conference on Genetic Algorithms, J.J. Grefenstette (ed.).
Hillsdale, New Jersey, 1987; pp 224\sphinxhyphen{}230.
%
\end{footnote}.

\end{fulllineitems}

\index{gaceCrossover() (in module pygace.ga)@\spxentry{gaceCrossover()}\spxextra{in module pygace.ga}}

\begin{fulllineitems}
\phantomsection\label{\detokenize{pygace:pygace.ga.gaceCrossover}}\pysiglinewithargsret{\sphinxcode{\sphinxupquote{pygace.ga.}}\sphinxbfcode{\sphinxupquote{gaceCrossover}}}{\emph{\DUrole{n}{indiv1}}, \emph{\DUrole{n}{indiv2}}, \emph{\DUrole{n}{crossover\_type}\DUrole{o}{=}\DUrole{default_value}{1}}, \emph{\DUrole{n}{cross\_num}\DUrole{o}{=}\DUrole{default_value}{8}}}{}
\sphinxAtStartPar
Executes a crossover specified by crossover type \sphinxtitleref{crossover\_type\textasciigrave{}and the
number of crossover \textasciigrave{}cross\_num} on the input \sphinxtitleref{sequence} individuals.
The two individuals are modified in place and both keep their original
length.
\begin{quote}\begin{description}
\item[{Parameters}] \leavevmode\begin{description}
\item[{\sphinxstylestrong{indiv1}}] \leavevmode{[}Cromosome object{]}
\sphinxAtStartPar
The first individual participating in the crossover.

\item[{\sphinxstylestrong{indiv2}}] \leavevmode{[}Cromosome object{]}
\sphinxAtStartPar
The second individual participating in the crossover.

\item[{\sphinxstylestrong{crossover\_type}}] \leavevmode{[}int{]}
\sphinxAtStartPar
The type of crossover method:
\begin{itemize}
\item {} 
\sphinxAtStartPar
\sphinxcode{\sphinxupquote{1}}: Partially\sphinxhyphen{}mapped crossover (PMX)

\item {} 
\sphinxAtStartPar
\sphinxcode{\sphinxupquote{2}}: Order Crossover (OX1)

\item {} 
\sphinxAtStartPar
\sphinxcode{\sphinxupquote{3}}: Position based crossover (POS)

\item {} 
\sphinxAtStartPar
\sphinxcode{\sphinxupquote{4}}: Order based Crossover (OX2)

\item {} 
\sphinxAtStartPar
\sphinxcode{\sphinxupquote{5}}: Cycle crossover (CX)

\item {} 
\sphinxAtStartPar
\sphinxcode{\sphinxupquote{6}}: Subtour exchange crossover (SXX)

\end{itemize}

\item[{\sphinxstylestrong{cross\_num}}] \leavevmode{[}int{]}
\sphinxAtStartPar
The number of crossover which determine the number exchange in each
crossover operation.

\end{description}

\item[{Returns}] \leavevmode\begin{description}
\item[{tuple}] \leavevmode
\sphinxAtStartPar
A tuple of two individuals.

\end{description}

\end{description}\end{quote}

\end{fulllineitems}

\index{gaceGA() (in module pygace.ga)@\spxentry{gaceGA()}\spxextra{in module pygace.ga}}

\begin{fulllineitems}
\phantomsection\label{\detokenize{pygace:pygace.ga.gaceGA}}\pysiglinewithargsret{\sphinxcode{\sphinxupquote{pygace.ga.}}\sphinxbfcode{\sphinxupquote{gaceGA}}}{\emph{\DUrole{n}{population}}, \emph{\DUrole{n}{toolbox}}, \emph{\DUrole{n}{cxpb}}, \emph{\DUrole{n}{ngen}}, \emph{\DUrole{n}{stats}\DUrole{o}{=}\DUrole{default_value}{None}}, \emph{\DUrole{n}{halloffame}\DUrole{o}{=}\DUrole{default_value}{None}}, \emph{\DUrole{n}{verbose}\DUrole{o}{=}\DUrole{default_value}{True}}, \emph{\DUrole{n}{checkpoint}\DUrole{o}{=}\DUrole{default_value}{None}}, \emph{\DUrole{n}{freq}\DUrole{o}{=}\DUrole{default_value}{10}}}{}
\sphinxAtStartPar
Genetic algorithm (GA) used in \sphinxcode{\sphinxupquote{pygace}}. Users can define their algorithms
based on \sphinxcode{\sphinxupquote{DEAP}} package or other GA framework.
\begin{quote}\begin{description}
\item[{Parameters}] \leavevmode\begin{description}
\item[{\sphinxstylestrong{population}}] \leavevmode{[}list{]}
\sphinxAtStartPar
A list represent population consists of all individual.

\item[{\sphinxstylestrong{toolbox}}] \leavevmode{[}toolbox object{]}
\sphinxAtStartPar
\sphinxcode{\sphinxupquote{DEAP}} toolbox object

\item[{\sphinxstylestrong{cxpb}}] \leavevmode{[}float{]}
\sphinxAtStartPar
The probability of crossover happens.

\item[{\sphinxstylestrong{ngen}}] \leavevmode{[}int{]}
\sphinxAtStartPar
The number of generations.

\item[{\sphinxstylestrong{stats :}}] \leavevmode
\sphinxAtStartPar
The random state of simulation.

\item[{\sphinxstylestrong{halloffame}}] \leavevmode{[}list{]}
\sphinxAtStartPar
Restored individual.

\item[{\sphinxstylestrong{verbose}}] \leavevmode{[}bool{]}
\sphinxAtStartPar
Whether to show more message of running.

\item[{\sphinxstylestrong{checkpoint}}] \leavevmode{[}str{]}
\sphinxAtStartPar
The filename of checkpoint file.

\item[{\sphinxstylestrong{freq}}] \leavevmode{[}int{]}
\sphinxAtStartPar
The number that determine how many step to write a checkpoint.

\end{description}

\item[{Returns}] \leavevmode\begin{description}
\item[{tuple}] \leavevmode
\sphinxAtStartPar
A tuple of population and log file.

\end{description}

\end{description}\end{quote}

\end{fulllineitems}

\index{gaceMutShuffleIndexes() (in module pygace.ga)@\spxentry{gaceMutShuffleIndexes()}\spxextra{in module pygace.ga}}

\begin{fulllineitems}
\phantomsection\label{\detokenize{pygace:pygace.ga.gaceMutShuffleIndexes}}\pysiglinewithargsret{\sphinxcode{\sphinxupquote{pygace.ga.}}\sphinxbfcode{\sphinxupquote{gaceMutShuffleIndexes}}}{\emph{\DUrole{n}{individual}}, \emph{\DUrole{n}{indpb}}}{}
\sphinxAtStartPar
Shuffle the attributes of the input individual and return the mutant.
The \sphinxtitleref{individual} is expected to be a \sphinxtitleref{sequence}. The \sphinxtitleref{indpb} argument
is the probability of each attribute to be moved. Usually this mutation is
applied on vector of indices.
\begin{quote}\begin{description}
\item[{Parameters}] \leavevmode\begin{description}
\item[{\sphinxstylestrong{individual}}] \leavevmode{[}Individual object{]}
\sphinxAtStartPar
Individual to be mutated.

\item[{\sphinxstylestrong{indpb}}] \leavevmode{[}float{]}
\sphinxAtStartPar
Independent probability for each attribute to be exchanged to another
position.

\end{description}

\item[{Returns}] \leavevmode\begin{description}
\item[{tuple}] \leavevmode
\sphinxAtStartPar
A tuple of one individual.

\end{description}

\end{description}\end{quote}

\end{fulllineitems}

\index{gaceVarAnd() (in module pygace.ga)@\spxentry{gaceVarAnd()}\spxextra{in module pygace.ga}}

\begin{fulllineitems}
\phantomsection\label{\detokenize{pygace:pygace.ga.gaceVarAnd}}\pysiglinewithargsret{\sphinxcode{\sphinxupquote{pygace.ga.}}\sphinxbfcode{\sphinxupquote{gaceVarAnd}}}{\emph{\DUrole{n}{population}}, \emph{\DUrole{n}{toolbox}}, \emph{\DUrole{n}{cxpb}}}{}
\sphinxAtStartPar
Execute crossover and mutation operation in genetic algorithm running
process.
\begin{quote}\begin{description}
\item[{Parameters}] \leavevmode\begin{description}
\item[{\sphinxstylestrong{population}}] \leavevmode{[}list{]}
\sphinxAtStartPar
The population of all individual.

\item[{\sphinxstylestrong{toolbox}}] \leavevmode{[}Toolbox object{]}
\sphinxAtStartPar
The \sphinxtitleref{Toolbox} object defined in \sphinxcode{\sphinxupquote{DEAP}}.

\item[{\sphinxstylestrong{cxpb}}] \leavevmode{[}float{]}
\sphinxAtStartPar
The probability or crossover.

\end{description}

\item[{Returns}] \leavevmode\begin{description}
\item[{list}] \leavevmode
\sphinxAtStartPar
The new generation.

\end{description}

\end{description}\end{quote}

\end{fulllineitems}

\index{order\_based\_crossover() (in module pygace.ga)@\spxentry{order\_based\_crossover()}\spxextra{in module pygace.ga}}

\begin{fulllineitems}
\phantomsection\label{\detokenize{pygace:pygace.ga.order_based_crossover}}\pysiglinewithargsret{\sphinxcode{\sphinxupquote{pygace.ga.}}\sphinxbfcode{\sphinxupquote{order\_based\_crossover}}}{\emph{\DUrole{n}{ind1}}, \emph{\DUrole{n}{ind2}}, \emph{\DUrole{n}{cross\_number}}}{}
\sphinxAtStartPar
Order based Crossover (OX2) operator selects at random several positions
in a parent tour, and the order of the cities in the selected positions
of this parent is imposed on the other parent.

\sphinxAtStartPar
OX2 algorithm example:
\begin{itemize}
\item {} 
\sphinxAtStartPar
parent1 \sphinxcode{\sphinxupquote{{[}1 |2 3 4 |5 |6 7 8 |9{]}}}

\item {} 
\sphinxAtStartPar
parent2 \sphinxcode{\sphinxupquote{{[}5 |4 6 3 |1 |9 2 7 |8{]}}}

\item {} 
\sphinxAtStartPar
child1 \sphinxcode{\sphinxupquote{{[}2 |4 5 |3 |1 6 9 |7 |8{]}}}

\item {} 
\sphinxAtStartPar
child2 \sphinxcode{\sphinxupquote{{[}4 |2 |3 1 |5 |6 |7 9 8{]}}}

\end{itemize}
\begin{quote}\begin{description}
\item[{Parameters}] \leavevmode\begin{description}
\item[{\sphinxstylestrong{ind1}}] \leavevmode{[}iteration object{]}
\sphinxAtStartPar
The first cromosome participating in the crossover.

\item[{\sphinxstylestrong{ind2}}] \leavevmode{[}iteration object{]}
\sphinxAtStartPar
The second cromosome participating in the crossover.

\item[{\sphinxstylestrong{cross\_number}}] \leavevmode{[}int{]}
\sphinxAtStartPar
The number of crossover which determine the number exchange in each
crossover operation.

\end{description}

\item[{Returns}] \leavevmode\begin{description}
\item[{tuple}] \leavevmode
\sphinxAtStartPar
A tuple of two cromosomes

\end{description}

\end{description}\end{quote}
\subsubsection*{References}

\sphinxAtStartPar
More details about OX2 can be seen Ref. %
\begin{footnote}[2]\sphinxAtStartFootnote
Syswerda, G. Handbook of Genetic Algorithms 1991, 332\sphinxhyphen{}349.
%
\end{footnote}.

\end{fulllineitems}

\index{order\_crossover() (in module pygace.ga)@\spxentry{order\_crossover()}\spxextra{in module pygace.ga}}

\begin{fulllineitems}
\phantomsection\label{\detokenize{pygace:pygace.ga.order_crossover}}\pysiglinewithargsret{\sphinxcode{\sphinxupquote{pygace.ga.}}\sphinxbfcode{\sphinxupquote{order\_crossover}}}{\emph{\DUrole{n}{ind1}}, \emph{\DUrole{n}{ind2}}, \emph{\DUrole{n}{cross\_number}}}{}
\sphinxAtStartPar
Order crossover (OX1) operator was proposed by Davis (1985). The OX1 exploits
a property of the path representation, that the order of cities (not their
positions) are important. It constructs an offspring by choosing a subtour
of one parent and preserving the relative order of cities of the other
parent.

\sphinxAtStartPar
order crossover algorithm example:
\begin{itemize}
\item {} 
\sphinxAtStartPar
parent1: \sphinxcode{\sphinxupquote{{[}1 2 |3 4 5 6| 7 8 9{]}}}

\item {} 
\sphinxAtStartPar
parent2: \sphinxcode{\sphinxupquote{{[}5 7 |4 9 1 3| 6 2 8{]}}}

\item {} 
\sphinxAtStartPar
child1: \sphinxcode{\sphinxupquote{{[}7 9 |3 4 5 6| 1 2 8{]}}}

\item {} 
\sphinxAtStartPar
child2: \sphinxcode{\sphinxupquote{{[}2 5 |4 9 1 3| 6 7 8{]}}}

\end{itemize}
\begin{quote}\begin{description}
\item[{Parameters}] \leavevmode\begin{description}
\item[{\sphinxstylestrong{ind1}}] \leavevmode{[}iteration object{]}
\sphinxAtStartPar
The first cromosome participating in the crossover.

\item[{\sphinxstylestrong{ind2}}] \leavevmode{[}iteration object{]}
\sphinxAtStartPar
The second cromosome participating in the crossover.

\item[{\sphinxstylestrong{cross\_number}}] \leavevmode{[}int{]}
\sphinxAtStartPar
The number of crossover which determine the number exchange in each
crossover operation.

\end{description}

\item[{Returns}] \leavevmode\begin{description}
\item[{tuple}] \leavevmode
\sphinxAtStartPar
A tuple of two cromosomes

\end{description}

\end{description}\end{quote}
\subsubsection*{References}

\sphinxAtStartPar
More details about OX1 can be seen Ref. %
\begin{footnote}[3]\sphinxAtStartFootnote
Davis, L. Applying Adaptive Algorithms to Epistatic Domains. Proceedings
of the 9th International Joint Conference on Artificial Intelligence \sphinxhyphen{}
Volume 1. San Francisco, CA, USA, 1985; pp 162\sphinxhyphen{}164.
%
\end{footnote}.

\end{fulllineitems}

\index{partial\_mapped\_crossover() (in module pygace.ga)@\spxentry{partial\_mapped\_crossover()}\spxextra{in module pygace.ga}}

\begin{fulllineitems}
\phantomsection\label{\detokenize{pygace:pygace.ga.partial_mapped_crossover}}\pysiglinewithargsret{\sphinxcode{\sphinxupquote{pygace.ga.}}\sphinxbfcode{\sphinxupquote{partial\_mapped\_crossover}}}{\emph{\DUrole{n}{ind1}}, \emph{\DUrole{n}{ind2}}, \emph{\DUrole{n}{cross\_number}}}{}
\sphinxAtStartPar
Partially\sphinxhyphen{}mapped crossover (PMX) operator was suggested by Goldberg and
Lingle (1985). It passes on ordering and value information from the
parent tours to the offspring tours. A portion of one parents’s string
is mapped onto a portion of the other parent’s string and the remaining
informatin is exchanged..

\sphinxAtStartPar
The algorithm example:
\begin{itemize}
\item {} 
\sphinxAtStartPar
parent1: \sphinxcode{\sphinxupquote{{[}1,2,|3,4,5,6|,7,8,9{]}}}

\item {} 
\sphinxAtStartPar
parent2: \sphinxcode{\sphinxupquote{{[}5,4,|6,9,2,1|,7,8,3{]}}}

\item {} 
\sphinxAtStartPar
child1: \sphinxcode{\sphinxupquote{{[}3,5,|6,9,2,1|,7,8,4{]}}}

\item {} 
\sphinxAtStartPar
child2: \sphinxcode{\sphinxupquote{{[}2,9,|3,4,5,6|,7,8,1{]}}}

\end{itemize}
\begin{quote}\begin{description}
\item[{Parameters}] \leavevmode\begin{description}
\item[{\sphinxstylestrong{ind1}}] \leavevmode{[}iteration object{]}
\sphinxAtStartPar
The first individual participating in the crossover.

\item[{\sphinxstylestrong{ind2}}] \leavevmode{[}iteration object{]}
\sphinxAtStartPar
The second individual participating in the crossover.

\item[{\sphinxstylestrong{cross\_number}}] \leavevmode{[}int{]}
\sphinxAtStartPar
The number of crossover which determine the number exchange in each
crossover operation.

\end{description}

\item[{Returns}] \leavevmode\begin{description}
\item[{tuple}] \leavevmode
\sphinxAtStartPar
A tuple of two individuals

\end{description}

\end{description}\end{quote}
\subsubsection*{References}

\sphinxAtStartPar
More details about PMX can be seen in Ref. %
\begin{footnote}[4]\sphinxAtStartFootnote
Goldberg, D.; Lingle, R.; Alleles, L. the Travelling Salesman Problem.
Proceedings of the 1st International Conference on Genetic Algorithms and
their Applications, J.J. Grefenstette (ed.). Carneige\sphinxhyphen{}Mellon University,
Pittsburgh, 1985.
%
\end{footnote}.

\end{fulllineitems}

\index{position\_based\_crossover() (in module pygace.ga)@\spxentry{position\_based\_crossover()}\spxextra{in module pygace.ga}}

\begin{fulllineitems}
\phantomsection\label{\detokenize{pygace:pygace.ga.position_based_crossover}}\pysiglinewithargsret{\sphinxcode{\sphinxupquote{pygace.ga.}}\sphinxbfcode{\sphinxupquote{position\_based\_crossover}}}{\emph{\DUrole{n}{ind1}}, \emph{\DUrole{n}{ind2}}, \emph{\DUrole{n}{cross\_number}}}{}
\sphinxAtStartPar
Position\sphinxhyphen{}based crossover (PBC)

\sphinxAtStartPar
PBC algorithm:
\begin{itemize}
\item {} 
\sphinxAtStartPar
parent1: \sphinxcode{\sphinxupquote{{[}1 |2 3 4 |5 |6 7 8 |9{]}}}

\item {} 
\sphinxAtStartPar
parent2: \sphinxcode{\sphinxupquote{{[}5 |4 6 4 |1 |9 2 7 |8{]}}}

\item {} 
\sphinxAtStartPar
child1: \sphinxcode{\sphinxupquote{{[}4 |2 3 1 |5 |6 7 8 |9{]}}}

\item {} 
\sphinxAtStartPar
child2: \sphinxcode{\sphinxupquote{{[}2 |4 3 5 |1 |9 6 7 |8{]}}}

\end{itemize}
\begin{quote}\begin{description}
\item[{Parameters}] \leavevmode\begin{description}
\item[{\sphinxstylestrong{ind1}}] \leavevmode{[}iteration object{]}
\sphinxAtStartPar
The first cromosome participating in the crossover.

\item[{\sphinxstylestrong{ind2}}] \leavevmode{[}iteration object{]}
\sphinxAtStartPar
The second cromosome participating in the crossover.

\item[{\sphinxstylestrong{cross\_number}}] \leavevmode{[}int{]}
\sphinxAtStartPar
The number of crossover which determine the number exchange in each
crossover operation.

\end{description}

\item[{Returns}] \leavevmode\begin{description}
\item[{tuple}] \leavevmode
\sphinxAtStartPar
A tuple of two cromosomes

\end{description}

\end{description}\end{quote}
\subsubsection*{References}

\sphinxAtStartPar
More details can be seen Ref. %
\begin{footnote}[5]\sphinxAtStartFootnote
Syswerda, G. Handbook of Genetic Algorithms 1991, 332\sphinxhyphen{}349.
%
\end{footnote}.

\end{fulllineitems}

\index{subtour\_exchange\_crossover() (in module pygace.ga)@\spxentry{subtour\_exchange\_crossover()}\spxextra{in module pygace.ga}}

\begin{fulllineitems}
\phantomsection\label{\detokenize{pygace:pygace.ga.subtour_exchange_crossover}}\pysiglinewithargsret{\sphinxcode{\sphinxupquote{pygace.ga.}}\sphinxbfcode{\sphinxupquote{subtour\_exchange\_crossover}}}{\emph{\DUrole{n}{ind1}}, \emph{\DUrole{n}{ind2}}, \emph{\DUrole{n}{cross\_number}}}{}
\sphinxAtStartPar
Subtour exchange crossover (SXX).

\sphinxAtStartPar
SXX algorithm:
\begin{itemize}
\item {} 
\sphinxAtStartPar
parent1: \sphinxcode{\sphinxupquote{{[}1 2 3 |4 5 6 7| 8 9{]}}}

\item {} 
\sphinxAtStartPar
parent2: \sphinxcode{\sphinxupquote{{[}3 |4 9 |7 8 |5 2 1 |6{]}}}

\item {} 
\sphinxAtStartPar
child1: \sphinxcode{\sphinxupquote{{[}1 2 3 |4 7 5 6| 8 9{]}}}

\item {} 
\sphinxAtStartPar
child2: \sphinxcode{\sphinxupquote{{[}3 |4 9 |5 8 |6 2 1 |7{]}}}

\end{itemize}
\begin{quote}\begin{description}
\item[{Parameters}] \leavevmode\begin{description}
\item[{\sphinxstylestrong{ind1}}] \leavevmode{[}iteration object{]}
\sphinxAtStartPar
The first cromosome participating in the crossover.

\item[{\sphinxstylestrong{ind2}}] \leavevmode{[}iteration object{]}
\sphinxAtStartPar
The second cromosome participating in the crossover.

\item[{\sphinxstylestrong{cross\_number}}] \leavevmode{[}int{]}
\sphinxAtStartPar
The number of crossover is not used in this algorithm.

\end{description}

\item[{Returns}] \leavevmode\begin{description}
\item[{tuple}] \leavevmode
\sphinxAtStartPar
A tuple of two cromosomes

\end{description}

\end{description}\end{quote}
\subsubsection*{References}

\sphinxAtStartPar
More details about SXX can be seen Ref. %
\begin{footnote}[6]\sphinxAtStartFootnote
Yamamura, M.; Ono, T.; Kobayashi, S. Japanese Society for Artificial
Intelligence.
%
\end{footnote}.

\end{fulllineitems}

\index{transfer\_from() (in module pygace.ga)@\spxentry{transfer\_from()}\spxextra{in module pygace.ga}}

\begin{fulllineitems}
\phantomsection\label{\detokenize{pygace:pygace.ga.transfer_from}}\pysiglinewithargsret{\sphinxcode{\sphinxupquote{pygace.ga.}}\sphinxbfcode{\sphinxupquote{transfer\_from}}}{\emph{\DUrole{n}{ind}}}{}~
\end{fulllineitems}



\subsection{pygace.gace module}
\label{\detokenize{pygace:module-pygace.gace}}\label{\detokenize{pygace:pygace-gace-module}}\index{module@\spxentry{module}!pygace.gace@\spxentry{pygace.gace}}\index{pygace.gace@\spxentry{pygace.gace}!module@\spxentry{module}}
\sphinxAtStartPar
GACE framework module

\sphinxAtStartPar
This module provide abstract GACE object used to be implemented by users in
their application, and it defines several interface which are called in
the application.
\index{AbstractApp (class in pygace.gace)@\spxentry{AbstractApp}\spxextra{class in pygace.gace}}

\begin{fulllineitems}
\phantomsection\label{\detokenize{pygace:pygace.gace.AbstractApp}}\pysiglinewithargsret{\sphinxbfcode{\sphinxupquote{class }}\sphinxcode{\sphinxupquote{pygace.gace.}}\sphinxbfcode{\sphinxupquote{AbstractApp}}}{\emph{\DUrole{n}{ce\_site}\DUrole{o}{=}\DUrole{default_value}{None}}, \emph{\DUrole{n}{ce\_dirname}\DUrole{o}{=}\DUrole{default_value}{\textquotesingle{}./data/iter1\textquotesingle{}}}, \emph{\DUrole{n}{params\_config\_dict}\DUrole{o}{=}\DUrole{default_value}{None}}}{}
\sphinxAtStartPar
Bases: \sphinxcode{\sphinxupquote{object}}

\sphinxAtStartPar
Abstract application object for \sphinxcode{\sphinxupquote{GACE}} framework.

\sphinxAtStartPar
AbstractApp initial process needs input parameters of CE simulation
and informatin of output directory. Also, the parameters for DFT
calculation should also be included in \sphinxcode{\sphinxupquote{params\_config\_dict}} for
user custom.
\begin{quote}\begin{description}
\item[{Parameters}] \leavevmode\begin{description}
\item[{\sphinxstylestrong{ce\_site}}] \leavevmode{[}int{]}
\sphinxAtStartPar
The concept of site used in \sphinxcode{\sphinxupquote{MAPS}} or \sphinxcode{\sphinxupquote{MMAPS}} in \sphinxcode{\sphinxupquote{ATAT}}
program.

\item[{\sphinxstylestrong{ce\_dirname}}] \leavevmode{[}:obj: str, optional{]}\begin{description}
\item[{A path of directory which contains information after running}] \leavevmode
\sphinxAtStartPar
\sphinxcode{\sphinxupquote{MMAPS}} or \sphinxcode{\sphinxupquote{MAPS}}.

\end{description}

\item[{\sphinxstylestrong{params\_config\_dict}}] \leavevmode{[}dict, optional{]}
\sphinxAtStartPar
A dict used to update DEFAULT\_DICT of AbstractApp object.

\end{description}

\item[{Attributes}] \leavevmode\begin{description}
\item[{\sphinxstylestrong{ce}}] \leavevmode{[}CE{]}
\sphinxAtStartPar
CE object defined in \sphinxtitleref{ce.CE}.

\item[{\sphinxstylestrong{params\_config\_dict}}] \leavevmode{[}dict{]}
\sphinxAtStartPar
Parameters used in to construct CE object and other parameters used
in GACE simulation. User can custom this dict for their own needs.

\item[{\sphinxstylestrong{energy\_database\_fname}}] \leavevmode{[}str{]}
\sphinxAtStartPar
Filename of file that restore energies for different configurations
to accelerate energy\sphinxhyphen{}calculation of a energy\sphinxhyphen{}unknown configuration.

\item[{\sphinxstylestrong{toolbox}}] \leavevmode{[}ToolBox{]}
\sphinxAtStartPar
The ToolBox object defined in \sphinxtitleref{deap.tools}.

\item[{\sphinxstylestrong{DEFAULT\_SETUP}}] \leavevmode{[}dict{]}
\sphinxAtStartPar
Class attribute which restores revelant parameters used in GA and
CE simulation process. See also \sphinxtitleref{params\_config\_dict} for custom.

\item[{\sphinxstylestrong{ENERGY\_DICAT}}] \leavevmode{[}dict{]}
\sphinxAtStartPar
A dict in which key is list of num representing a configuration and
value is the fitness value of the configuration, e.g., total energy or
formation energy of point defects.

\item[{\sphinxstylestrong{PREVIOUS\_COUNT}}] \leavevmode{[}int{]}
\sphinxAtStartPar
A parameter used to restore the execution step of previous simulation
in order to run from previous stop step.

\item[{\sphinxstylestrong{TYPES\_ENERGY\_DICT}}] \leavevmode{[}dict{]}
\sphinxAtStartPar
A dict restores different elements and their responding number index
in order to convert a element to a number in GA simulation, e.g.,
\{‘Hf’:1, ‘O’:2, ‘Vac’:3\}.

\item[{\sphinxstylestrong{TEMPLATE\_FILE\_STR}}] \leavevmode{[}str{]}
\sphinxAtStartPar
A string to restore the template of \sphinxtitleref{lat.in} which is a main
input file in \sphinxcode{\sphinxupquote{ATAT}}.

\end{description}

\end{description}\end{quote}
\index{DEFAULT\_SETUP (pygace.gace.AbstractApp attribute)@\spxentry{DEFAULT\_SETUP}\spxextra{pygace.gace.AbstractApp attribute}}

\begin{fulllineitems}
\phantomsection\label{\detokenize{pygace:pygace.gace.AbstractApp.DEFAULT_SETUP}}\pysigline{\sphinxbfcode{\sphinxupquote{DEFAULT\_SETUP}}\sphinxbfcode{\sphinxupquote{ = \{\textquotesingle{}DFT\_CAL\_DIR\textquotesingle{}: \textquotesingle{}./dft\_dirs\textquotesingle{}, \textquotesingle{}NB\_DEFECT\textquotesingle{}: None, \textquotesingle{}PICKLE\_DIR\textquotesingle{}: \textquotesingle{}/Users/yxcheng/PycharmProjects/pygace/doc/pickle\_bakup\textquotesingle{}, \textquotesingle{}TEMPLATE\_FILE\textquotesingle{}: \textquotesingle{}./data/lat\_in.template\textquotesingle{}, \textquotesingle{}TEST\_RES\_DIR\textquotesingle{}: \textquotesingle{}/Users/yxcheng/PycharmProjects/pygace/doc/res\_dir\textquotesingle{}, \textquotesingle{}TMP\_DIR\textquotesingle{}: \textquotesingle{}/Users/yxcheng/PycharmProjects/pygace/doc/tmp\_dir\textquotesingle{}\}}}}
\end{fulllineitems}

\index{evalEnergy() (pygace.gace.AbstractApp method)@\spxentry{evalEnergy()}\spxextra{pygace.gace.AbstractApp method}}

\begin{fulllineitems}
\phantomsection\label{\detokenize{pygace:pygace.gace.AbstractApp.evalEnergy}}\pysiglinewithargsret{\sphinxbfcode{\sphinxupquote{evalEnergy}}}{\emph{\DUrole{n}{individual}}}{}~
\end{fulllineitems}

\index{get\_ce() (pygace.gace.AbstractApp method)@\spxentry{get\_ce()}\spxextra{pygace.gace.AbstractApp method}}

\begin{fulllineitems}
\phantomsection\label{\detokenize{pygace:pygace.gace.AbstractApp.get_ce}}\pysiglinewithargsret{\sphinxbfcode{\sphinxupquote{get\_ce}}}{}{}
\sphinxAtStartPar
obtain inner ce object
\begin{quote}\begin{description}
\item[{Returns}] \leavevmode\begin{description}
\item[{CE object}] \leavevmode
\end{description}

\end{description}\end{quote}

\end{fulllineitems}

\index{get\_energy\_info\_from\_database() (pygace.gace.AbstractApp method)@\spxentry{get\_energy\_info\_from\_database()}\spxextra{pygace.gace.AbstractApp method}}

\begin{fulllineitems}
\phantomsection\label{\detokenize{pygace:pygace.gace.AbstractApp.get_energy_info_from_database}}\pysiglinewithargsret{\sphinxbfcode{\sphinxupquote{get\_energy\_info\_from\_database}}}{}{}
\sphinxAtStartPar
Initial energy database
\begin{quote}\begin{description}
\item[{Returns}] \leavevmode\begin{description}
\item[{None}] \leavevmode
\end{description}

\end{description}\end{quote}

\end{fulllineitems}

\index{ind\_to\_elis() (pygace.gace.AbstractApp method)@\spxentry{ind\_to\_elis()}\spxextra{pygace.gace.AbstractApp method}}

\begin{fulllineitems}
\phantomsection\label{\detokenize{pygace:pygace.gace.AbstractApp.ind_to_elis}}\pysiglinewithargsret{\sphinxbfcode{\sphinxupquote{ind\_to\_elis}}}{\emph{\DUrole{n}{individual}}}{}
\sphinxAtStartPar
Convert a object used in GA to a object used in \sphinxcode{\sphinxupquote{ATAT}}.

\sphinxAtStartPar
This method is used to convert a list which contains number
to a list containing chemistry element, e.g., {[}2,2,1,3{]} to
{[}‘Hf’, ‘Hf’, ‘O’, ‘Vac’{]}
\begin{quote}\begin{description}
\item[{Parameters}] \leavevmode\begin{description}
\item[{\sphinxstylestrong{individual: list}}] \leavevmode
\sphinxAtStartPar
Convert a list of \sphinxcode{\sphinxupquote{int}} to a list of chemistry element.

\end{description}

\item[{Returns}] \leavevmode\begin{description}
\item[{None}] \leavevmode
\end{description}

\item[{Raises}] \leavevmode\begin{description}
\item[{NotImplementedError}] \leavevmode
\sphinxAtStartPar
This method must be implemented in subclass.

\end{description}

\end{description}\end{quote}

\end{fulllineitems}

\index{initial() (pygace.gace.AbstractApp method)@\spxentry{initial()}\spxextra{pygace.gace.AbstractApp method}}

\begin{fulllineitems}
\phantomsection\label{\detokenize{pygace:pygace.gace.AbstractApp.initial}}\pysiglinewithargsret{\sphinxbfcode{\sphinxupquote{initial}}}{}{}
\sphinxAtStartPar
Initialization for GA simulation.
\begin{quote}\begin{description}
\item[{Returns}] \leavevmode\begin{description}
\item[{\sphinxstylestrong{toolbox}}] \leavevmode{[}Toolbox{]}
\sphinxAtStartPar
A Toolbox object contains responding parameters used in GA.

\end{description}

\end{description}\end{quote}

\end{fulllineitems}

\index{run() (pygace.gace.AbstractApp method)@\spxentry{run()}\spxextra{pygace.gace.AbstractApp method}}

\begin{fulllineitems}
\phantomsection\label{\detokenize{pygace:pygace.gace.AbstractApp.run}}\pysiglinewithargsret{\sphinxbfcode{\sphinxupquote{run}}}{\emph{\DUrole{n}{iter\_idx}\DUrole{o}{=}\DUrole{default_value}{1}}, \emph{\DUrole{n}{target\_epoch}\DUrole{o}{=}\DUrole{default_value}{0}}}{}~\begin{quote}\begin{description}
\item[{Parameters}] \leavevmode\begin{description}
\item[{\sphinxstylestrong{iter\_idx}}] \leavevmode{[}int{]}
\sphinxAtStartPar
The index of GA\sphinxhyphen{}to\sphinxhyphen{}CE iteration, in which a DFT calculation is
usually executed for update \sphinxcode{\sphinxupquote{eci.out}} file in \sphinxcode{\sphinxupquote{ATAT}}.

\item[{\sphinxstylestrong{target\_epoch}}] \leavevmode{[}int{]}
\sphinxAtStartPar
The repeat times of identical simulation of GA, for which the
results of GA simulation is relevant with random number, thus
a different GA simulation maybe select a different ground\sphinxhyphen{}state
configuration. This is useful especially in complex system with
substantial \sphinxcode{\sphinxupquote{sites}} to substitute for different configurations.

\end{description}

\item[{Returns}] \leavevmode\begin{description}
\item[{None}] \leavevmode
\end{description}

\item[{Raises}] \leavevmode\begin{description}
\item[{NotImplementedError}] \leavevmode
\sphinxAtStartPar
If this method is not implemented, this type error would be raised.

\end{description}

\end{description}\end{quote}

\end{fulllineitems}

\index{set\_dir() (pygace.gace.AbstractApp method)@\spxentry{set\_dir()}\spxextra{pygace.gace.AbstractApp method}}

\begin{fulllineitems}
\phantomsection\label{\detokenize{pygace:pygace.gace.AbstractApp.set_dir}}\pysiglinewithargsret{\sphinxbfcode{\sphinxupquote{set\_dir}}}{}{}
\sphinxAtStartPar
Initial directory.
\begin{quote}\begin{description}
\item[{Returns}] \leavevmode\begin{description}
\item[{None}] \leavevmode
\end{description}

\end{description}\end{quote}

\end{fulllineitems}

\index{transver\_to\_struct() (pygace.gace.AbstractApp method)@\spxentry{transver\_to\_struct()}\spxextra{pygace.gace.AbstractApp method}}

\begin{fulllineitems}
\phantomsection\label{\detokenize{pygace:pygace.gace.AbstractApp.transver_to_struct}}\pysiglinewithargsret{\sphinxbfcode{\sphinxupquote{transver\_to\_struct}}}{\emph{\DUrole{n}{element\_lis}}}{}
\sphinxAtStartPar
Convert element list to \sphinxtitleref{ATAT} \sphinxtitleref{str.out} file

\sphinxAtStartPar
The chemistry symbol in \sphinxcode{\sphinxupquote{element\_lis}} would be substituted in
\sphinxcode{\sphinxupquote{str.out}} file in \sphinxcode{\sphinxupquote{ATAT}}.
\begin{quote}\begin{description}
\item[{Parameters}] \leavevmode\begin{description}
\item[{\sphinxstylestrong{element\_lis}}] \leavevmode{[}list{]}
\sphinxAtStartPar
a list of chemistry symbol, e.g. {[}‘Hf’, ‘Hf’, ‘O’{]}

\item[{\sphinxstylestrong{test\_param1}}] \leavevmode{[}int{]}
\sphinxAtStartPar
the first test parameter

\end{description}

\item[{Returns}] \leavevmode\begin{description}
\item[{str}] \leavevmode
\sphinxAtStartPar
filename of \sphinxcode{\sphinxupquote{ATAT}} structure file, default \sphinxtitleref{str.out}

\end{description}

\end{description}\end{quote}

\end{fulllineitems}

\index{update\_ce() (pygace.gace.AbstractApp method)@\spxentry{update\_ce()}\spxextra{pygace.gace.AbstractApp method}}

\begin{fulllineitems}
\phantomsection\label{\detokenize{pygace:pygace.gace.AbstractApp.update_ce}}\pysiglinewithargsret{\sphinxbfcode{\sphinxupquote{update\_ce}}}{\emph{\DUrole{n}{site}\DUrole{o}{=}\DUrole{default_value}{1}}, \emph{\DUrole{n}{dirname}\DUrole{o}{=}\DUrole{default_value}{None}}}{}
\sphinxAtStartPar
Update inner CE object.

\sphinxAtStartPar
The parameters should contained the \sphinxcode{\sphinxupquote{site}} information in \sphinxcode{\sphinxupquote{MMAPS}}
and a path of directory containing output file after a CE fitting.
\begin{quote}\begin{description}
\item[{Parameters}] \leavevmode\begin{description}
\item[{\sphinxstylestrong{site: :obj: \textasciigrave{}int\textasciigrave{}, optional}}] \leavevmode
\sphinxAtStartPar
The number of \sphinxcode{\sphinxupquote{site}} in a crystal structure, which does not
contain a specific element instead of a site used to restore
different type of atoms to simulate alloy configurations in
\sphinxcode{\sphinxupquote{ATAT}}, more detail see \sphinxcode{\sphinxupquote{lat.in}} file in \sphinxcode{\sphinxupquote{ATAT}}.

\item[{\sphinxstylestrong{dirname: :obj: \textasciigrave{}str\textasciigrave{}, optional}}] \leavevmode
\end{description}

\item[{Returns}] \leavevmode\begin{description}
\item[{None}] \leavevmode
\end{description}

\end{description}\end{quote}

\end{fulllineitems}


\end{fulllineitems}

\index{AbstractRunner (class in pygace.gace)@\spxentry{AbstractRunner}\spxextra{class in pygace.gace}}

\begin{fulllineitems}
\phantomsection\label{\detokenize{pygace:pygace.gace.AbstractRunner}}\pysiglinewithargsret{\sphinxbfcode{\sphinxupquote{class }}\sphinxcode{\sphinxupquote{pygace.gace.}}\sphinxbfcode{\sphinxupquote{AbstractRunner}}}{\emph{\DUrole{n}{app}\DUrole{o}{=}\DUrole{default_value}{None}}, \emph{\DUrole{n}{iter\_idx}\DUrole{o}{=}\DUrole{default_value}{None}}}{}
\sphinxAtStartPar
Bases: \sphinxcode{\sphinxupquote{object}}

\sphinxAtStartPar
Abstract Runner for running a GACE simulation.

\sphinxAtStartPar
This object is used to execute a GACE simulation, user only need to
implement several interfaces to custom their application.
\begin{quote}\begin{description}
\item[{Parameters}] \leavevmode\begin{description}
\item[{\sphinxstylestrong{app}}] \leavevmode{[}subclass of AbstractApp{]}
\sphinxAtStartPar
A subclass object of AbstractApp, default is \sphinxtitleref{None}.

\item[{\sphinxstylestrong{iter\_idx}}] \leavevmode{[}int{]}
\sphinxAtStartPar
Index of GA\sphinxhyphen{}to\sphinxhyphen{}CE iteration, default is \sphinxtitleref{None}.

\end{description}

\item[{Raises}] \leavevmode\begin{description}
\item[{NotImplementedError}] \leavevmode
\sphinxAtStartPar
If \sphinxtitleref{run()} or \sphinxtitleref{print\_gs()} method is not implemented by subclass of
\sphinxtitleref{AbstractRunner}, this type of error would be raised.

\end{description}

\item[{Attributes}] \leavevmode\begin{description}
\item[{\sphinxstylestrong{app}}] \leavevmode{[}AbstractApp{]}
\sphinxAtStartPar
A subclass object of AbstractApp.

\item[{\sphinxstylestrong{iter\_idx}}] \leavevmode{[}int{]}
\sphinxAtStartPar
Index of GA\sphinxhyphen{}to\sphinxhyphen{}CE iteration.

\end{description}

\end{description}\end{quote}
\index{app (pygace.gace.AbstractRunner property)@\spxentry{app}\spxextra{pygace.gace.AbstractRunner property}}

\begin{fulllineitems}
\phantomsection\label{\detokenize{pygace:pygace.gace.AbstractRunner.app}}\pysigline{\sphinxbfcode{\sphinxupquote{property }}\sphinxbfcode{\sphinxupquote{app}}}~
\end{fulllineitems}

\index{iter\_idx (pygace.gace.AbstractRunner property)@\spxentry{iter\_idx}\spxextra{pygace.gace.AbstractRunner property}}

\begin{fulllineitems}
\phantomsection\label{\detokenize{pygace:pygace.gace.AbstractRunner.iter_idx}}\pysigline{\sphinxbfcode{\sphinxupquote{property }}\sphinxbfcode{\sphinxupquote{iter\_idx}}}~
\end{fulllineitems}

\index{print\_gs() (pygace.gace.AbstractRunner method)@\spxentry{print\_gs()}\spxextra{pygace.gace.AbstractRunner method}}

\begin{fulllineitems}
\phantomsection\label{\detokenize{pygace:pygace.gace.AbstractRunner.print_gs}}\pysiglinewithargsret{\sphinxbfcode{\sphinxupquote{print\_gs}}}{}{}
\sphinxAtStartPar
Function used to check ground\sphinxhyphen{}state configurations, to obtain their
formation energy predicted by CE, and to determine whether a DFT
calculation is needed to executed for next GA\sphinxhyphen{}to\sphinxhyphen{}CE iteration.
\begin{quote}\begin{description}
\item[{Returns}] \leavevmode\begin{description}
\item[{None}] \leavevmode
\end{description}

\item[{Raises}] \leavevmode\begin{description}
\item[{NotImplementedError}] \leavevmode
\sphinxAtStartPar
if this function is not implemented in their subclass, this type
error would be raised.

\end{description}

\end{description}\end{quote}

\end{fulllineitems}

\index{run() (pygace.gace.AbstractRunner method)@\spxentry{run()}\spxextra{pygace.gace.AbstractRunner method}}

\begin{fulllineitems}
\phantomsection\label{\detokenize{pygace:pygace.gace.AbstractRunner.run}}\pysiglinewithargsret{\sphinxbfcode{\sphinxupquote{run}}}{}{}
\sphinxAtStartPar
Main runction for running GACE simulation.
\begin{quote}\begin{description}
\item[{Returns}] \leavevmode\begin{description}
\item[{None}] \leavevmode
\end{description}

\item[{Raises}] \leavevmode\begin{description}
\item[{NotImplementedError}] \leavevmode
\sphinxAtStartPar
if this function is not implemented in their subclass, this type
error would be raised.

\end{description}

\end{description}\end{quote}

\end{fulllineitems}


\end{fulllineitems}



\subsection{pygace.general\_gace module}
\label{\detokenize{pygace:module-pygace.general_gace}}\label{\detokenize{pygace:pygace-general-gace-module}}\index{module@\spxentry{module}!pygace.general\_gace@\spxentry{pygace.general\_gace}}\index{pygace.general\_gace@\spxentry{pygace.general\_gace}!module@\spxentry{module}}
\sphinxAtStartPar
A GA\sphinxhyphen{}to\sphinxhyphen{}CE example of oxygen\sphinxhyphen{}vacancy\sphinxhyphen{}containing HfO2 system given in
this module.
\index{GeneralApp (class in pygace.general\_gace)@\spxentry{GeneralApp}\spxextra{class in pygace.general\_gace}}

\begin{fulllineitems}
\phantomsection\label{\detokenize{pygace:pygace.general_gace.GeneralApp}}\pysiglinewithargsret{\sphinxbfcode{\sphinxupquote{class }}\sphinxcode{\sphinxupquote{pygace.general\_gace.}}\sphinxbfcode{\sphinxupquote{GeneralApp}}}{\emph{\DUrole{n}{ele\_type\_list}}, \emph{\DUrole{n}{defect\_concentrations}}, \emph{\DUrole{n}{ce\_dirname}\DUrole{o}{=}\DUrole{default_value}{\textquotesingle{}./data/iter1\textquotesingle{}}}, \emph{\DUrole{n}{params\_config\_dict}\DUrole{o}{=}\DUrole{default_value}{None}}}{}
\sphinxAtStartPar
Bases: {\hyperref[\detokenize{pygace:pygace.gace.AbstractApp}]{\sphinxcrossref{\sphinxcode{\sphinxupquote{pygace.gace.AbstractApp}}}}}

\sphinxAtStartPar
An app of general system which is implemented from AbstractApp object

\sphinxAtStartPar
This object is used to execute a GACE simulation, user only need to
implement several interfaces to custom their application.
\begin{quote}\begin{description}
\item[{Parameters}] \leavevmode\begin{description}
\item[{\sphinxstylestrong{ce\_site: int}}] \leavevmode
\sphinxAtStartPar
the concept of site used in ATAT program.

\item[{\sphinxstylestrong{ce\_dirname: str}}] \leavevmode
\sphinxAtStartPar
The name of a directory which contain information of MMAPS or MAPS
running

\item[{\sphinxstylestrong{ele\_1st: str}}] \leavevmode
\sphinxAtStartPar
The first type of element in the \sphinxcode{\sphinxupquote{site}} in \sphinxcode{\sphinxupquote{ATAT}}.

\item[{\sphinxstylestrong{ele\_2nd: str}}] \leavevmode
\sphinxAtStartPar
The second type of element in the \sphinxcode{\sphinxupquote{site}} in \sphinxcode{\sphinxupquote{ATAT}}.

\item[{\sphinxstylestrong{params\_config\_dict: dirt}}] \leavevmode
\sphinxAtStartPar
Parameter dict used to custom GACE AbstractApp.

\end{description}

\item[{Attributes}] \leavevmode\begin{description}
\item[{\sphinxstylestrong{app}}] \leavevmode{[}AbstractApp{]}
\sphinxAtStartPar
A subclass object of AbstractApp.

\item[{\sphinxstylestrong{iter\_idx}}] \leavevmode{[}int{]}
\sphinxAtStartPar
Index of GA\sphinxhyphen{}to\sphinxhyphen{}CE iteration.

\end{description}

\end{description}\end{quote}
\index{evalEnergy() (pygace.general\_gace.GeneralApp method)@\spxentry{evalEnergy()}\spxextra{pygace.general\_gace.GeneralApp method}}

\begin{fulllineitems}
\phantomsection\label{\detokenize{pygace:pygace.general_gace.GeneralApp.evalEnergy}}\pysiglinewithargsret{\sphinxbfcode{\sphinxupquote{evalEnergy}}}{\emph{\DUrole{n}{individual}}}{}
\sphinxAtStartPar
Evaluation function for the ground\sphinxhyphen{}state searching problem.

\sphinxAtStartPar
The problem is to determine a configuration of n vacancies
on a crystalline structures such that the energy of crystalline
structures can obtain minimum value.
\begin{quote}\begin{description}
\item[{Parameters}] \leavevmode\begin{description}
\item[{\sphinxstylestrong{individual}}] \leavevmode
\end{description}

\item[{Returns}] \leavevmode\begin{description}
\item[{float}] \leavevmode
\sphinxAtStartPar
Fittness value

\end{description}

\end{description}\end{quote}

\end{fulllineitems}

\index{ind\_to\_elis() (pygace.general\_gace.GeneralApp method)@\spxentry{ind\_to\_elis()}\spxextra{pygace.general\_gace.GeneralApp method}}

\begin{fulllineitems}
\phantomsection\label{\detokenize{pygace:pygace.general_gace.GeneralApp.ind_to_elis}}\pysiglinewithargsret{\sphinxbfcode{\sphinxupquote{ind\_to\_elis}}}{\emph{\DUrole{n}{individual}}}{}
\sphinxAtStartPar
Convert individual (number list) to element list
\begin{quote}\begin{description}
\item[{Parameters}] \leavevmode\begin{description}
\item[{\sphinxstylestrong{individual}}] \leavevmode
\end{description}

\item[{Returns}] \leavevmode\begin{description}
\item[{list}] \leavevmode
\sphinxAtStartPar
A list of element symbol string.

\end{description}

\end{description}\end{quote}

\end{fulllineitems}

\index{run() (pygace.general\_gace.GeneralApp method)@\spxentry{run()}\spxextra{pygace.general\_gace.GeneralApp method}}

\begin{fulllineitems}
\phantomsection\label{\detokenize{pygace:pygace.general_gace.GeneralApp.run}}\pysiglinewithargsret{\sphinxbfcode{\sphinxupquote{run}}}{\emph{\DUrole{n}{iter\_idx}\DUrole{o}{=}\DUrole{default_value}{1}}, \emph{\DUrole{n}{default\_epoch}\DUrole{o}{=}\DUrole{default_value}{4}}, \emph{\DUrole{n}{target\_epoch}\DUrole{o}{=}\DUrole{default_value}{4}}, \emph{\DUrole{n}{cross\_method}\DUrole{o}{=}\DUrole{default_value}{1}}, \emph{\DUrole{n}{cross\_num}\DUrole{o}{=}\DUrole{default_value}{8}}, \emph{\DUrole{n}{cp\_fname\_prefix}\DUrole{o}{=}\DUrole{default_value}{\textquotesingle{}ground\_states\_iter\textquotesingle{}}}, \emph{\DUrole{n}{task\_prefix}\DUrole{o}{=}\DUrole{default_value}{\textquotesingle{}general\sphinxhyphen{}app\textquotesingle{}}}, \emph{\DUrole{n}{gs\_selection}\DUrole{o}{=}\DUrole{default_value}{1}}}{}
\sphinxAtStartPar
Main function to run a GACE simulation which will be called by
\sphinxtitleref{AbstractRunner}.
\begin{quote}\begin{description}
\item[{Parameters}] \leavevmode\begin{description}
\item[{\sphinxstylestrong{iter\_idx}}] \leavevmode{[}int{]}
\sphinxAtStartPar
Determine which iteration the ECI is used in.

\item[{\sphinxstylestrong{target\_epoch}}] \leavevmode{[}int{]}
\sphinxAtStartPar
Iteration in GA simulation.

\item[{\sphinxstylestrong{default\_epoch}}] \leavevmode{[}int{]}
\sphinxAtStartPar
Default epoch setting for GA.

\item[{\sphinxstylestrong{target\_epoch}}] \leavevmode{[}int{]}
\sphinxAtStartPar
Target epoch for GA.

\item[{\sphinxstylestrong{cross\_method}}] \leavevmode{[}int{]}
\sphinxAtStartPar
Crossover operator type.

\item[{\sphinxstylestrong{cross\_num :}}] \leavevmode
\sphinxAtStartPar
The exchange number used in crossover operator.

\item[{\sphinxstylestrong{cp\_fname\_prefix}}] \leavevmode{[}str{]}
\sphinxAtStartPar
The prefix of checkpoint file name.

\item[{\sphinxstylestrong{task\_prefix}}] \leavevmode{[}str{]}
\sphinxAtStartPar
The prefix of task filename of a single simulation.

\item[{\sphinxstylestrong{gs\_selection}}] \leavevmode{[}int{]}
\sphinxAtStartPar
Ground\sphinxhyphen{}state structures selected from \sphinxtitleref{target\_epoch} GA simulation.

\end{description}

\item[{Returns}] \leavevmode\begin{description}
\item[{None}] \leavevmode
\end{description}

\end{description}\end{quote}

\end{fulllineitems}

\index{update\_defect\_concentration() (pygace.general\_gace.GeneralApp method)@\spxentry{update\_defect\_concentration()}\spxextra{pygace.general\_gace.GeneralApp method}}

\begin{fulllineitems}
\phantomsection\label{\detokenize{pygace:pygace.general_gace.GeneralApp.update_defect_concentration}}\pysiglinewithargsret{\sphinxbfcode{\sphinxupquote{update\_defect\_concentration}}}{\emph{\DUrole{n}{c}\DUrole{o}{=}\DUrole{default_value}{None}}}{}~
\end{fulllineitems}


\end{fulllineitems}

\index{GeneralEleIndv (class in pygace.general\_gace)@\spxentry{GeneralEleIndv}\spxextra{class in pygace.general\_gace}}

\begin{fulllineitems}
\phantomsection\label{\detokenize{pygace:pygace.general_gace.GeneralEleIndv}}\pysiglinewithargsret{\sphinxbfcode{\sphinxupquote{class }}\sphinxcode{\sphinxupquote{pygace.general\_gace.}}\sphinxbfcode{\sphinxupquote{GeneralEleIndv}}}{\emph{\DUrole{n}{ele\_lis}}, \emph{\DUrole{n}{app}\DUrole{o}{=}\DUrole{default_value}{None}}}{}
\sphinxAtStartPar
Bases: {\hyperref[\detokenize{pygace:pygace.utility.EleIndv}]{\sphinxcrossref{\sphinxcode{\sphinxupquote{pygace.utility.EleIndv}}}}}

\sphinxAtStartPar
A class that use list chemistry element to represent individual.
\begin{quote}\begin{description}
\item[{Parameters}] \leavevmode\begin{description}
\item[{\sphinxstylestrong{ele\_lis}}] \leavevmode{[}list{]}
\sphinxAtStartPar
A list of chemistry element.

\item[{\sphinxstylestrong{app}}] \leavevmode{[}AbstractApp{]}
\sphinxAtStartPar
An application of GACE which is used to obtain ground\sphinxhyphen{}state
structures based generic algorithm and cluster expansion method.

\end{description}

\item[{Attributes}] \leavevmode\begin{description}
\item[{\sphinxstylestrong{app: AbstractApp}}] \leavevmode
\sphinxAtStartPar
An application handling GACE running process.

\item[{\sphinxstylestrong{ele\_lis: list}}] \leavevmode
\sphinxAtStartPar
A list of chemistry element string.

\end{description}

\end{description}\end{quote}
\index{ce\_energy (pygace.general\_gace.GeneralEleIndv property)@\spxentry{ce\_energy}\spxextra{pygace.general\_gace.GeneralEleIndv property}}

\begin{fulllineitems}
\phantomsection\label{\detokenize{pygace:pygace.general_gace.GeneralEleIndv.ce_energy}}\pysigline{\sphinxbfcode{\sphinxupquote{property }}\sphinxbfcode{\sphinxupquote{ce\_energy}}}
\sphinxAtStartPar
Return CE energy
\begin{quote}\begin{description}
\item[{Returns}] \leavevmode\begin{description}
\item[{float}] \leavevmode
\sphinxAtStartPar
Energy predicted by CE method.

\end{description}

\end{description}\end{quote}

\end{fulllineitems}

\index{ce\_energy\_corrdump (pygace.general\_gace.GeneralEleIndv property)@\spxentry{ce\_energy\_corrdump}\spxextra{pygace.general\_gace.GeneralEleIndv property}}

\begin{fulllineitems}
\phantomsection\label{\detokenize{pygace:pygace.general_gace.GeneralEleIndv.ce_energy_corrdump}}\pysigline{\sphinxbfcode{\sphinxupquote{property }}\sphinxbfcode{\sphinxupquote{ce\_energy\_corrdump}}}
\sphinxAtStartPar
Return relative energy defined in \sphinxcode{\sphinxupquote{ATAT}} and computed by \sphinxcode{\sphinxupquote{corrdump}}
program.
\begin{quote}\begin{description}
\item[{Returns}] \leavevmode\begin{description}
\item[{float}] \leavevmode
\sphinxAtStartPar
Relative energy generated by \sphinxcode{\sphinxupquote{corrdump}} program.

\end{description}

\end{description}\end{quote}

\end{fulllineitems}

\index{dft\_energy() (pygace.general\_gace.GeneralEleIndv method)@\spxentry{dft\_energy()}\spxextra{pygace.general\_gace.GeneralEleIndv method}}

\begin{fulllineitems}
\phantomsection\label{\detokenize{pygace:pygace.general_gace.GeneralEleIndv.dft_energy}}\pysiglinewithargsret{\sphinxbfcode{\sphinxupquote{dft\_energy}}}{\emph{\DUrole{n}{iters}\DUrole{o}{=}\DUrole{default_value}{None}}, \emph{\DUrole{n}{vasp\_cmd}\DUrole{o}{=}\DUrole{default_value}{None}}, \emph{\DUrole{n}{update\_eci}\DUrole{o}{=}\DUrole{default_value}{True}}}{}
\sphinxAtStartPar
Return DFT energy
\begin{quote}\begin{description}
\item[{Parameters}] \leavevmode\begin{description}
\item[{\sphinxstylestrong{iters}}] \leavevmode{[}int{]}
\sphinxAtStartPar
index of iteration of GA\sphinxhyphen{}to\sphinxhyphen{}CE

\end{description}

\item[{Returns}] \leavevmode\begin{description}
\item[{None or float}] \leavevmode
\end{description}

\end{description}\end{quote}

\end{fulllineitems}

\index{run\_fake\_vasp() (pygace.general\_gace.GeneralEleIndv method)@\spxentry{run\_fake\_vasp()}\spxextra{pygace.general\_gace.GeneralEleIndv method}}

\begin{fulllineitems}
\phantomsection\label{\detokenize{pygace:pygace.general_gace.GeneralEleIndv.run_fake_vasp}}\pysiglinewithargsret{\sphinxbfcode{\sphinxupquote{run\_fake\_vasp}}}{}{}~
\end{fulllineitems}

\index{run\_vasp() (pygace.general\_gace.GeneralEleIndv method)@\spxentry{run\_vasp()}\spxextra{pygace.general\_gace.GeneralEleIndv method}}

\begin{fulllineitems}
\phantomsection\label{\detokenize{pygace:pygace.general_gace.GeneralEleIndv.run_vasp}}\pysiglinewithargsret{\sphinxbfcode{\sphinxupquote{run\_vasp}}}{\emph{\DUrole{n}{vasp\_cmd}}}{}~
\end{fulllineitems}


\end{fulllineitems}

\index{Runner (class in pygace.general\_gace)@\spxentry{Runner}\spxextra{class in pygace.general\_gace}}

\begin{fulllineitems}
\phantomsection\label{\detokenize{pygace:pygace.general_gace.Runner}}\pysiglinewithargsret{\sphinxbfcode{\sphinxupquote{class }}\sphinxcode{\sphinxupquote{pygace.general\_gace.}}\sphinxbfcode{\sphinxupquote{Runner}}}{\emph{\DUrole{n}{app}\DUrole{o}{=}\DUrole{default_value}{None}}, \emph{\DUrole{n}{iter\_idx}\DUrole{o}{=}\DUrole{default_value}{None}}}{}
\sphinxAtStartPar
Bases: {\hyperref[\detokenize{pygace:pygace.gace.AbstractRunner}]{\sphinxcrossref{\sphinxcode{\sphinxupquote{pygace.gace.AbstractRunner}}}}}

\sphinxAtStartPar
A runner for running a GACE simulation.

\sphinxAtStartPar
This object is used to execute a GACE simulation in HfO2 system.
\begin{quote}\begin{description}
\item[{Parameters}] \leavevmode\begin{description}
\item[{\sphinxstylestrong{app}}] \leavevmode{[}subclass of GeneralApp{]}
\sphinxAtStartPar
A subclass object of GeneralApp, default is \sphinxtitleref{None}.

\item[{\sphinxstylestrong{iter\_idx}}] \leavevmode{[}int{]}
\sphinxAtStartPar
Index of GA\sphinxhyphen{}to\sphinxhyphen{}CE iteration, default is \sphinxtitleref{None}.

\end{description}

\item[{Attributes}] \leavevmode\begin{description}
\item[{\sphinxstylestrong{app}}] \leavevmode{[}GeneralApp{]}
\sphinxAtStartPar
A subclass object of GeneralApp.

\item[{\sphinxstylestrong{iter\_idx}}] \leavevmode{[}int{]}
\sphinxAtStartPar
Index of GA\sphinxhyphen{}to\sphinxhyphen{}CE iteration.

\end{description}

\end{description}\end{quote}
\index{compare\_gs() (pygace.general\_gace.Runner method)@\spxentry{compare\_gs()}\spxextra{pygace.general\_gace.Runner method}}

\begin{fulllineitems}
\phantomsection\label{\detokenize{pygace:pygace.general_gace.Runner.compare_gs}}\pysiglinewithargsret{\sphinxbfcode{\sphinxupquote{compare\_gs}}}{\emph{\DUrole{n}{new\_gs}}, \emph{\DUrole{n}{old\_gs}}}{}
\sphinxAtStartPar
Determine whether current and previous ground\sphinxhyphen{}state are identical.
\begin{quote}\begin{description}
\item[{Parameters}] \leavevmode\begin{description}
\item[{\sphinxstylestrong{new\_gs}}] \leavevmode{[}str{]}
\sphinxAtStartPar
Ground\sphinxhyphen{}state configuration predicted by current iteration.

\item[{\sphinxstylestrong{old\_gs :}}] \leavevmode
\sphinxAtStartPar
Ground\sphinxhyphen{}state configuration predicted by previous iteration.

\end{description}

\item[{Returns}] \leavevmode\begin{description}
\item[{bool}] \leavevmode
\item[{Raises:}] \leavevmode
\item[{RuntimeError}] \leavevmode
\sphinxAtStartPar
when the number of point defect (oxygen vacancy here) is not equal
in two iteration.

\end{description}

\end{description}\end{quote}

\end{fulllineitems}

\index{print\_gs() (pygace.general\_gace.Runner method)@\spxentry{print\_gs()}\spxextra{pygace.general\_gace.Runner method}}

\begin{fulllineitems}
\phantomsection\label{\detokenize{pygace:pygace.general_gace.Runner.print_gs}}\pysiglinewithargsret{\sphinxbfcode{\sphinxupquote{print\_gs}}}{\emph{\DUrole{n}{vasp\_cmd}\DUrole{o}{=}\DUrole{default_value}{None}}}{}
\sphinxAtStartPar
Function used to extract ground\sphinxhyphen{}state information from pickle file
saved during GACE running.
\begin{quote}\begin{description}
\item[{Returns}] \leavevmode\begin{description}
\item[{None}] \leavevmode
\end{description}

\end{description}\end{quote}

\end{fulllineitems}

\index{run() (pygace.general\_gace.Runner method)@\spxentry{run()}\spxextra{pygace.general\_gace.Runner method}}

\begin{fulllineitems}
\phantomsection\label{\detokenize{pygace:pygace.general_gace.Runner.run}}\pysiglinewithargsret{\sphinxbfcode{\sphinxupquote{run}}}{}{}
\sphinxAtStartPar
Main function to run.
\begin{quote}\begin{description}
\item[{Returns}] \leavevmode\begin{description}
\item[{None}] \leavevmode
\end{description}

\end{description}\end{quote}

\end{fulllineitems}

\index{str2energy() (pygace.general\_gace.Runner method)@\spxentry{str2energy()}\spxextra{pygace.general\_gace.Runner method}}

\begin{fulllineitems}
\phantomsection\label{\detokenize{pygace:pygace.general_gace.Runner.str2energy}}\pysiglinewithargsret{\sphinxbfcode{\sphinxupquote{str2energy}}}{\emph{\DUrole{n}{string}}}{}
\sphinxAtStartPar
Obtain energy from string consists of numbers joined by ‘\_’, e.g.,
\sphinxcode{\sphinxupquote{\textquotesingle{}1\_2\_3\_19\_\textquotesingle{}}}, in which the number is the position index in lattice
structure template file.
\begin{quote}\begin{description}
\item[{Parameters}] \leavevmode\begin{description}
\item[{\sphinxstylestrong{string}}] \leavevmode{[}str{]}
\sphinxAtStartPar
The string consists by index of point defect.

\item[{\sphinxstylestrong{Returns}}] \leavevmode
\item[{\sphinxstylestrong{——\sphinxhyphen{}}}] \leavevmode
\item[{\sphinxstylestrong{float}}] \leavevmode
\sphinxAtStartPar
CE energy.

\end{description}

\end{description}\end{quote}

\end{fulllineitems}


\end{fulllineitems}



\subsection{pygace.parse module}
\label{\detokenize{pygace:module-pygace.parse}}\label{\detokenize{pygace:pygace-parse-module}}\index{module@\spxentry{module}!pygace.parse@\spxentry{pygace.parse}}\index{pygace.parse@\spxentry{pygace.parse}!module@\spxentry{module}}\index{GaceMcsqs (class in pygace.parse)@\spxentry{GaceMcsqs}\spxextra{class in pygace.parse}}

\begin{fulllineitems}
\phantomsection\label{\detokenize{pygace:pygace.parse.GaceMcsqs}}\pysiglinewithargsret{\sphinxbfcode{\sphinxupquote{class }}\sphinxcode{\sphinxupquote{pygace.parse.}}\sphinxbfcode{\sphinxupquote{GaceMcsqs}}}{\emph{\DUrole{n}{structure}}}{}
\sphinxAtStartPar
Bases: \sphinxcode{\sphinxupquote{object}}
\index{nb\_occu\_sites (pygace.parse.GaceMcsqs property)@\spxentry{nb\_occu\_sites}\spxextra{pygace.parse.GaceMcsqs property}}

\begin{fulllineitems}
\phantomsection\label{\detokenize{pygace:pygace.parse.GaceMcsqs.nb_occu_sites}}\pysigline{\sphinxbfcode{\sphinxupquote{property }}\sphinxbfcode{\sphinxupquote{nb\_occu\_sites}}}~
\end{fulllineitems}

\index{structure\_from\_string() (pygace.parse.GaceMcsqs static method)@\spxentry{structure\_from\_string()}\spxextra{pygace.parse.GaceMcsqs static method}}

\begin{fulllineitems}
\phantomsection\label{\detokenize{pygace:pygace.parse.GaceMcsqs.structure_from_string}}\pysiglinewithargsret{\sphinxbfcode{\sphinxupquote{static }}\sphinxbfcode{\sphinxupquote{structure\_from\_string}}}{\emph{\DUrole{n}{data}}}{}
\sphinxAtStartPar
Parses a rndstr.in or lat.in file into pymatgen’s
Structure format.
\begin{quote}\begin{description}
\item[{Parameters}] \leavevmode
\sphinxAtStartPar
\sphinxstyleliteralstrong{\sphinxupquote{data}} \textendash{} contents of a rndstr.in or lat.in file

\item[{Returns}] \leavevmode
\sphinxAtStartPar
Structure object

\end{description}\end{quote}

\end{fulllineitems}

\index{to\_string() (pygace.parse.GaceMcsqs method)@\spxentry{to\_string()}\spxextra{pygace.parse.GaceMcsqs method}}

\begin{fulllineitems}
\phantomsection\label{\detokenize{pygace:pygace.parse.GaceMcsqs.to_string}}\pysiglinewithargsret{\sphinxbfcode{\sphinxupquote{to\_string}}}{}{}
\sphinxAtStartPar
Returns a structure in mcsqs rndstr.in format.
:return (str):

\end{fulllineitems}

\index{to\_template() (pygace.parse.GaceMcsqs method)@\spxentry{to\_template()}\spxextra{pygace.parse.GaceMcsqs method}}

\begin{fulllineitems}
\phantomsection\label{\detokenize{pygace:pygace.parse.GaceMcsqs.to_template}}\pysiglinewithargsret{\sphinxbfcode{\sphinxupquote{to\_template}}}{\emph{\DUrole{n}{ele\_dict}\DUrole{o}{=}\DUrole{default_value}{None}}}{}
\sphinxAtStartPar
Returns a structure in lat.in format template.
\begin{quote}\begin{description}
\item[{Parameters}] \leavevmode\begin{description}
\item[{\sphinxstylestrong{ele\_dict}}] \leavevmode{[}dict{]}
\sphinxAtStartPar
A dict used to convert element type in ATAT to
element type in pymatgen.

\end{description}

\item[{Returns}] \leavevmode\begin{description}
\item[{str}] \leavevmode
\end{description}

\end{description}\end{quote}

\end{fulllineitems}


\end{fulllineitems}



\subsection{pygace.utility module}
\label{\detokenize{pygace:module-pygace.utility}}\label{\detokenize{pygace:pygace-utility-module}}\index{module@\spxentry{module}!pygace.utility@\spxentry{pygace.utility}}\index{pygace.utility@\spxentry{pygace.utility}!module@\spxentry{module}}
\sphinxAtStartPar
There are some general helper function defined in this module.
\index{EleIndv (class in pygace.utility)@\spxentry{EleIndv}\spxextra{class in pygace.utility}}

\begin{fulllineitems}
\phantomsection\label{\detokenize{pygace:pygace.utility.EleIndv}}\pysiglinewithargsret{\sphinxbfcode{\sphinxupquote{class }}\sphinxcode{\sphinxupquote{pygace.utility.}}\sphinxbfcode{\sphinxupquote{EleIndv}}}{\emph{\DUrole{n}{ele\_lis}}, \emph{\DUrole{n}{app}\DUrole{o}{=}\DUrole{default_value}{None}}}{}
\sphinxAtStartPar
Bases: \sphinxcode{\sphinxupquote{object}}

\sphinxAtStartPar
A class that use list chemistry element to represent individual.
\begin{quote}\begin{description}
\item[{Parameters}] \leavevmode\begin{description}
\item[{\sphinxstylestrong{ele\_lis}}] \leavevmode{[}list{]}
\sphinxAtStartPar
A list of chemistry element.

\item[{\sphinxstylestrong{app}}] \leavevmode{[}AbstractApp{]}
\sphinxAtStartPar
An application of GACE which is used to obtain ground\sphinxhyphen{}state
structures based generic algorithm and cluster expansion method.

\end{description}

\item[{Attributes}] \leavevmode\begin{description}
\item[{\sphinxstylestrong{app: AbstractApp}}] \leavevmode
\sphinxAtStartPar
An application handling GACE running process.

\item[{\sphinxstylestrong{ele\_lis: list}}] \leavevmode
\sphinxAtStartPar
A list of chemistry element string.

\end{description}

\end{description}\end{quote}
\index{ce\_energy (pygace.utility.EleIndv property)@\spxentry{ce\_energy}\spxextra{pygace.utility.EleIndv property}}

\begin{fulllineitems}
\phantomsection\label{\detokenize{pygace:pygace.utility.EleIndv.ce_energy}}\pysigline{\sphinxbfcode{\sphinxupquote{property }}\sphinxbfcode{\sphinxupquote{ce\_energy}}}
\sphinxAtStartPar
The absolute energy predicted by CE.
\begin{quote}\begin{description}
\item[{Returns}] \leavevmode\begin{description}
\item[{float}] \leavevmode
\sphinxAtStartPar
CE absolute energy.

\end{description}

\end{description}\end{quote}

\end{fulllineitems}

\index{ce\_energy\_ref (pygace.utility.EleIndv property)@\spxentry{ce\_energy\_ref}\spxextra{pygace.utility.EleIndv property}}

\begin{fulllineitems}
\phantomsection\label{\detokenize{pygace:pygace.utility.EleIndv.ce_energy_ref}}\pysigline{\sphinxbfcode{\sphinxupquote{property }}\sphinxbfcode{\sphinxupquote{ce\_energy\_ref}}}
\sphinxAtStartPar
The relative energy predicted by CE.
\begin{quote}\begin{description}
\item[{Returns}] \leavevmode\begin{description}
\item[{float}] \leavevmode
\sphinxAtStartPar
CE relative energy

\end{description}

\end{description}\end{quote}

\end{fulllineitems}

\index{ce\_object (pygace.utility.EleIndv property)@\spxentry{ce\_object}\spxextra{pygace.utility.EleIndv property}}

\begin{fulllineitems}
\phantomsection\label{\detokenize{pygace:pygace.utility.EleIndv.ce_object}}\pysigline{\sphinxbfcode{\sphinxupquote{property }}\sphinxbfcode{\sphinxupquote{ce\_object}}}~
\end{fulllineitems}

\index{dft\_energy() (pygace.utility.EleIndv method)@\spxentry{dft\_energy()}\spxextra{pygace.utility.EleIndv method}}

\begin{fulllineitems}
\phantomsection\label{\detokenize{pygace:pygace.utility.EleIndv.dft_energy}}\pysiglinewithargsret{\sphinxbfcode{\sphinxupquote{dft\_energy}}}{\emph{\DUrole{n}{iters}\DUrole{o}{=}\DUrole{default_value}{None}}}{}
\sphinxAtStartPar
The DFT energy of individual represented by element list.
\begin{quote}\begin{description}
\item[{Parameters}] \leavevmode\begin{description}
\item[{\sphinxstylestrong{iters}}] \leavevmode{[}int{]}
\sphinxAtStartPar
Specific which iteration DFT energy are computed.

\end{description}

\item[{Returns}] \leavevmode\begin{description}
\item[{float or None}] \leavevmode
\sphinxAtStartPar
If the directory of DFT calculated exists and the calculation has
been finished the DFT energy will be return, or a new DFT
calculation directory will be created and first\sphinxhyphen{}principles
calculation should be performed in this directory.

\end{description}

\end{description}\end{quote}

\end{fulllineitems}

\index{is\_correct() (pygace.utility.EleIndv method)@\spxentry{is\_correct()}\spxextra{pygace.utility.EleIndv method}}

\begin{fulllineitems}
\phantomsection\label{\detokenize{pygace:pygace.utility.EleIndv.is_correct}}\pysiglinewithargsret{\sphinxbfcode{\sphinxupquote{is\_correct}}}{}{}
\sphinxAtStartPar
Determine whether the dft energy and the ce energy of indv equivalent
are identical within error.
\begin{quote}\begin{description}
\item[{Returns}] \leavevmode\begin{description}
\item[{bool}] \leavevmode
\end{description}

\end{description}\end{quote}

\end{fulllineitems}

\index{set\_app() (pygace.utility.EleIndv method)@\spxentry{set\_app()}\spxextra{pygace.utility.EleIndv method}}

\begin{fulllineitems}
\phantomsection\label{\detokenize{pygace:pygace.utility.EleIndv.set_app}}\pysiglinewithargsret{\sphinxbfcode{\sphinxupquote{set\_app}}}{\emph{\DUrole{n}{app}}}{}~
\end{fulllineitems}


\end{fulllineitems}

\index{compare\_crystal() (in module pygace.utility)@\spxentry{compare\_crystal()}\spxextra{in module pygace.utility}}

\begin{fulllineitems}
\phantomsection\label{\detokenize{pygace:pygace.utility.compare_crystal}}\pysiglinewithargsret{\sphinxcode{\sphinxupquote{pygace.utility.}}\sphinxbfcode{\sphinxupquote{compare\_crystal}}}{\emph{\DUrole{n}{str1}}, \emph{\DUrole{n}{str2}}, \emph{\DUrole{n}{compare\_crystal\_cmd}\DUrole{o}{=}\DUrole{default_value}{\textquotesingle{}CompareCrystal \textquotesingle{}}}, \emph{\DUrole{n}{str\_template}\DUrole{o}{=}\DUrole{default_value}{None}}, \emph{\DUrole{o}{**}\DUrole{n}{kwargs}}}{}
\sphinxAtStartPar
To determine whether structures are identical based crystal symmetry
analysis. The program used in this package is based on \sphinxcode{\sphinxupquote{XtalComp}} library
which developed by David C. Lonie.
\begin{quote}\begin{description}
\item[{Parameters}] \leavevmode\begin{description}
\item[{\sphinxstylestrong{str1}}] \leavevmode{[}str{]}
\sphinxAtStartPar
The first string used to represent elements .

\item[{\sphinxstylestrong{str2}}] \leavevmode{[}str{]}
\sphinxAtStartPar
The second string used to represent elements.

\item[{\sphinxstylestrong{compare\_crystal\_cmd}}] \leavevmode{[}str{]}
\sphinxAtStartPar
The program developed to determine whether two
crystal structures are identical, default \sphinxtitleref{CompareCrystal}.

\item[{\sphinxstylestrong{str\_template}}] \leavevmode{[}str{]}
\sphinxAtStartPar
String template for the definition of lattice site.

\item[{\sphinxstylestrong{kwargs}}] \leavevmode{[}dict arguments{]}
\sphinxAtStartPar
Other arguments used in \sphinxtitleref{compare\_crystal\_cmd}.

\end{description}

\item[{Returns}] \leavevmode\begin{description}
\item[{bool}] \leavevmode
\end{description}

\end{description}\end{quote}
\subsubsection*{References}

\sphinxAtStartPar
\sphinxurl{https://github.com/allisonvacanti/XtalComp}

\end{fulllineitems}

\index{copytree() (in module pygace.utility)@\spxentry{copytree()}\spxextra{in module pygace.utility}}

\begin{fulllineitems}
\phantomsection\label{\detokenize{pygace:pygace.utility.copytree}}\pysiglinewithargsret{\sphinxcode{\sphinxupquote{pygace.utility.}}\sphinxbfcode{\sphinxupquote{copytree}}}{\emph{\DUrole{n}{src}}, \emph{\DUrole{n}{dst}}, \emph{\DUrole{n}{symlinks}\DUrole{o}{=}\DUrole{default_value}{False}}, \emph{\DUrole{n}{ignore}\DUrole{o}{=}\DUrole{default_value}{None}}}{}
\sphinxAtStartPar
Recursively copy a directory tree using copy2().

\sphinxAtStartPar
The destination directory must not already exist.
If exception(s) occur, an Error is raised with a list of reasons.

\sphinxAtStartPar
If the optional symlinks flag is true, symbolic links in the
source tree result in symbolic links in the destination tree; if
it is false, the contents of the files pointed to by symbolic
links are copied.

\sphinxAtStartPar
The optional ignore argument is a callable. If given, it
is called with the \sphinxtitleref{src} parameter, which is the directory
being visited by copytree(), and \sphinxtitleref{names} which is the list of
\sphinxtitleref{src} contents, as returned by os.listdir():
\begin{quote}

\sphinxAtStartPar
callable(src, names) \sphinxhyphen{}\textgreater{} ignored\_names
\end{quote}

\sphinxAtStartPar
Since copytree() is called recursively, the callable will be
called once for each directory that is copied. It returns a
list of names relative to the \sphinxtitleref{src} directory that should
not be copied.

\sphinxAtStartPar
XXX Consider this example code rather than the ultimate tool.

\end{fulllineitems}

\index{get\_num\_lis() (in module pygace.utility)@\spxentry{get\_num\_lis()}\spxextra{in module pygace.utility}}

\begin{fulllineitems}
\phantomsection\label{\detokenize{pygace:pygace.utility.get_num_lis}}\pysiglinewithargsret{\sphinxcode{\sphinxupquote{pygace.utility.}}\sphinxbfcode{\sphinxupquote{get\_num\_lis}}}{\emph{\DUrole{n}{nb\_Nb}}, \emph{\DUrole{n}{nb\_site}}}{}
\sphinxAtStartPar
Get number list by given the number point defect and site defined in
lattice file
\begin{quote}\begin{description}
\item[{Parameters}] \leavevmode\begin{description}
\item[{\sphinxstylestrong{nb\_Nb}}] \leavevmode{[}the number of point defect{]}
\item[{\sphinxstylestrong{nb\_site}}] \leavevmode{[}int{]}
\sphinxAtStartPar
The number of site defined in lattice file

\end{description}

\item[{Yields}] \leavevmode\begin{description}
\item[{All combinations.}] \leavevmode
\end{description}

\end{description}\end{quote}

\end{fulllineitems}

\index{reverse\_dict() (in module pygace.utility)@\spxentry{reverse\_dict()}\spxextra{in module pygace.utility}}

\begin{fulllineitems}
\phantomsection\label{\detokenize{pygace:pygace.utility.reverse_dict}}\pysiglinewithargsret{\sphinxcode{\sphinxupquote{pygace.utility.}}\sphinxbfcode{\sphinxupquote{reverse\_dict}}}{\emph{\DUrole{n}{d}}}{}
\sphinxAtStartPar
Exchange \sphinxtitleref{key} and \sphinxtitleref{value} of given dict
\begin{quote}\begin{description}
\item[{Parameters}] \leavevmode\begin{description}
\item[{\sphinxstylestrong{d}}] \leavevmode{[}dict{]}
\sphinxAtStartPar
A dict needed to be converted.

\end{description}

\item[{Returns}] \leavevmode\begin{description}
\item[{Dict}] \leavevmode
\sphinxAtStartPar
The new dict in which \sphinxtitleref{key} and \sphinxtitleref{value} are exchanged with respect to
original dict.

\end{description}

\end{description}\end{quote}

\end{fulllineitems}

\index{save\_to\_pickle() (in module pygace.utility)@\spxentry{save\_to\_pickle()}\spxextra{in module pygace.utility}}

\begin{fulllineitems}
\phantomsection\label{\detokenize{pygace:pygace.utility.save_to_pickle}}\pysiglinewithargsret{\sphinxcode{\sphinxupquote{pygace.utility.}}\sphinxbfcode{\sphinxupquote{save\_to\_pickle}}}{\emph{\DUrole{n}{f}}, \emph{\DUrole{n}{python\_obj}}}{}
\sphinxAtStartPar
Save python object in pickle file.
\begin{quote}\begin{description}
\item[{Parameters}] \leavevmode\begin{description}
\item[{\sphinxstylestrong{f}}] \leavevmode{[}fileobj{]}
\sphinxAtStartPar
File object to restore python object

\item[{\sphinxstylestrong{python\_obj}}] \leavevmode{[}obj{]}
\sphinxAtStartPar
Object need to be saved.

\end{description}

\item[{Returns}] \leavevmode\begin{description}
\item[{None}] \leavevmode
\end{description}

\end{description}\end{quote}

\end{fulllineitems}



\subsection{Module contents}
\label{\detokenize{pygace:module-pygace}}\label{\detokenize{pygace:module-contents}}\index{module@\spxentry{module}!pygace@\spxentry{pygace}}\index{pygace@\spxentry{pygace}!module@\spxentry{module}}
\sphinxAtStartPar
Searching the most stable atomic\sphinxhyphen{}structure of a solid with point defects
(including the extrinsic alloying/doping elements), is one of the central issues in
materials science. Both adequate sampling of the configuration space and the
accurate energy evaluation at relatively low cost are demanding for the structure
prediction. In this work, we have developed a framework combining genetic
algorithm, cluster expansion (CE) method and first\sphinxhyphen{}principles calculations, which
can effectively locate the ground\sphinxhyphen{}state or meta\sphinxhyphen{}stable states of the relatively
large/complex systems. We employ this framework to search the stable structures
of two distinct systems, i.e., oxygen\sphinxhyphen{}vacancy\sphinxhyphen{}containing HfO(2\sphinxhyphen{}x) and the
Nb\sphinxhyphen{}doped SrTi(1\sphinxhyphen{}x)NbxO3 , and more stable structures are found compared with
the structures available in the literature. The present framework can be applied
to the ground\sphinxhyphen{}state search of extensive alloyed/doped materials, which is
particularly significant for the design of advanced engineering alloys and
semiconductors.


\chapter{Indices and tables}
\label{\detokenize{index:indices-and-tables}}\begin{itemize}
\item {} 
\sphinxAtStartPar
\DUrole{xref,std,std-ref}{genindex}

\item {} 
\sphinxAtStartPar
\DUrole{xref,std,std-ref}{modindex}

\item {} 
\sphinxAtStartPar
\DUrole{xref,std,std-ref}{search}

\end{itemize}


\renewcommand{\indexname}{Python Module Index}
\begin{sphinxtheindex}
\let\bigletter\sphinxstyleindexlettergroup
\bigletter{p}
\item\relax\sphinxstyleindexentry{pygace}\sphinxstyleindexpageref{pygace:\detokenize{module-pygace}}
\item\relax\sphinxstyleindexentry{pygace.ce}\sphinxstyleindexpageref{pygace:\detokenize{module-pygace.ce}}
\item\relax\sphinxstyleindexentry{pygace.config}\sphinxstyleindexpageref{pygace:\detokenize{module-pygace.config}}
\item\relax\sphinxstyleindexentry{pygace.ga}\sphinxstyleindexpageref{pygace:\detokenize{module-pygace.ga}}
\item\relax\sphinxstyleindexentry{pygace.gace}\sphinxstyleindexpageref{pygace:\detokenize{module-pygace.gace}}
\item\relax\sphinxstyleindexentry{pygace.general\_gace}\sphinxstyleindexpageref{pygace:\detokenize{module-pygace.general_gace}}
\item\relax\sphinxstyleindexentry{pygace.parse}\sphinxstyleindexpageref{pygace:\detokenize{module-pygace.parse}}
\item\relax\sphinxstyleindexentry{pygace.scripts}\sphinxstyleindexpageref{pygace.scripts:\detokenize{module-pygace.scripts}}
\item\relax\sphinxstyleindexentry{pygace.scripts.rungace}\sphinxstyleindexpageref{pygace.scripts:\detokenize{module-pygace.scripts.rungace}}
\item\relax\sphinxstyleindexentry{pygace.utility}\sphinxstyleindexpageref{pygace:\detokenize{module-pygace.utility}}
\end{sphinxtheindex}

\renewcommand{\indexname}{Index}
\printindex
\end{document}